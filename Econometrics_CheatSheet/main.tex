\documentclass[10pt,landscape,a4paper]{article}
\usepackage[utf8]{inputenc}

%allows you to neatly integrate an abstract once you start with the actual content
\usepackage{abstract}
\usepackage{microtype}

%%%%%%%%%%%%%%%%%%%%%%%%%%%%%%%%%%%%%%%%%%%%%%%%%%%%%%%%%%%%%%%%%%%%%%%%%%%%%%%%%%%%%%%%%% 
% SETTING THE GEOMETRY
%%%%%%%%%%%%%%%%%%%%%%%%%%%%%%%%%%%%%%%%%%%%%%%%%%%%%%%%%%%%%%%%%%%%%%%%%%%%%%%%%%%%%%%%%%

%borders and such
\usepackage[top=0mm,bottom=1mm,left=0mm,right=1mm]{geometry}
\usepackage{setspace}
\singlespacing
\usepackage{enumitem}
\setlist[itemize]{noitemsep, topsep=0pt}
\setlength{\parindent}{0pt}
\setlength{\parskip}{0pt}

%%%%%%%%%%%%%%%%%%%%%%%%%%%%%%%%%%%%%%%%%%%%%%%%%%%%%%%%%%%%%%%%%%%%%%%%%%%%%%%%%%%%%%%%%% 
% FONTS & COLORS
%%%%%%%%%%%%%%%%%%%%%%%%%%%%%%%%%%%%%%%%%%%%%%%%%%%%%%%%%%%%%%%%%%%%%%%%%%%%%%%%%%%%%%%%%% 
%if you want to start a multicolomns part inside your normally one columned article
\usepackage{multicol}
%lets you change the style of your (sub)section headers
% \usepackage{sectsty}
% %I chose to make my headers cyan-colored
% \sectionfont{\fontsize{10}{10}\selectfont}
% \subsectionfont{\fontsize{10}{10}\selectfont}
% \subsubsectionfont{\fontsize{10}{10}\selectfont}

\usepackage[compact]{titlesec}
\AtBeginDocument{%                     % this will reduce spaces between parts (above and below) of texts within a (sub)section to 0pt, for example - like between an 'eqnarray' and text
  \setlength\abovedisplayskip{0pt}
  \setlength\belowdisplayskip{0pt}}
  
\usepackage{xcolor}

\titleformat*{\section}{\normalfont\tiny\bfseries\color{blue}}
\titleformat*{\subsection}{\normalfont\tiny\bfseries\color{blue}}
\titleformat*{\subsubsection}{\tiny\color{cyan}}
\titleformat*{\paragraph}{\normalfont\tiny\bfseries\color{cyan}}
\titleformat*{\subparagraph}{\tiny}

\titlespacing{\section}{0pt}{0pt}{0pt}
\titlespacing{\subsection}{0pt}{0pt}{0pt}
\titlespacing{\subsubsection}{0pt}{0pt}{0pt}
\titlespacing{\paragraph}{0pt}{0pt}{0pt}

% superscript 1st, 2nd, 3rd, 4th ... use \nth{8}
\usepackage[super]{nth}

%%%%%%%%%%%%%%%%%%%%%%%%%%%%%%%%%%%%%%%%%%%%%%%%%%%%%%%%%%%%%%%%%%%%%%%%%%%%%%%%%%%%%%%%%% 
% INSERTING PICTURES
%%%%%%%%%%%%%%%%%%%%%%%%%%%%%%%%%%%%%%%%%%%%%%%%%%%%%%%%%%%%%%%%%%%%%%%%%%%%%%%%%%%%%%%%%% 

%very useful if you want to add a jpfg or png file. Usage: \includegraphics{path/to/file}
\usepackage{grffile}
\usepackage{graphicx}
\usepackage{subcaption}
\usepackage[format=plain,
            % labelfont=it,
            font=footnotesize,
            % width=0.85\textwidth,
            % textfont=it,
            % singlelinecheck=on
            ]{caption}
% Figures can't live outside the section
\usepackage[section]{placeins}
% and then also \FloatBarrier at the end of a section to enforce harder.

% For figures turned 90 degrees
\usepackage{rotating}
% For uppercase subfigure panels
\renewcommand{\thesubfigure}{\Alph{subfigure}}

%%%%%%%%%%%%%%%%%%%%%%%%%%%%%%%%%%%%%%%%%%%%%%%%%%%%%%%%%%%%%%%%%%%%%%%%%%%%%%%%%%%%%%%%%% 
% INSERTING TABLES
%%%%%%%%%%%%%%%%%%%%%%%%%%%%%%%%%%%%%%%%%%%%%%%%%%%%%%%%%%%%%%%%%%%%%%%%%%%%%%%%%%%%%%%%%% 

\usepackage{array}
\usepackage{tabularx}
\usepackage{longtable}

\usepackage{multirow}

%%%%%%%%%%%%%%%%%%%%%%%%%%%%%%%%%%%%%%%%%%%%%%%%%%%%%%%%%%%%%%%%%%%%%%%%%%%%%%%%%%%%%%%%%% 
% LINKS AND WWW
%%%%%%%%%%%%%%%%%%%%%%%%%%%%%%%%%%%%%%%%%%%%%%%%%%%%%%%%%%%%%%%%%%%%%%%%%%%%%%%%%%%%%%%%%% 

% Provides clickable links in the PDF-document for \ref and the TOC
\usepackage[hidelinks]{hyperref} %you can remove the hidelinks option to make the hyperrefs visible. looks very ugly to me, but might be useful to less experienced readers of pdfs
% Lets you typeset urls with linebreak. Usage: \url{http://...}  
\usepackage{url}

%%%%%%%%%%%%%%%%%%%%%%%%%%%%%%%%%%%%%%%%%%%%%%%%%%%%%%%%%%%%%%%%%%%%%%%%%%%%%%%%%%%%%%%%%% 
% CODE INTEGRATION
%%%%%%%%%%%%%%%%%%%%%%%%%%%%%%%%%%%%%%%%%%%%%%%%%%%%%%%%%%%%%%%%%%%%%%%%%%%%%%%%%%%%%%%%%% 

% Source Code Listings. Usage: \begin{lstlisting}...\end{lstlisting}
\usepackage{listings}
%this defines a lot of parameters for code. makes it look nice IMO
\lstset{ 
	breaklines=true,                 % sets automatic line breaking
	commentstyle=\color{mygreen},    % comment style
	escapeinside={\%*}{*)},          % if you want to add LaTeX within your code
	frame=single,	                   % adds a frame around the code
	keepspaces=true,                 % keeps spaces in text, useful for keeping indentation of code (possibly needs columns=flexible)
	keywordstyle=\color{blue},       % keyword style
	numbers=left,                    % where to put the line-numbers; possible values are (none, left, right)
	numbersep=5pt,                   % how far the line-numbers are from the code
	numberstyle=\tiny\color{mygray}, % the style that is used for the line-numbers
	rulecolor=\color{black},         % if not set, the frame-color may be changed on line-breaks within not-black text (e.g. comments (green here))
	showspaces=false,                % show spaces everywhere adding particular underscores; it overrides 'showstringspaces'
	showstringspaces=false,          % underline spaces within strings only
	showtabs=false,                  % show tabs within strings adding particular underscores
	stepnumber=2,                    % the step between two line-numbers. If it's 1, each line will be numbered
	stringstyle=\color{mymauve},     % string literal style
	tabsize=2,	                   % sets default tabsize to 2 spaces
	title=\lstname                   % show the filename of files included with \lstinputlisting; also try caption instead of title
}         

%%%%%%%%%%%%%%%%%%%%%%%%%%%%%%%%%%%%%%%%%%%%%%%%%%%%%%%%%%%%%%%%%%%%%%%%%%%%%%%%%%%%%%%%%% 
% MATH
%%%%%%%%%%%%%%%%%%%%%%%%%%%%%%%%%%%%%%%%%%%%%%%%%%%%%%%%%%%%%%%%%%%%%%%%%%%%%%%%%%%%%%%%%% 

%some very useful packages for writing down beautiful formulas
\usepackage{amsmath}
\usepackage{amssymb}
\usepackage{mathtools}
\usepackage{mathrsfs}  
\usepackage{bm}
\usepackage{empheq}
% Nice rules for tables. Usage \begin{tabular}\toprule ... \midrule ... \bottomrule
\usepackage{booktabs}          

%%%%%%%%%%%%%%%%%%%%%%%%%%%%%%%%%%%%%%%%%%%%%%%%%%%%%%%%%%%%%%%%%%%%%%%%%%%%%%%%%%%%%%%%%% 
% ALL ABOUT YOU
%%%%%%%%%%%%%%%%%%%%%%%%%%%%%%%%%%%%%%%%%%%%%%%%%%%%%%%%%%%%%%%%%%%%%%%%%%%%%%%%%%%%%%%%%% 


\title{Summaries}
\author{Tom Rodriguez}
\date{\today}

%start your document with \begin{document} and type everything inbetween \begin{document} and \end{document}
\let\oldcenter\center
\let\oldendcenter\endcenter
\renewenvironment{center}{\setlength\topsep{0pt}\oldcenter}{\oldendcenter}
\begin{document}
\pagenumbering{arabic}

%%%%%%%%%%%%%%%%%%%%%%%%%%%%%%%%%%%%%%%%%%%%%%%%%%%%%%%%%%%%%%%%%%%%%%%%%%%%%%%%%%%%%%%%%% 
 % ACTUAL CONTENT 
%%%%%%%%%%%%%%%%%%%%%%%%%%%%%%%%%%%%%%%%%%%%%%%%%%%%%%%%%%%%%%%%%%%%%%%%%%%%%%%%%%%%%%%%%%
\tiny

\begin{multicols*}{5}
\setlength{\columnseprule}{0.4pt}

\section{Probability Concepts}

\begin{align*}
    P(A\cup B)&=P(A) + P(B) - P(A\cap B) \\
    \mathbb{P}(A| B) &= \frac{\mathbb{P}(A \cap B)}{\mathbb{P}(B)} = \mathbb{P}(B| A)\frac{\mathbb{P}(A)}{\mathbb{P}(B)}\\
    &= \frac{\mathbb{P}(B| A)\mathbb{P}(A)}{\mathbb{P}(B| A)\mathbb{P}(A) + \mathbb{P}(B| A^c)\mathbb{P}(A^c)}
\end{align*}
\hrule
\section{Random Variables}

$F_X(x) = \mathbb{P}(X\leq x)$ and $f_X(x) = \mathbb{P}(X = x_i) $

\subsection{More than one variable}

$F_{X,Y}(x,y)=\mathbb{P}(X\leq x, Y\leq y)$

DRV: $P_{X,Y}(x_i,y_j)=P(X=x_i,Y=y_j)$

CRV: $P((X,Y)\in A) = \int\limits_{A}\int f(x,y) dx \: dy$

\paragraph{Marginal Distribution}

\begin{center}
    \resizebox{.2\textwidth}{!}{
	\begin{tabular}{ c | c }
		DRV & CRV  \\ 
		$p_x(x_i) = \sum\limits_{j} p(x_i,y_j)$ & $f_x(x) = \int\limits_{-\infty}^{\infty} f(x,y)dy $
	\end{tabular}
	}
\end{center}

\paragraph{Conditional Distribution}

\begin{center}
    \resizebox{.2\textwidth}{!}{
	\begin{tabular}{ c| c }
		DRV & CRV  \\ 
		$P(X=x_i |  Y= y_j) = \frac{P(X=x_i,Y=y_j)}{P(Y=y_j)} = \frac{P_{XY}(x_i,y_j) }{P_Y(y_j)}$ & $f_{X| Y}(x| y) =  \frac{f_{XY}(x,y)}{f_Y(y)} $
	\end{tabular}
	}
\end{center}

\paragraph{Law of total probability}

\begin{center}
    \resizebox{.2\textwidth}{!}{
	\begin{tabular}{ c| c }
		DRV & CRV  \\ 
		$P_X(x_i)=\sum\limits_{y_j} P_{X| Y}(x_i| y_j)P_Y(y_j)$ & $f_X(x_i) = \int\limits_{-\infty}^{\infty} f_{X| Y}(x_i| y_j)f_Y(y_j)dy $
	\end{tabular}
	}
\end{center}

\subsection{Expectations}

\begin{center}
    \resizebox{.2\textwidth}{!}{
	\begin{tabular}{ c| c }
		DRV & CRV  \\ 
		$\mathbb{E}[X]=\sum\limits_{i}x_ip(x_i)$ & $\mathbb{E}[X] = \int\limits_{-\infty}^{\infty}xf(x)dx$ \\
		$\mathbb{E}[g(X)]=\sum\limits_{i}g(x_i)p(x_i)$ & $\mathbb{E}[g(X)] = \int\limits_{-\infty}^{\infty}g(x)f(x)dx$
	\end{tabular}
	}
\end{center}

\begin{align*}
	\mathbb{E}[X]=\int\limits_0^\infty 1 - F(X) dx \quad\text{for any nonnegative RV}
\end{align*}


\paragraph{Conditional Expectation}

\begin{center}
    \resizebox{.2\textwidth}{!}{
	\begin{tabular}{ c| c }
		DRV & CRV  \\ 
		$\mathbb{E}[X| Y=y] = \sum\limits_{x}xp_{X| Y}(x| y)$ & $\mathbb{E}[X| Y=y]  = \int xf_{X| Y}(x| y) dx$ \\
		$\mathbb{E}[g(X)| Y=y] = \sum\limits_{x}g(x)p_{X| Y}(x| y)$  & $\mathbb{E}[g(X)| Y=y]  = \int g(x)f_{X| Y}(x| y) dx$
	\end{tabular}
	}
\end{center}

\paragraph{Law of Iterated Expectations}

\begin{align*}
	\mathbb{E}_Y[Y] &= \mathbb{E}_X[\mathbb{E}_{Y| X}(Y| X=x)]
\end{align*}

\subsection{Transformation of RVs}

\begin{align*}
    \text{Let } Y&=g(X) \\
	f_Y(y) &= f_X(g^{-1}(y)) \cdot | \frac{d}{dy}g^{-1}(y)|  \\
	f_Y(y) &= f_X(g^{-1}(y)) \cdot | J| 
\end{align*}

\subsection{Moments}

\begin{align*}
    M(t) &= \mathbb{E}(e^{tX}) \quad\text{mgf} \\
    M^{(j)}(t) &= \int x^j e^{tx} f_X(x) dx \\
    M^{(j)}(0) &= \mathbb{E}(X^j) \\
\end{align*}

\subsection{(Co)variance and Correlation}

\begin{center}
    \resizebox{.2\textwidth}{!}{
	\begin{tabular}{ c| c }
		DRV & CRV  \\ 
		$Var(X) = \sum\limits_{i}(x_i-\mu)^2p(x_i)$ & $Var(x) = \int\limits_{-\infty}^{\infty}(x-\mu)^2f(x)dx$
	\end{tabular}
	}
\end{center}

\begin{align*}
    Var(X) &= \mathbb{E}[(X-\mathbb{E}[X])^2] = \mathbb{E}[X^2] - (\mathbb{E}[X])^2 \\
	Cov(X,X) &= \mathbb{E}[(X-\mu_X)(Y-\mu_Y)] \\
	&= \mathbb{E}[XY] - \mathbb{E}[X]\mathbb{E}[Y] \\
	\mathbb{E}[XY] &= \int \int xyf(x,y) \: dx \: dy \\
	Cov(X,Y) &= \int\limits_{-\infty}^{\infty}\int\limits_{-\infty}^{\infty}(x-\mu_X) (y-\mu_Y) \\
	&\cdot f(x,y) \: dx \: dy \\
	\rho &= \frac{Cov(X,Y)}{\sqrt{Var(X)Var(Y)}}
\end{align*}
\hrule
\section{Selected Probability Distributions}

\begin{minipage}{.09\textwidth}
  \subsection{Binomial}

    \begin{align*}
    	p(k)&={{n}\choose{k} }p^k (1-p)^{n-k} \\
    	E[X]&=np \\
    	Var(X)&=np(1-p) \\
    	M(t)&=(1-p+pe^{t})^{n}
    \end{align*}

Note: if $n=1$, it's a Bernoulli distribution.
\end{minipage}% This must go next to `\end{minipage}`
\begin{minipage}{.09\textwidth}
  \subsection{Poisson}

    \begin{align*}
    	p(k)&=\frac{\lambda^ke^{-\lambda}}{k!} \\
    	E[X]&=\lambda \\
    	Var(X)&=\lambda \\
    	M(t)&=e^{\lambda(e^t-1)}
    \end{align*}
\end{minipage}

\begin{minipage}{.07\textwidth}
  \subsection{Uniform}

\begin{align*}
	f(x) &= \frac{1}{b-a} \\
	E[X]&=\frac{1}{2} (a+b) \\
	Var(X)&=\frac{1}{12} (b-a)^2 \\
	M(t)&=\frac{e^{bt} - e^{at}}{(b-a)t}
\end{align*}
\end{minipage}% This must go next to `\end{minipage}`
\begin{minipage}{.11\textwidth}
  \subsection{Univariate Normal}

\begin{align*}
	f(x) &= \frac{1}{\sigma \sqrt{2\pi}}e^{-\frac{1}{2\sigma^2}(x-\mu)^2 } \\
	E[X]&=\mu \\
	Var(X)&=\sigma^2  \\
	M(t)&=e^{\mu t}e^{\frac{\sigma^2 t^2}{2}} \\
\end{align*}
\end{minipage}

\begin{minipage}{.09\textwidth}
  \paragraph{$\chi^2$ Distribution}

\begin{align*}
	U&:=\sum\limits_{i=1}^{n}Z_i^2 \\
	U& \sim \chi^2_n \\
	E[U]&=n , \forall n \geq 1\\
	Var(U) &= 2n , \forall n \geq 1
\end{align*}
\end{minipage}% This must go next to `\end{minipage}`
\begin{minipage}{.09\textwidth}
  \paragraph{t Distribution}

\begin{align*}
	T&:= \frac{Z}{\sqrt{U/n}}\\
	T& \sim t_n \\
	E[T]&=0 , \forall n \geq 2\\
	Var(T) &= \frac{n}{n-2} , \forall n \geq 3
\end{align*}
\end{minipage}

\paragraph{F Distribution}

\begin{align*}
	W&:= \frac{U/m}{V/n} &	&\\
	W& \sim F_{m,n} &	&\\
	E[W]&=\frac{n}{n-2}&	&\qquad \forall n \geq 3\\
	Var(W) &= \frac{2n^2(m+n-2)}{m(n-2)^2(n-4)}  &	&\qquad \forall n \geq 5
\end{align*}

\subsection{Multivariate Normal}

\paragraph{Notes on matrix algebra:}
Let $\Sigma$ be a positive definite matrix. Then it can be factored as $\Sigma = AA'$. $A=\Sigma^{1/2}$, and then $\Sigma^{-1}=(A')^{-1}A^{-1}$.
$| A| ^{-1}=| A^{-1}| $ and $| A| =| A'| $.

\begin{align*}
	f(x) &= \frac{1}{\sigma \sqrt{2\pi}}e^{-\frac{1}{2\sigma^2}(x-\mu)^2 } \\
	E[X]&=\mu \\
	Var(X)&=\sigma^2  \\
	M(t)&=e^{\mu t}e^{\frac{\sigma^2 t^2}{2}}
\end{align*}

\paragraph{Theorems}

\begin{center}
    \resizebox{.19\textwidth}{!}{
    \begin{tabular}{l|l}
        A &  Linear functions of X are normally distributed\\
          &  $Y=\eta+B X \sim N_k\left(\eta+B \mu, B \Sigma B^{\prime}\right)$\\
          \hline
        B &  $X$ has density given by\\
          &  $f_X(x)=\frac{1}{(2 \pi)^{p / 2}| \Sigma| ^{1 / 2}} \exp \left\{-\frac{1}{2}(x-\mu)^{\prime} \Sigma^{-1}(x-\mu)\right\}$\\
          \hline
        C &  Independent normally distributed RVs are jointly normal.\\
          &  $X=(X_1^\prime , X_2^\prime)^\prime \sim N_{p+q}\left(\mu, \Sigma\right)$\\
          &  $\mu=\left(\begin{array}{l} \mu_1 \\ 
             \mu_2 \end{array}\right) \quad \text { and } \quad \Sigma=\left(\begin{array}{cc} \Sigma_1 & 0 \\
             0 & \Sigma_2 \end{array}\right)$\\
          \hline
        D &  Conditional normal distribution\\
          &  $\left(X_1 | X_2=x_2\right) \sim N \left(\mu_1+\Sigma_{12} \Sigma_{22}^{-1}\left(x_2-\mu_2\right), \Sigma_{11}-\Sigma_{12} \Sigma_{22}^{-1} \Sigma_{21}\right)$\\
          \hline     
        E &  Suppose $X_2 \sim N \left(\mu_2, \Sigma_{22}\right)$ and $X_1 | X_2=x_2 \sim N \left(A+B x_2, \Omega\right)$.\\
        &Then $X=(X_1^\prime , X_2^\prime)^\prime$ has a multivariate normal distribution\\
          &  $\left(\begin{array}{c} X_1 \\ X_2 \end{array}\right)
          \sim N \left(\left(\begin{array}{c} A+B \mu_2 \\ \mu_2 \end{array}\right),
          \left(\begin{array}{cc} B \Sigma_{22} B^{\prime}+\Omega & B \Sigma_{22} \\ \Sigma_{22} B^{\prime} & \Sigma_{22} \end{array}\right)\right)$\\
          \hline
        F & Sums of independent normals \\
          &  $X_1+X_2 \sim N \left(\mu_1+\mu_2, \Sigma_1+\Sigma_2\right)$\\
          \hline
    \end{tabular}
    }
\end{center}

Let $X \sim N_p(\mu, \Sigma)$. Also let $X=(X_1^\prime, X_2^\prime)^\prime$, $\mu=(\mu_1^\prime, \mu_2^\prime)^\prime$, and $\Sigma=\left(\begin{array}{cc} \Sigma_{11} & \Sigma_{12} \\ \Sigma_{21} & \Sigma_{22} \end{array}\right)$.

\begin{center}
    \resizebox{.19\textwidth}{!}{
    \begin{tabular}{l|l}
        G &  The marginal distribution of $X_1$ is $N_k(\mu_1, \Sigma_{11})$\\
          \hline
        H &  For a normal, a zero correlation implies independence.\\
          \hline
        I &  Characterizing independence of linear combinations of \\
          & normal variables. \\
          &  If $X \sim N_p(\mu, \Sigma)$, $B$ is a $p\times k$ matrix, and \\
          & $C$ is a $p\times m$ matrix, then $B\prime X$ and $C^\prime X$\\
          & are independent iff $B^\prime\Sigma C = 0$.\\
          & Note that $B^\prime X$ and $C^\prime X$ are jointly normal and\\ 
          & $B^\prime\Sigma C$ is the covariance.\\
          \hline
    \end{tabular}
    }
\end{center}

Quadratics: Assume $A$ is symmetric, then $Y\prime AY$ is a quadratic form.

\begin{center}
    \resizebox{.19\textwidth}{!}{
    \begin{tabular}{l|l}
        J & If $X \sim N_p(\mu, \Sigma)$ where $\Sigma$ has rank $p$,\\
          &then $(X-\mu)^\prime\Sigma^{-1}(X-\mu)\sim\chi_p^2$. \\
          \hline
        K & Let $M$ denote an idempotent $p\times p$ matrix with rank $k$,\\ 
          & then $Z^\prime MZ \sim\chi_k^2$\\
          &  $M=P\Lambda P^\prime$, where $\Lambda$ contains the eigenvalues of $M$\\ 
          & on the diagonal, \\
          & and the rows of $P$ are the orthonormal eigenvectors.\\
          & Then $M=\left[ P_1 \: P_2 \right] \left[ \begin{array}{cc} I_k & 0 \\
        0 & 0 \end{array} \right] \left[ \begin{array}{c} P_1^\prime \\
        P_2^\prime \end{array} \right] = P_1 P_1^\prime$ \\
          & Thus, $P_1^\prime Z \sim N(0,P_1^\prime P_1)$, where $P_1^\prime P_1 = I_k$\\
          \hline
        L & Let $X=PZ$, and $Q=Z^\prime AZ$, where $PA=0$, \\ 
          & then $X$ and $Q$ are independent. \\
          \hline
        M & Let $Q_1 = Z^\prime A_1 Z$, and $Q_2 = Z^\prime A_2 Z$, where $A_1 A_2=0$.\\ 
          & Then $Q_1$ and $Q_2$ are independent.\\
          \hline
    \end{tabular}
    }
\end{center}
\hrule
\section{Some Useful Inequalities}

\subsection{Jensen's Inequality}

\begin{center}
	\begin{tabular}{ c| c }
		$g(x) $ concave & $g(x)$ convex  \\ 
		$E[g(x)]\leq g(E[X]) $& $E[g(x)]\geq g(E[X]) $
	\end{tabular}
\end{center}

\subsection{Chebyshev's and Markov's Inequality}

If $X$ is a random Variable with mean $\mu$ and variance $\sigma^2$, then 

\begin{align*}
	P(| X-\mu|  \geq \varepsilon) &\leq \frac{\sigma^2}{\varepsilon^2} &\qquad \forall \varepsilon > 0 \\
	P(| X|  \geq \varepsilon) &\leq \frac{\mathbb{E}(| X| ^p)}{\varepsilon^p} &\qquad \forall \varepsilon > 0 \: \text{(Markov)}
\end{align*}
\hrule
\section{Large Sample Theory}

$X_n \stackrel{a s}{\longrightarrow} X \text { if } P \left\{\omega | \lim _{n \rightarrow \infty} X_n(\omega)=X(\omega)\right\}=1$

$X_n \stackrel{p}{\longrightarrow} X \text { if } \forall \varepsilon>0, \lim\limits_{n\rightarrow \infty} \mathbb{P}(|X_n-X|\geq\varepsilon)=0$

$X_n \stackrel{ms}{\longrightarrow} X \text { if } \lim\limits_{n\rightarrow \infty} \mathbb{E}\left[(X_n - X)^2\right]=0$

$X_n \stackrel{d}{\rightarrow} X \text { if } \lim _{n \rightarrow \infty} F_{X_n}(x)=F_X(x)$

\paragraph{Relationship between convergences}

\begin{itemize}
    \item If $X_n \stackrel{a.s.}{\longrightarrow} X$ then $X_n \stackrel{p}{\longrightarrow} X$
    \item If $X_n \stackrel{ms}{\longrightarrow} X$ then $X_n \stackrel{p}{\longrightarrow} X$
    \item If $X_n \stackrel{p}{\longrightarrow} X$ then $X_n \stackrel{d}{\longrightarrow} X$
\end{itemize}

\paragraph{Slutsky's Theorem}

If \\

$X_n \xrightarrow[]{% above
		\substack{D}
	} X \in \mathbb{R}^k, \qquad \text{where $X$ can be random}$

$Y_n \xrightarrow[]{% above
		\substack{P}
	} A \in \mathbb{R}^p, \qquad \text{where $A$ is fixed}$

$Z_n \xrightarrow[]{% above
		\substack{P}
	} B \in \mathbb{R}^{p \times k}, \qquad \text{where $B$ is fixed}$

Then, $Y_n + Z_n X_n \xrightarrow[]{% above
		\substack{D}
	} A + BX$

% \paragraph{Continuous Mapping Theorem}

% \begin{align*}
% 	X_n \xrightarrow[]{% above
% 		\substack{\text{a.s.}\\P\\D}
% 	} X \Rightarrow
% 	g(X_n) \xrightarrow[]{% above
% 		\substack{\text{a.s.}\\P\\D}
% 	} g(X) 
% \end{align*}

\subsection{Law of Large Numbers}

\paragraph{The sample mean}

By LLN: $\Bar{X} \stackrel{a}{\sim} N\left( \mu, \frac{\sigma^2}{n} \right)$

\paragraph{weak LLN}

Let $X_1, X_2, \dots$ be a sequence of random variables with $\mathbb{E}(X_i) = \mu$, and $\text{Var}(X_i)=\sigma^2$, and $\text{Cov}(X_i, X_j)=0 \: \forall i\neq j$. Then one can use Chebyshev to show that:

\begin{align*}
    \Bar{X} \stackrel{p}{\longrightarrow} \mu
\end{align*}

\paragraph{strong LLN}

Let $X_1, X_2, \dots$ be $i.i.d.$ with $\mathbb{E}(X_i) = \mu < \infty$, then without saying anything about \nth{2} moments:

\begin{align*}
    \Bar{X} \stackrel{a.s.}{\longrightarrow} \mu
\end{align*}

\subsection{Central Limit Theorem}
Let $Y_1,Y_2,...$ be a sequence of $k$-dimensional i.i.d. random vectors with $E[Y_i]=\mu_Y$ and $Var(Y_i) = \Sigma$.

\begin{align*}
	\sqrt{n} \Sigma^{-\frac{1}{2}}(\bar{Y}_n- \mu) &\xrightarrow[]{% above
		\substack{d}
	} N(0,I_k) \\
	\lim\limits_{n \rightarrow \infty} P (\sqrt{n}\Sigma^{-\frac{1}{2}}(\bar{Y}_n - \mu) \leq y) &= \Phi_k(y), \qquad \forall y \in \mathbb{R}^k
\end{align*}

\paragraph{Delta Method}

Let $U_n$ denote a sequence of scalar random variables, and let $V_n = \sqrt{n}(U_n)-a$, where $a$ is a constant.
Let $g(\cdot)$ be a continuously differentiable function. Suppose $V_n \stackrel{p}{\longrightarrow} V \sim N(\mu, \sigma^2)$. Then

\begin{align*}
    \sqrt{n}\left(g\left(U_n\right)-g(a)\right) &\Rightarrow \frac{d g(a)}{d a} V \sim N \left(0,\left[\frac{d g(a)}{d a}\right]^2 \sigma^2\right) \\
	\sqrt{n}\left(g\left(U_n\right)-g(a)\right) & \\
	\Rightarrow \frac{d g(a)}{d a} V &\sim N \left(0,\left[\frac{d g(a)}{d a}\right] \Sigma\left[\frac{d g(a)}{d a}\right],\right)
\end{align*}
\hrule
\section{Estimators}

\begin{itemize}
    \item An estimator is unbiased, if $\mathbb{E}(\hat{\theta})=\theta$. Where Bias is defined as $\operatorname{Bias}(\hat{\theta})= \mathbb{E}(\hat{\theta})-\theta$
    \item Loss Function , say $L(\hat{\theta}, \theta)=(\hat{\theta}-\theta)^2=\text { quadratic loss }$. This is not the same as expected quadratic loss, which is MSE:
    
    \begin{align*}
        E (L(\hat{\theta}, \theta))&= E \left((\hat{\theta}-\theta)^2\right)=\operatorname{mse}(\hat{\theta}) \\
        &= \operatorname{Var}(\hat{\theta})+[\operatorname{Bias}(\hat{\theta})]^2
    \end{align*}
    \item Conclusion: for unbiased estimator $\operatorname{mse}(\hat{\theta})=\operatorname{Var}(\hat{\theta})$
    \item An estimator is consistent, if $\hat{\theta} \stackrel{p}{\rightarrow} \theta$
\end{itemize}

\subsection{The Likelihood Function}

\begin{align*}
    L (\theta, Y)&=f(Y | \theta) \\
    S(\theta, y)&=\frac{\partial \ln f(y | \theta)}{\partial \theta}=\frac{1}{f(y | \theta)} \frac{\partial f(y | \theta)}{\partial \theta}\\
    E [S(\theta, Y)]&=\int \frac{\partial f(y | \theta)}{\partial \theta} d y=\int S(\theta, y) f(y | \theta) d y= 0\\
    I (\theta)&=- E \left[\frac{\partial S(\theta, Y)}{\partial \theta}\right] = E \left[S(\theta, Y)^2\right]\\
    &=\operatorname{Var}[S(\theta, Y)]
\end{align*}

\paragraph{The Cramer-Rao inequality and unbiased estimators}

\begin{align*}
    \operatorname{Var}(\hat{\theta}) &\geq \frac{1}{\operatorname{Var}(S(\theta, Y))}= I (\theta)^{-1}
\end{align*}

\paragraph{Maximum Likelihood Estimators}

\begin{align*}
    \max\limits_{\theta} L _n(\theta)&= \max\limits_{\theta} \prod_{i=1}^n f\left(Y_i | \theta\right) \\
    \widehat{\theta}_{m l e} &\stackrel{p}{\rightarrow} \theta_0 \\
    I \left(\theta_o\right)^{1 / 2} \sqrt{n}\left(\widehat{\theta}_{m l e}-\theta_0\right) &\stackrel{d}{\rightarrow} N(0, I)\\
    \widehat{\theta}_{m l e} &\stackrel{a}{\sim} N\left(\theta_0, n^{-1} I \left(\theta_0\right)^{-1}\right)
\end{align*}

\subsection{Method of Moment Estimators}

Assume $\mu = h(\theta_0)$, where $\mu$ is $l\times 1$, $\theta_0$ is $k\times 1$ with $k\leq l$. Then $\widehat{\theta}_{m m}$ solves

\begin{align*}
    \min\limits_{\theta} J_n(\theta)&=\min\limits_{\theta} (\bar{Y}-h(\theta))^{\prime}(\bar{Y}-h(\theta))\\
    \widehat{\theta}_{m m} &\stackrel{p}{\rightarrow} \theta_0 \\
    \widehat{\theta}_{m m} &\stackrel{a}{\sim} N\left(\theta_0, V\right) \\
    V &= n^{-1} H^{-1}\left[\frac{\partial h\left(\theta_o\right)}{\partial \theta^{\prime}}\right]^{\prime} \Sigma\left[\frac{\partial h(\theta)}{\partial \theta^{\prime}}\right] H^{-1} \\
    H&=\left[\frac{\partial h\left(\theta_o\right)}{\partial \theta^{\prime}}\right]^\prime \left[\frac{\partial h\left(\theta_o\right)}{\partial \theta^{\prime}}\right]
\end{align*}
\hrule
\section{Sufficient Statistics}

Pdf of $Y$ as $f_Y(y | \theta)$, the pdf of $S$ as $f_S(s | \theta)$ and the conditional pdf of $Y$ given $S=s$ as $f_{Y | S}(y| s, \theta)=f_{Y| S}(y| s)$, that is the conditional density of $Y$ given $S$ does not depend on $\theta$.

\paragraph{Factorization Theorem}

\begin{align*}
    \hat{\theta}(Y)=\arg \max _\theta f(Y | \theta)=\arg \max _\theta g(S | \theta)=\hat{\theta}(S)
\end{align*}

\paragraph{Rao-Blackwell Theorem}

$Y$ is RV with mean $\mu$ and variance $\sigma_Y^2$. $X$ is another RV. Let $\mu(x)=\mathbb{E}[Y| X=x]$. Then $\operatorname{Var}(\mu(x))\leq\sigma_Y^2$:

\begin{align*}
    E (\mu(X))&=\mu \\
    Y&=\mu(X)+(Y-\mu(X)) \\
    \sigma_Y^2&=\operatorname{Var}(\mu(X))+\operatorname{Var}(Y-\mu(X)) \\
    \operatorname{Var}(\mu(X)) &\leq \sigma_Y^2
\end{align*}

Use this result in estimations: 
Suppose $\hat{\theta}(Y)$ is an unbiased estimator of $\theta$, so that $\theta=\mathbb{E}[\hat{\theta}(Y)]$, and let $S$ be a sufficient statistic for $\theta$. Then $E [\tilde{\theta}(S)]$ is an unbiased estimator of $\theta$ but the variance is lower by Rao-Blackwell.

\begin{align*}
    \theta&= E [\hat{\theta}(Y)]= E [ E [\hat{\theta}(Y) | S]]= E [\tilde{\theta}(S)]
\end{align*}


\hrule
\section{Hypothesis Tests}

\subsection{Wald Tests}

\begin{align*}
    H_0: & \theta = \theta_0 \quad H_a: \theta \neq \theta_0 \\
    \xi&=\left(\hat{\theta}-\theta_0\right)^{\prime} \Omega^{-1}\left(\hat{\theta}-\theta_0\right) \\
    \xi &\sim \chi_k^2 \quad \text{under the null}
\end{align*}

Therefore, we accept $H_0$, if $\xi\leq \operatorname{cv}$, and reject $H_0$ if $\xi > \operatorname{cv}$. $\operatorname{cv}$ solves $\mathbb{P}(\xi>\operatorname{cv}| \theta = \theta_0)=\alpha$.

$\operatorname{Power} = \mathbb{P}(\xi>\operatorname{cv}| H_a \text{ is true}) = \mathbb{P}(\xi>\operatorname{cv}| \theta \neq \theta_0)$. Because $H_a$ has many values for $\theta$, the power differs for each value. But assuming that the distribution of $\hat{\theta}$ was based on CLT ($\sqrt{n}(\hat{\theta}-\theta) \stackrel{d}{\longrightarrow} N (0, V)$). Then $\Omega=n^{-1} V$, and our test statistic becomes

\begin{align*}
    \xi&=n\left(\hat{\theta}-\theta_0\right)^{\prime} V^{-1}\left(\hat{\theta}-\theta_0\right)
\end{align*}

If the mean of $\hat{\theta}$ is equal to a fixed constant (that is not $\theta_0$), then $\xi \rightarrow\infty$, and $\mathbb{P}(\xi>\operatorname{cv})\rightarrow 1$. The test has therefore power $=1$ for any fixed value of $\theta$ under the alternative. When power $\rightarrow 1$, a test is said to be consistent.

\paragraph{Hypotheses involving linear functions of $\theta$}

\begin{align*}
    H_0: R\theta &= r_0 \quad \text{where $R$ is a $j\times k$ matrix with rank $j$}\\
    H_a: R\theta &\neq r_0 \quad \text{where $R$ is a $j\times k$ matrix with rank $j$}\\
    R \hat{\theta} &\sim N\left(R \theta, R \Omega R^{\prime}\right) \\
    \xi&=\left(R \hat{\theta}-r_0\right)^{\prime}\left(R \Omega R^{\prime}\right)^{-1}\left(R \hat{\theta}-r_0\right) \\
    \xi&\sim\chi_j^2 \quad \text{under } H_0
\end{align*}

\paragraph{Hypotheses involving nonlinear functions of $\theta$}

\begin{align*}
    H_0: R(\theta) &= r_0 \\
    H_a: R(\theta) &\neq r_0 \\
    \sqrt{n}(R(\hat{\theta})-R(\theta)) &\Rightarrow N \left(0, H V H^{\prime}\right) \quad \text{by delta method}\\
    H&=\frac{\partial R(\theta)}{\partial \theta^{\prime}} \\
    R(\hat{\theta}) &\stackrel{a}{\sim} N (R(\theta), \tilde{\Omega}) \text{ where } \tilde{\Omega}=n^{-1} H V H^{\prime} \\
    \xi&=\left(R(\hat{\theta})-r_0\right)^{\prime} \tilde{\Omega}^{-1}\left(R(\hat{\theta})-r_0\right)
\end{align*}

\subsection{Neyman-Pearson Tests}

We choose the probability of a type 1 error (accept $H_a$ when $H_0$) beforehand (size), and then minimize the probability of making a type 2 error (maximize power).

\paragraph{Neyman-Pearson Lemma} 
Choose critical region based on the LR to maximize power.

\begin{align*}
    L R(Y)&=\frac{ L _a(Y)}{ L _o(Y)} \\
    W&=\{y | L R(y)> cv \} \\
    P \left[L R(Y)> cv | Y \sim F_o\right]&=\alpha \text{ to determine cv}
\end{align*}

We reject $H_0: \mu=\mu_0$, if LR is large when $H_a: \mu>\mu_0$. We reject $H_0: \mu=\mu_0$, if LR is small when $H_a: \mu<\mu_0$.
Since the LR critical regions are the same for all of the simple hypotheses making up $H_a$ and each is most powerful, then the LR procedure is said to be Uniformly Most Powerful (UMP) for $H_0$ vs. $H_a$.

\subsection{Maximizing Weighted Average Power}

\begin{align*}
    H_o: \theta=\theta_0 &\text{ simple null hypothesis}\\
    H_a: \theta\in \Theta_a &\text{ composite alternative hypothesis}\\
    w(\theta) &\text{ weight function for values of } \theta\in\Theta_a \\
    f(y | \theta) &\text{ density of $y$, conditional on a value of } \theta \\
    \int_W f(y | \theta) d y &\text{ power of the test for a particular } \theta\\
    WAP &=\int_{\Theta_{a}}\left[\int_W f(y | \theta) d y\right] w(\theta) d \theta \\
    &=\int_W\left[\int_{\Theta_a} f(y | \theta) w(\theta) d \theta\right] d y\\
    &=\int_W g(y) d y
\end{align*}

$g(Y)$ is the density of $Y$ under the assumption that $\theta$ is random with density $w(\theta)$. Thus, the problem is equivalent to testing with a simple alternative $\tilde{H}_a: y\sim g(y)$. Then use NP test:

\begin{align*}
    L R(Y)&=\frac{g(Y)}{f\left(Y | \theta_o\right)}=\frac{\int_{\Theta_a} f(Y | \theta) w(\theta) d \theta}{f\left(Y | \theta_o\right)}
\end{align*}
\hrule
\section{Confidence Sets}
\begin{align*}
    C(Y)&=\left\{\theta||(\widehat{\theta}-\theta)^{\prime}\left[\frac{1}{n} \widehat{V}\right]^{-1}(\widehat{\theta}-\theta) \leq \chi_{\kappa, 1-\alpha}^2\right\} \\
    C(Y)&=\left\{\widehat{\theta}\pm Z_{1-\alpha / 2} \times \sqrt{\frac{1}{n} \widehat{V}} \right\}
\end{align*}
\hrule
\section{The Bayes Approach to Estimation and Inference}

\subsection{Basic concepts and some jargon}

\begin{align*}
    f_{\theta | Y}(\tilde{\theta} | y)=\frac{f_{Y, \theta}(y, \tilde{\theta})}{f_Y(y)}&=\frac{f_{Y | \theta}(y | \tilde{\theta}) f_\theta(\tilde{\theta})}{\int f_{Y, \theta}(y, \tilde{\theta}) d \tilde{\theta}}\\
    &=\frac{f_{Y | \theta}(y | \tilde{\theta}) f_\theta(\tilde{\theta})}{\int f_{Y | \theta}(y | \tilde{\theta}) f_\theta(\tilde{\theta}) d \tilde{\theta}} \\
    \text { Posterior }&=\frac{\text { Likelihood } \times \text { Prior }}{\text { Marginal Likelihood }}
\end{align*}

\subsection{Bayes Estimators}

\begin{align*}
    \min _{\hat{\theta}} \mathbb{E} _{\theta | Y=y}[L(\hat{\theta}, \theta)] \quad\text{posterior risk}
\end{align*}

If loss is quadratic, then we know that $\hat{\theta}= E _{\theta | Y=y}(\theta)$ minimizes the MSE, which is just the posterior mean.

\subsection{Bayes Credible Sets}

\begin{align*}
    P _{\theta | Y=y}(\theta \in C(y))=1-\alpha
\end{align*}

where the notation emphasizes that the probability is computed using the posterior for $\theta|Y=y$. Because this holds for all $y$, we also have

\begin{align*}
    P _{\theta, Y}(\theta \in C(Y))=1-\alpha
\end{align*}

where now the probability is computed over $\theta$ and $Y$.
\hrule

\end{multicols*}
\newpage

\begin{multicols*}{5}
\setlength{\columnseprule}{0.4pt}

\section{Hayashi 1}

\subsection{Assumptions}

\subsubsection{Linearity}

$y=X \beta+\varepsilon, X$ is $(n\times k) \beta$ is $(k\times1)$

\subsubsection{Strict Exogeneity}

\begin{align*}
    E\left(\varepsilon_i \mid X\right)=0 \\
    E\left(\varepsilon_i\right)=E\left[E\left(\varepsilon_i \mid X\right)\right]=0
\end{align*}

\subsubsection{Full Rank}

No multicollinearity. $(XX^\prime)$ has to be invertible.

\subsubsection{Homoskedasticity}

\begin{align*}
    E\left(\varepsilon_i^2 \mid X\right)=\sigma^2>0 \\
    E\left(\varepsilon_i \varepsilon_j \mid X\right)=0 \text { for } \mathrm{i} \neq \mathrm{j}
\end{align*}

\subsubsection{Normality}

\begin{align*}
    \varepsilon \mid X \sim N\left(0, \sigma^2 I\right) \\
    \frac{\varepsilon}{\sigma} \mid X \sim N(0, I)
\end{align*}

for inference

\hrule

\subsection{OLS estimation}

\begin{align*}
    \widehat{\beta}_{O L S}&=\left(X^{\prime} X\right)^{-1} X^{\prime} y=\left(\frac{1}{n} X^{\prime} X\right)^{-1}\left(\frac{1}{n} X^{\prime} y\right)\\
    &=\left(\frac{1}{n} \sum_{i=1}^n x_i x_i^{\prime}\right)^{-1}\left(\frac{1}{n} \sum_{i=1}^n x_i y_i\right)
\end{align*}

for one regressor: $\widehat{\beta}_1=\frac{\widehat{\operatorname{Cov}}(x,y)}{\widehat{\operatorname{Var}}(x)}$, and $\widehat{\beta}_0 = \bar{y} - \widehat{\beta}_1 \bar{x}$

\begin{align*}
    \hat{y}=X b \\
    P \equiv X\left(X^{\prime} X\right)^{-1} X^{\prime} \\
    M \equiv I-P \\
    \hat{y}=P y=X\left(X^{\prime} X\right)^{-1} X^{\prime} y=X b \\
    e=M y=y-P y=y-\hat{y} \\
    P X=X \\
    M X=0 \\
    e=M y=M(X \beta+\varepsilon)=M \varepsilon
\end{align*}

\hrule

\subsection{Finite sample properties}

OLS estimator is unbiased with following variance:

\begin{align*}
    \widehat{\beta}_{\text{OLS}}&=\left(X^{\prime} X\right)^{-1} X^{\prime}(X \beta+\varepsilon)\\
    &=\beta+\left(X^{\prime} X\right)^{-1} X^{\prime} \varepsilon\\
    E[\hat{\beta}]&=\beta+\left(X^{\prime} X\right)^{-1} X^{\prime} E[\varepsilon \mid x]=\beta \\
    V[\hat{\beta} \mid X]&=E\left[(\hat{\beta}-E[\hat{\beta}])(\hat{\beta}-E[\hat{\beta}])^{\prime} \mid X\right] \\
    &= \left(X^{\prime} X\right)^{-1} X^{\prime} E\left[\varepsilon \varepsilon^{\prime} \mid X\right] X\left(X^{\prime} X\right)^{-1} \\
    &=\sigma^2\left(X^{\prime} X\right)^{-1}
\end{align*}

CR lower bound (achieved by MLE): $\frac{2\sigma^4}{n}$.

\subsubsection{Misspecification}

Two models:

\begin{align*}
    A:&\quad y=X \beta+Z \gamma+\epsilon \\
    B:&\quad y=X \beta+\epsilon
\end{align*}

\paragraph{OVB:}

$A$ is true, $B$ is false. Then $\hat{\beta}$ is biased.

\begin{align*}
    \hat{\beta}=\beta+\underbrace{\left(X^{\prime} X\right)^{-1} X^{\prime} Z \gamma}_{\text {Bias }}+\left(X^{\prime} X\right)^{-1} X^{\prime} \varepsilon
\end{align*}

\paragraph{Irrelevant variable:}

$B$ is true, $A$ is false. Then $\hat{\beta}$ is unbiased, but inefficient.

\begin{align*}
    V(\hat{\beta} \mid X, Z) \geq \sigma^2\left(X^{\prime} X\right)^{-1}=V(\hat{\beta} \mid X)
\end{align*}

\hrule

\hrule
\section{Hayashi 2}

\subsection{Inference}

\begin{align*}
    t_j = \frac{\widehat{\beta}_j-\beta_j}{\operatorname{se}\left(\widehat{\beta}_j\right)} \\
    \beta^0 \in \widehat{\beta}_j \pm t_{n-K}^\alpha \operatorname{se}\left(\widehat{\beta}_j\right)
\end{align*}

\subsubsection{Linear restrictions}

\begin{align*}
    F &= \frac{\left(R^{\prime} \widehat{\beta}-r\right)^{\prime}\left[R^{\prime}\left(X^{\prime} X\right)^{-1} R\right]^{-1}\left(R^{\prime} \widehat{\beta}-r\right) / p}{\left((n-k) s^2 / \sigma^2\right) /(n-k)} \\
    &= \frac{\left(e^{\prime *} e^*-e^{\prime} e\right) / p}{e^{\prime} e /(n-K)}=\frac{\left(S S R^*-S S R\right) / p}{S S R /(n-K)} \\
    &= \frac{\left(R^2-R^{* 2}\right) / p}{\left(1-R^2\right) /(n-K)} =\frac{\left(S S T^*-S S R\right) / p}{S S R /(n-K)}
\end{align*}

$p$: number of regressors w/o constant \newline
$n-K$: number of individuals minus number of regressors with constant
\newline
last equation only for regression output useful

\subsubsection{Goodness of Fit}

\begin{align*}
    R^2&=\frac{\sum\left(\widehat{y}_i-\bar{y}\right)^2}{\sum\left(y_i-\bar{y}\right)^2}=M S S / T S S \\
    &=1-\frac{\sum e_i^2}{\sum\left(y_i-\bar{y}\right)^2}=1-(R S S / T S S) \\
    \hat{R}^2&=1-\frac{\sum e_i^2 /(n-K)}{\sum\left(y_i-\bar{y}\right)^2 /(n-1)} \\
    & = 1-\left(1-R^2\right) \frac{n-1}{n-k}
\end{align*}

\subsubsection{Wald Statistic}

$V(\widehat{\beta} \mid X)$ is given in outputs. Do not forget to square $\operatorname{SE}(\hat{\beta})$.

\begin{align*}
    W &=\left(R^{\prime} \widehat{\beta}-r\right)^{\prime}\left[R^{\prime} V(\widehat{\beta} \mid X) R\right]^{-1}\left(R^{\prime} \widehat{\beta}-r\right)\\
    &=n\left(R^{\prime} \widehat{\beta}-r\right)^{\prime}\left[R^{\prime} \hat{\operatorname{Avar}}(\widehat{\beta}) R\right]^{-1}\left(R^{\prime} \widehat{\beta}-r\right)\\
    W&=a(\hat{\beta})^{\prime}\left[\nabla a(\hat{\beta})^{\prime} \hat{V}(\hat{\beta}\mid X) \nabla a(\hat{\beta})\right]^{-1} a(\hat{\beta}) \\
    &= n a(\hat{\beta})^{\prime}\left[\nabla a(\hat{\beta})^{\prime} \hat{\operatorname{Avar}}(\hat{\beta}) \nabla a(\hat{\beta})\right]^{-1} a(\hat{\beta})
\end{align*}


\hrule
\section{Hayashi 3: IV and GMM}

\subsection{Assumptions IV}

\subsubsection{Linearity}

\begin{align*}
    y_i=z_i^{\prime} \delta+\varepsilon_i \quad \text { but } \quad E\left[z_i \varepsilon_i\right] \neq 0
\end{align*}

\subsubsection{Ergodicity and Stationarity}

$\left(y_i, z_i, x_i\right)$ ergodic and stationary for LLN.

\subsubsection{Exogenous Instrument}

\begin{align*}
    E\left[g_i\right]=E\left[x_i \varepsilon_i\right]=E\left[x_i\left(y_i-z_i^{\prime} \delta\right)\right]=0
\end{align*}

\subsubsection{Identification}

$E\left[x_i z_j^{\prime}\right]$ has full rank $L$ where $\operatorname{dim}(z_i) = L \leq K = \operatorname{dim}(x_i)$

\subsection{Estimator}

\begin{align*}
    \delta&=E\left[x_i z_i^{\prime}\right]^{-1} E\left[x_i y_i\right]=\Sigma_{x z}^{-1} \sigma_{x y} \\
    \widehat{\delta}&=\left(\frac{1}{n} \sum_{i=1}^n x_i z_i^{\prime}\right)^{-1} \frac{1}{n} \sum_{i=1}^n x_i y_i\\
    \sqrt{n}(\widehat{\delta}-\delta)=&\left(\frac{1}{n} \sum_{i=1}^n x_i z_i^{\prime}\right)^{-1} \frac{1}{\sqrt{n}} \sum_{i=1}^n x_i \varepsilon_i \\
    & \stackrel{d}{\longrightarrow} E\left[x_i z_i^{\prime}\right]^{-1} \cdot N(0, S) \\
    & \stackrel{d}{\longrightarrow} N(0, E\left[x_i z_i^{\prime}\right]^{-1} S E\left[x_i z_i^{\prime}\right]^{-1}) \\
    & \stackrel{d}{\longrightarrow} N(0, E\left[x_i z_i^{\prime}\right]^{-1} E\left[\varepsilon_i^2 x_i x_i^\prime \right] E\left[x_i z_i^{\prime}\right]^{-1}) \\
    S&=E\left[g_i g_i^{\prime}\right] = \left[(x_i \varepsilon_i)(x_i \varepsilon_i)^{\prime}\right] 
\end{align*}

\subsubsection{Case 1: exogenous error}

Assume 

\begin{align*}
    E\left[\varepsilon_i \mid x_i, z_i\right]&=0 \\
    E\left[\varepsilon_i^2 \mid x_i, z_i\right]&=f(x_i)
\end{align*}

Then it is pretty much OLS:

\begin{align*}
    \widehat{\delta}_{\operatorname{OLS}} &\stackrel{p}{\longrightarrow} \delta \\
    \sqrt{n}(\widehat{\delta}_{\operatorname{OLS}}-\delta) &\stackrel{d}{\longrightarrow} N(0, V) \\
    V &= E\left[z_i z_i^{\prime}\right]^{-1}  E\left[\varepsilon_i^2 z_i z_i^\prime \right] E\left[z_i z_i^{\prime}\right]^{-1}
\end{align*}

And for GMM estimator:

\begin{align*}
    W&=S^{-1}=E\left[\varepsilon_i^2 x_i x_i^{\prime}\right]^{-1} \\
    \sqrt{n}\left(\widehat{\delta}\left(\widehat{S}^{-1}\right)-\delta\right) &\stackrel{d}{\longrightarrow} N\left(0,\left(\Sigma_{x z}^{\prime} S^{-1} \Sigma_{x z}\right)^{-1}\right)
\end{align*}

\subsubsection{Case 2: endogenous error with homoskedasticity}

Assume 

\begin{align*}
    E\left[\varepsilon_i \mid z_i\right]&\neq 0 \\
    E\left[\varepsilon_i \mid x_i\right]&=0 \\
    E\left[\varepsilon_i^2 \mid x_i, z_i\right]&=\sigma^2
\end{align*}

OLS-estimator:

\begin{align*}
    \hat{\delta}_{\operatorname{OLS}} &\stackrel{p}{\rightarrow} \delta+E\left[z_i z_i^\prime\right]^{-1} E\left[z_i \varepsilon_i\right] = \bar{\delta} \\
    \sqrt{n}\left(\hat{\delta}_{\operatorname{OLS}} - \bar{\delta} \right)&\stackrel{d}{\rightarrow} N(0,V) \\
    V &= E\left[z_i z_i^\prime\right]^{-1} E\left[u_i^2 z_i z_i^\prime \right] E\left[z_i z_i^\prime\right]^{-1} \\
    &= E\left[z_i z_i^\prime\right]^{-1} E\left[u_i^2 | z_i \right] \\
    u_i &= z_i^\prime \bar{\delta} - y_i
\end{align*}

Efficient GMM estimator:

\begin{align*}
    W&=S^{-1}=E\left[\varepsilon_i^2 x_i x_i^{\prime}\right]^{-1} \\
    \sqrt{n}\left(\widehat{\delta}\left(\widehat{S}^{-1}\right)-\delta\right) &\stackrel{d}{\longrightarrow} N\left(0,\left(\Sigma_{x z}^{\prime} S^{-1} \Sigma_{x z}\right)^{-1}\right)
\end{align*}

\subsubsection{Case 3: endogenous error with heteroskedasticity}

Assume 

\begin{align*}
    E\left[\varepsilon_i \mid z_i\right]&\neq 0 \\
    E\left[\varepsilon_i \mid x_i\right]&=0 \\
    E\left[\varepsilon_i^2 \mid x_i, z_i\right]&=f(x_i)
\end{align*}

Then we use the GMM estimator which also works in the case of overidentification with $W = \mathbb{E}\left[x_i x_i^\prime\right]^{-1}$:

\begin{align*}
    &\sqrt{n}(\widehat{\delta}(\widehat{W})-\delta) \\
    &\qquad =\left(S_{x z}^{\prime} \widehat{W} S_{x z}\right)^{-1} S_{x z}^{\prime} \widehat{W}\left(\frac{1}{\sqrt{n}} \sum_{i=1}^n x_i \varepsilon_i\right) \\
    &\qquad \stackrel{d}{\longrightarrow}\left(\Sigma_{x z}^{\prime} W \Sigma_{x z}\right)^{-1} \Sigma_{x z}^{\prime} W \cdot N(0, S) \\
    &\qquad \stackrel{d}{\longrightarrow}N\left(0,\Omega\right) \\
    \Omega & = \left(\Sigma_{x z}^{\prime} W \Sigma_{x z}\right)^{-1} \Sigma_{x z}^{\prime} W S W \Sigma_{x z}\left(\Sigma_{x z}^{\prime} W \Sigma_{x z}\right)^{-1}
\end{align*}
\hrule
\section{Notes}

\begin{align*}
    \left[
    \begin{array}{l}
        x \\
        y
    \end{array}
    \right]^{\prime}
    \left[
    \begin{array}{ll}
        a & b \\
        c & d
    \end{array}
    \right]^{-1}
    \left[
    \begin{array}{l}
        x \\
        y
    \end{array}
    \right]=\frac{d x^2-(b+c) x y+a y^2}{a d-b c}
\end{align*}

\subsection{Couples' data}

\begin{align*}
    &\hat{\beta}_{\text{2SLS}}-\beta  \\
    &=\left(\frac{1}{n} \sum_{i=1}^n \sum_{j=1}^2 x_{i j} z_{i j}^\prime\right)^{-1}\left(\frac{1}{n} \sum_{i=1}^n \sum_{j=1}^2 x_{i j} \varepsilon_{i j}\right) \\
    &\sqrt{n}\left(\hat{\beta}_{\text{2SLS}}-\beta\right)  \\
    &=\underbrace{\left(\frac{1}{n} \sum_{i=1}^n \sum_{i=1}^2 x_{i j} z_{i j}^\prime\right)^{-1}}_A \underbrace{\left(\frac{1}{\sqrt{n}} \sum_{i=1 j=1}^n \sum_{i j}^2 x_{i j}\right)}_B \\
    \text{LLN: }& A \stackrel{p}{\longrightarrow} E\left[\sum_{i=1}^2 x_{i j} z_{i j}^\prime\right]=C \\
    &B \stackrel{d}{\longrightarrow} N\left(0, E\left[\sum_{j=1}^2 x_{i j} \varepsilon_{i j}\left(\sum_{j=1}^2 x_{i j} \varepsilon_{i j}\right)^{\prime}\right]\right)=D \\
    \text{CLT: }& \sqrt{n}\left(\hat{\beta}_{\text{2SLS}}-\beta\right)\stackrel{d}{\longrightarrow} N\left(0, C^{-1}DC^{-1}\right)
\end{align*}

\subsection{Normal distribution}

\begin{align*}
    \mathbb{E}\left[X^2\right] &= \mu^2+\sigma^2 \\
    \mathbb{E}\left[X^3\right] &= \mu^3+3 \mu \sigma^2 \\
    \mathbb{E}\left[X^4\right] &= \mu^1+6 \mu^2 \sigma^2+3 \sigma^1 \\
    \mathbb{E}\left[X^2|\mu=0\right] &=  \sigma^2 \\
    \mathbb{E}\left[X^3|\mu=0\right] &=  0 \\
    \mathbb{E}\left[X^4|\mu=0\right] &= 3 \sigma^4
\end{align*}
\hrule

\end{multicols*}
\newpage

\setcounter{section}{0}

\begin{multicols*}{5}
\setlength{\columnseprule}{0.4pt}

\section{Some Basic Time Series Concepts}

\subsection{Strict Stationarity}

\begin{align*}
    f_{y_1}=f_{y_2}=f_{y_3}=\ldots
\end{align*}

\subsection{Covariance Stationarity}

\begin{align*}
    \mu_t&=\mathbb{E}\left(Y_t\right)=\mu \:;\:
    \lambda_{t, k}=\lambda_k \text { for all } t
\end{align*}

\subsection{Martingale (Difference)}

\begin{align*}
    \tag{M}
    \mathbb{E}\left\{Y_t \mid \Omega_{t-1}\right\}&=Y_{t-1} \\
    \tag{MDS}
    \mathbb{E}\left\{Y_t \mid \Omega_{t-1}\right\}&=0
\end{align*}

\subsection{AR(p) process}

\subsubsection{VAR(1)}

\begin{align*}
    Y_t &= \Phi Y_{t-1}+\varepsilon_t =\Phi^t Y_0+\sum_{i=0}^{t-1} \Phi^i \varepsilon_{t-i} \\
    A(L) Y_t &= (I-A_1 L-\ldots-A_p L^p)Y_t = \varepsilon_t
\end{align*}

For covariance stationarity, all eigenvalues of $\Phi$ are less than one in modulus, thus $\Phi^t \rightarrow 0 \text { as } t \rightarrow \infty$. 
OR
$|A(z)|$ has roots outside unit circle.

\subsubsection{AR(p)}

AR(p) as VAR(1) by companion form:

\begin{align*}
    Z_t&=\Phi Z_{t-1}+e_t \: ; \: Z_t= \left[Y_t \: \ldots \: Y_{t-p+1} \right]^\prime \\
    \Phi&= \left[ \begin{array}{llll}
    \phi_1 & \phi_2 & \ldots & \phi_p \\
    1 & 0 & \ldots & 0 \\
    \vdots & & \vdots & \vdots \\
    0 & 0 & \ldots & 0
    \end{array} \right]\\
    e_t&= \left[ \varepsilon_t \: 0 \: \ldots \: 0\right]^\prime
\end{align*}

\subsection{MA(q) are stationary}

\begin{align*}
    Y_t&=\varepsilon_t-\ldots-\theta_q \varepsilon_{t-q} \:;\:
    \varepsilon_t \sim i i d\left(0, \sigma^2\right) \\
    \mathbb{E}\left(Y_t\right)&=0 \: ; \:
    \operatorname{Var}\left(Y_t\right) =\sigma^2\left(1+\sum_{i=1}^q \theta_i^2\right) \\
    \operatorname{cov}\left(Y_t Y_{t-k}\right)&=\sigma^2\left(-\theta_\kappa+\sum_{j=1}^{q-k} \theta_j \theta_{k+j}\right) \text{ for } k\leq q \\
    \operatorname{cov}\left(Y_t Y_{t-k}\right)&=0 \text { for }|k|>q
\end{align*}

\subsubsection{Invertibility restriction}

MA(1): $|\theta|<1$ or roots of $\theta(z)=(1-\theta z)$ greater $1$ in modulus.
MA(q): roots of $\theta(z)=\left(1-\theta_1 z-\theta_2 z^2-\ldots-\theta_q z^q\right)$ greater $1$ in modulus.

Need the restriction for identification as we cannot be sure if we recover $\theta$ or $\tilde{\theta}=\theta^{-1}$, with $\tilde{\sigma}^2 = \sigma^2 \theta^2$.

\subsection{Autocovariance generating functions}

\begin{align*}
    \lambda(z)&=\sum_{j=-\infty}^{\infty} \lambda_j z^j \\
    \lambda(z)&=\sigma^2 \theta(z) \theta\left(z^{-1}\right) \text{ for MA(q)}
\end{align*}

\subsection{ARMA(p,q) models}

\begin{align*}
    \phi(L) Y_t&=\theta(L) \varepsilon_t \\
    \lambda_0&=\sigma^2\left(1+\frac{(\phi-\theta)^2}{1-\phi^2}\right) \\
    \lambda_h&=\sigma^2\left((\phi-\theta) \phi^{h-1}+\frac{(\phi-\theta)^2 \phi^h}{1-\phi^2}\right)
\end{align*}

For the ACGF transform into MA(q) representation:
\begin{align*}
    \lambda(z)&=\sigma^2 c(z) c\left(z^{-1}\right) \\
              &= \sigma^2 \phi(z)^{-1} \theta(z) \phi\left(z^{-1}\right)^{-1} \theta\left(z^{-1}\right)
\end{align*}
\hrule
\section{The Likelihood Function for Time Series}

\begin{align*}
    f\left(Y_{1: T}\right) =&f\left(Y_T \mid Y_{1: T-1}\right) f\left(Y_{1: T-1}\right) \\
    =&f\left(Y_T \mid Y_{1: T-1}\right) f\left(Y_{T-1} \mid Y_{1: T-2}\right) \cdot \\
    &f\left(Y_{1: T-2}\right) \\
    =&\prod_{t=2}^T f\left(Y_t \mid Y_{1: t-1}\right) f\left(Y_1\right)
\end{align*}


\hrule
\section{The Kalman Filter}

\subsection{The Basic Linear Model}

\begin{align*}
    y_t&=A^{\prime} x_t+H^{\prime} \xi_t+w_t \: &; \: 
    \mathbb{E}\left(w_t w_t^{\prime}\right)=R \\
    \xi_t&=F \xi_{t-1}+v_t \: &; \: 
    \mathbb{E}\left(v_t v_t^{\prime}\right)=Q
\end{align*}

\subsection{Signal extraction and the Kalman Filter}

\begin{align*}
    y_{1: t}&=\left\{y_i\right\}_{i=1}^t \\
    \xi_{t \mid k}&=\mathbb{E}\left(\xi_t \mid y_{1: k}\right) \\
    P_{t \mid k}&=\operatorname{var}\left(\xi_t \mid y_{1: k}\right) \\
    \left[\begin{array}{c}
    w_t \\
    v_t
    \end{array}\right] &\sim \operatorname{Niid}\left(\left[\begin{array}{l}
    0 \\
    0
    \end{array}\right],\left[\begin{array}{cc}
    R & \color{red}G\color{black} \\
    \color{red}G^\prime\color{black} & Q
    \end{array}\right]\right)       
\end{align*}

\subsubsection{Kalman Filter equations}

\begin{align*}
    \xi_{t \mid t-1}&=F \xi_{t-1 \mid t-1} \:\left(\mu_1\right) \\
    y_{t \mid t-1}&=A^{\prime} x_t+H^{\prime} \xi_{t \mid t-1} \:\left(\mu_2\right) \\
    P_{t \mid t-1}&=F P_{t-1 \mid t-1} F^{\prime}+Q \:\left(\Sigma_{11}\right) \\
    h_t&=H^{\prime} P_{t \mid t-1} H+R
    \color{red} + H^\prime G^\prime + GH \color{black} \:\left(\Sigma_{22}\right) \\
    K_t&=(P_{t \mid t-1} H \color{red} + G \color{black})h_t^{-1} \:\left(\Sigma_{12} \Sigma_{22}^{-1}\right) \\
    \eta_t&=y_t-y_{t \mid t-1} \:\left(z_2-\mu_2\right) \\
    \xi_{t \mid t}&=\xi_{t \mid t-1}+K_t \eta_t \:\left(\mu_1+\Sigma_{12} \Sigma_{22}^{-1}\left(z_2-\mu_2\right)\right) \\
    P_{t \mid t}&=P_{t \mid t-1}-K_t H^{\prime} (P_{t \mid t-1}\color{red}+G\color{black}) \\
    &\left(\Sigma_{11}-\Sigma_{12} \Sigma_{22}^{-1} \Sigma_{21}\right)
\end{align*}

If $\xi_t$ is covariance stationary, then $\xi_{0 \mid 0}=\mathbb{E}\left(\xi_0\right)=0$, $P_{0 \mid 0}=\operatorname{Var}\left(\xi_0\right)$.

\subsection{Hamilton}

\begin{align*}
    y_t&=c_k+\beta_k x_t+\epsilon_{k, t}, \quad \epsilon_{k, t} \sim N\left(0, \sigma_k\right)
\end{align*}

\begin{align*}
    \xi_{i, t-1}&=P\left(s_{t-1}=i \mid \tilde{y}_{t-1} ; \theta\right) \\
    P\left(s_t=j \mid s_{t-1}=i\right)&=P_{i j}=\left[
        \begin{array}{ll}
        p_{00} & p_{01} \\
        p_{10} & p_{11}
        \end{array}
    \right]
\end{align*}

\begin{align*}
    \eta_{j t} & =f\left(y_t \mid s_t=j, \tilde{y_{t-1}} ; \theta\right) \\
    & =\frac{1}{\sqrt{2 \pi} \sigma} \exp \left[-\frac{\left(y_t-c_j-\beta_j x_t\right)^2}{2 \sigma_j^2}\right]
\end{align*}

\subsection{Likelihood function}

Gaussian density for $y_{1:T}$

\begin{align*}
    f\left(y_{1: T}\right)=&\prod_{t=1}^T f\left(y_t \mid y_{1: t-1}\right) \: ; \: y_{1: 0}=\{\emptyset\} \\
    f\left(y_t \mid y_{1: t-1}\right)=&\left(\frac{1}{\sqrt{2 \pi\left|h_t\right|}}\right)^n \exp \left(-\frac{1}{2} \eta_t^{\prime} h_t^{-1} \eta_t\right)
\end{align*}


Then the likelihood is:

\begin{align*}
    f\left(Y_{1: T}\right)=&\left(\frac{1}{\sqrt{2 \pi}}\right)^{n T}\left(\prod_{t=1}^T\left|h_t\right|^{-1 / 2}\right) \\
    &\exp \left(-\frac{1}{2} \sum_{t=1}^T\left(\eta_t^{\prime} h_t^{-1} \eta_t\right)\right) \\
    \eta_t=&y_t-y_{t \mid t-1}
\end{align*}

This is the same expression as in the last section with $h_t=\sigma_{t-1}^2$ and $y_{t \mid t-1}=\mu_{t-1}$.
\hrule
\section{The Linear model with Serially Correlated Data}

\subsection{Asymptotics for serially correlated processes}

\subsubsection{Ergodicity}

A process is ergodic if its elements are asymptotically independent.

Suppose $\left\{z_t\right\}$ is stationary and ergodic with $E\left(z_t\right)=\mu$. Then $T^{-1} \sum_{i=1}^T z_t \stackrel{a . s}{\longrightarrow} \mu$.

If $z_t$ is stationary and ergodic, then so is $x_t=f\left(z_t\right)$ for arbitrary function $f$.

\subsubsection{CLT for martingale difference sequences (MDS)}

Let $\left\{g_t\right\}$ be a (possibly vector-valued) mds that is stationary and ergodic with $E\left(g_t g_t^{\prime}\right)=\Sigma_{g g}$.

\begin{align*}
    \sqrt{T} \bar{g}=\frac{1}{\sqrt{T}} \sum_{t=1}^T g_t \Rightarrow N\left(0, \Sigma_{g g}\right)
\end{align*}

\subsection{Linear and Serially Correlated Regressors}

\begin{align*}
    y_t=x_t^{\prime} \beta+\varepsilon_t
\end{align*}

\subsubsection{Assumptions}

(2) $\left\{y_t, x_t\right\}$ is a stationary and ergodic process

(3) $\mathbb{E}\left(\varepsilon_t x_t\right)=0$, or letting $g_t=\varepsilon_t x_t$ then $E\left(g_t\right)=0$

(4) $\mathbb{E}\left(x_t x_t^{\prime}\right)=\Sigma_{x x}$ which is non-singular

(5) $\left\{g_t\right\}$ is a mds with $E\left(g_t g_t^{\prime}\right)=\Sigma_{g g}$

If in addition to (2)-(5), $\mathbb{E}\left[\left(x_{t, i} x_{t, j}\right)^2\right]$ is finite for all $i$ and $j$, and let $\widehat{g}_t=\widehat{\varepsilon}_t x_t=(y_t-x_t \widehat{\beta})x_t$, and $S_{\hat{g} \hat{g}}=\frac{1}{T} \sum \widehat{g}_t^2=\frac{1}{T} \sum \hat{\varepsilon}_t^2 x_t x_t^{\prime}$. Then

\begin{align*}
    S_{\hat{g} \hat{g}} \stackrel{p}{\rightarrow} \Sigma_{g g}
\end{align*}

\subsubsection{OLS} 

\begin{align*}
    \widehat{\beta} \stackrel{p}{\rightarrow}& \beta \\
    \sqrt{T}(\widehat{\beta}-\beta) \stackrel{d}{\rightarrow}& N\left(0, \Sigma_{x x}^{-1} \Sigma_{g g} \Sigma_{x x}^{-1}\right) \\
    \xi_W=&T(R \widehat{\beta}-r)^{\prime}\left(R \widehat{V}_{\widehat{\beta}} R^{\prime}\right)^{-1}(R \widehat{\beta}-r)\\
    \xi_W \Rightarrow& \chi_m^2 \: ; \:
    \frac{\xi_W}{m} \Rightarrow F_{m, \infty}
\end{align*}

% \subsubsection{Application: AR(p)}

% Suppose $y_t = \theta(L)\varepsilon_t$ is a MA process. If $\sum_{i=0}^{\infty}\left|\theta_i\right|<\infty$, then $\sum_{i=-\infty}^{\infty}\left|\lambda_i\right|<\infty$ and the process is stationary and ergodic.

% Suppose $\phi(L)y_t = \varepsilon_t$ and the roots of $\phi(L)$ lie outside the unit circle, then $y_t = \theta(L)\varepsilon_t$ with $\sum_{i=0}^{\infty}\left|\theta_i\right|<\infty$. Thus, the process is stationary and ergodic.

% For this process: could also write $y_t=x_t^{\prime} \beta+\varepsilon_t$ and then we can apply OLS:

% \begin{align*}
%     \widehat{\beta} &\stackrel{p}{\rightarrow} \beta\\
%     \sqrt{T}(\widehat{\beta}-\beta) &\Rightarrow N\left(0, V_{\widehat{\beta}}\right)\\
%     V_{\widehat{\beta}}&=\sigma^2 \Sigma_{x x}^{-1}
% \end{align*}

\paragraph{AR(1) example:} $y_t = \phi x_{t-1}+\varepsilon_t$ and $\sqrt{T}(\widehat{\phi}-\phi) \Rightarrow N\left(0, \sigma^2 \Sigma_{x x}^{-1} \right)$. Then use from AR(1): $\Sigma_{x x}=\frac{\sigma^2}{1-\phi^2}$:

\begin{align*}
    \widehat{\phi} \stackrel{a}{\sim} N\left(\phi, \frac{1}{T}\left(1-\phi^2\right)\right)
\end{align*}

\subsection{Let $g_t$ not be a MDS}

\begin{align*}
    \frac{1}{\sqrt{T}} \sum_{t=1}^T g_t &\Rightarrow N(0, \Omega) \\
    \Omega &= \sum_{j=-T+1}^{T-1} \lambda_j-\frac{1}{T} \sum_{j=1}^{T-1} j\left(\lambda_j+\lambda_{-j}\right) \\
    & \rightarrow \sum_{j=-\infty}^{\infty} \lambda_j \\
    \Omega &= \lambda(z=1)
\end{align*}

\subsubsection{OLS With Serially Correlated Errors}

Let $\frac{1}{\sqrt{T}} \sum_{t=1}^T g_t \Rightarrow N(0, \Omega)$. Then OLS gives

\begin{align*}
    \sqrt{T}(\hat{\beta}-\beta) \Rightarrow N\left(0, \Sigma_{X X}^{-1} \Omega \Sigma_{X X}^{-1}\right)
\end{align*}

\subsection{HAC and HAR inference}

Let $\hat{V}_{\widehat{\beta}}=S_{X X}^{-1} \hat{\Omega} S_{X X}^{-1}$ and $\xi_W=T(\hat{\beta}-\beta)^{\prime} \hat{V}_{\hat{\beta}}^{-1}(\hat{\beta}-\beta)$.
If $\hat{\Omega} \stackrel{p}{\longrightarrow} \Omega$, then $\hat{V}_{\hat{\beta}} \stackrel{p}{\longrightarrow} V_{\hat{\beta}}$, and $\xi_W \Rightarrow \chi_k^2$.

\subsubsection{Estimators for $\Omega$}

With finite sample, impossible to consistently estimate $\Omega$ for all possible sequences $\left\{\lambda_j\right\}$. Sometimes it is:

Suppose $\lambda_{|j|}=0$ for $|j|>q$ (so $g_t$ follows an MA(q) process). Only estimate the variance and first $q$ auto-covariances. These are consistent.

\subsubsection{HAC Estimators for $\Omega$}

\paragraph{Truncated: } $\hat{\Omega}^{\text {Trunc }}=\sum_{j=-k}^k \hat{\lambda}_j \text { with } \hat{\lambda}_j=T^{-1} \sum_{t=1}^{T-j} g_t g_{t+j}$

\paragraph{Weighted Truncated: } $\hat{\Omega}(w)=\sum_{j=-k}^k w_j \hat{\lambda}_j$ where $w_j$ are weights.

\begin{align*}
    \hat{\Omega}^{N W}&=\sum_{j=-k}^k w_j \hat{\lambda}_j \: ; \:
    w_{|j|}=\frac{k+1-|j|}{k+1}
\end{align*}

These HAC estimators yield test statistics with good size/power properties in cases when there is
limited autocorrelation.

\subsection{OLS and HAC vs. GLS}

OLS is perfect if $\operatorname{Var}(u \mid X)=\Lambda=\sigma^2 I$. When $\Lambda \neq \sigma^2 I$, use

\begin{align*}
    \hat{\beta}^{G L S}=\left(X^{\prime} \Lambda^{-1} X\right)^{-1} X^{\prime} \Lambda^{-1} Y
\end{align*}

If $\Lambda$ is unknown, use feasible GLS:

\begin{align*}
    \hat{\beta}^{F G L S}&=\left(X^{\prime} \hat{\Lambda}^{-1} X\right)^{-1} X^{\prime} \hat{\Lambda}^{-1} Y \\
    \hat{\Lambda}&=\Lambda(\hat{\theta})
\end{align*}

\subsection{OLS (with HAC inference) or GLS?}

\begin{align*}
    y_t&=x_t^{\prime} \beta+u_t \\
    E\left(u_t \mid x_t\right)=0 &\Rightarrow E\left(u_t x_t\right)=0 \\
    u_t&=\rho u_{t-1}+\varepsilon_t \text { where } \varepsilon_t \stackrel{\text{iid}}{\sim} \left(0, \sigma^2\right) \\
    \widetilde{y}_t&=y_t-\rho y_{t-1} \text{ and } \widetilde{x}_t=x_t-\rho x_{t-1} \\
    \varepsilon_t&=u_t-\rho u_{t-1}
\end{align*}

For GLS where we regress $\widetilde{y}_t$ on $\widetilde{x}_t$, we need $\mathbb{E}\left(\varepsilon_t \widetilde{x}\right)=0$:

\begin{align*}
    &E\left[\left(u_t-\rho u_{t-1}\right)\left(x_t-\rho x_{t-1}\right)\right] \\
    &=E\left(u_t x_t\right)+\rho^2 E\left(u_{t-1} x_{t-1}\right) \\
    &-\rho E\left(u_t x_{t-1}\right)-\rho E\left(u_{t-1} x_t\right) =0 
\end{align*}

Thus the following four must hold. The first two are implied by $\mathbb{E}(u_t \mid x_t)=0$. The others need stronger assumptions.

\begin{align*}
    E\left(u_t x_t\right) & =0 \: ; \:
    E\left(u_{t-1} x_{t-1}\right) =0 \\
    E\left(u_t x_{t-1}\right) & =0 \: ; \:
    E\left(u_{t-1} x_t\right) =0
\end{align*}

Exogenous or predetermined: $E\left(u_t \mid x_t, x_{t-1}, \ldots\right)=0$.
Strictly exogenous: $E\left(u_t \mid \ldots x_{t+1}, x_t, x_{t-1}, \ldots\right)=0$. This is needed for GLS.
\hrule
\section{The Functional Central Limit}

\subsection{Wiener Process}

$W(s)$ defined on $s\in[0,1]$. 

We have $W(0)=0$.

$W\left(t_i\right)-W\left(t_{i-1}\right) \sim \mathbb{N}\left(0, t_i-t_{i-1}\right)$ are all iid.

Thus: $W(1) \sim N(0,1)$. And realizations of $W(s)$ are continuous with probability $1$.

Suppose $\varepsilon_t \stackrel{\text{iid}}{\sim}N(0,1)$, and $\xi_T(t / T)=\frac{1}{\sqrt{T}} \sum_{i=1}^t \varepsilon_i$ is linear interpolation between the points.

\subsubsection{Theorem 1 (Weak Convergence of random functions on $C[0,1]$)}

Function cannot go too crazy as $T$ grows and at the origin.

\subsubsection{Theorem 2 (CMT)}

\begin{align*}
    g: C[0,1] \rightarrow \mathbb{R} &\text{ and } \xi_T(.) \Rightarrow \xi(.) \\
    g\left(\xi_T\right) &\Rightarrow g(\xi)
\end{align*}

\subsubsection{Theorem 3 (Functional CLT)}

Suppose $\varepsilon_t$ is a MDS with $\sigma^2_\varepsilon$ and bounded $2+\delta$ moments.
Then any function $\xi_T(s)$ that linearly interpolates between the points $\xi(t/T) = \frac{1}{\sqrt{T}}\sum\limits_{i=1}^t \varepsilon_i(t/T)$ converges in distribution to a Wiener process:

\begin{align*}
    \xi_T &\Rightarrow \sigma_\varepsilon W \\
    \nu_T &=\frac{1}{T^{3 / 2}} \sum_{t=1}^T x_t=\frac{1}{T} \sum_{t=1}^T\left[\frac{1}{T^{1 / 2}} \sum_{i=1}^t \varepsilon_i\right] \\
    &=\sigma_{\varepsilon} \int_0^1 \xi_T(s) d s \Rightarrow \sigma_{\varepsilon} \int_0^1 W(s) d s=\nu
\end{align*}

\subsection{Application: Testing for a break}

Null and alternative: $H_0: \delta=0$ vs. $H_a: \delta \neq 0$

\begin{align*}
    y_t&=\beta_t+\varepsilon_t \text {, where } \varepsilon_t \sim i i d\left(0, \sigma_{\varepsilon}^2\right) \\
    \beta_t&=
    \left\{\begin{array}{c}
    \beta \text { for } t \leq \tau \\
    \beta+\delta \text { for } t>\tau
    \end{array}\right.
\end{align*}

\subsubsection{Chow Test (known break)}

\begin{align*}
    \hat{\delta}&=\bar{Y}_2-\bar{Y}_1 \\
    \bar{Y}_1&=\frac{1}{\tau} \sum_{t=1}^\tau y_t \text { and } \bar{Y}_2=\frac{1}{T-\tau} \sum_{t=\tau+1}^T y_t \\
    \hat{\delta} &\stackrel{a}{\sim} \mathbb{N}\left(\delta, \sigma_{\varepsilon}^2\left(\frac{1}{\tau}+\frac{1}{T-\tau}\right)\right) \\
    \xi_W&=\frac{1}{\hat{\sigma}_{\varepsilon}^2} \frac{\widehat{\delta}^2}{\left(\frac{1}{\tau}+\frac{1}{T-\tau}\right)} \Rightarrow \xi \sim \chi_1^2
\end{align*}

\subsubsection{Quandt Test (unknown break)}

Compute Chow statistic for many possible values of $\tau$ and use largest.

\subsection{Application: Unit root AR(1) model}

$\phi=1$. Note that the following distribution is only negative, when the numerator is: $P(\phi < 0) \approx 65\%$

\begin{align*}
    \widehat{\phi} &= \frac{\sum y_t y_{t-1}}{\sum y_{t-1}^2} \:;\:
    T(\widehat{\phi}-1) \Rightarrow \frac{\frac{1}{2}\left[\chi_1^2-1\right]}{\int_0^1 W(s)^2 d s} \\
    t&=\frac{\int_0^1 W(s) d W(s)}{\left[\int_0^1 W(s)^2 d s\right]^{\frac{1}{2}}}
\end{align*}
\hrule
\section{VARs and Related Topics}

\subsection{Basic Concepts and Notation}

\subsubsection{VAR and MA representation}

\begin{align*}
    A(L) Y_t&=\eta_t \\
    A(L)&=I-A_1 L-\ldots-A_p L^p
\end{align*}

Where $\eta_t$ is a MDS with $\Sigma_\eta$. $Y_t$ is covariance stationary. Invert $A(L)$ for MA process:

\begin{align*}
    Y_t&=C(L) \eta_t =\eta_t+C_1 \eta_{t-1}+C_2 \eta_{t-2}+\ldots
\end{align*}

$\eta_t=Y_t-\mathbb{E}\left(Y_t \mid Y^{t-1}\right)$ are the one-period-ahead forecast errors (or Wold shocks).

\subsubsection{SVAR and SMA representation}

$\varepsilon_t$ is mds vector of STRUCTURAL shocks. Then:

\begin{align*}
    \eta_t&=H \varepsilon_t \: ; \: \operatorname{Var}\left(\varepsilon_t\right) =\Sigma_{\varepsilon} \\
    B(L) Y_t&=H^{-1} A(L) Y_t = \varepsilon_t  \text{ (SVAR)}\\
    Y_t&= C(L) H \varepsilon_t=D(L) \varepsilon_t \text{ (SMA)}
\end{align*}

\subsubsection{Objects of interest}

\paragraph{Impulse Responses}

Write the SMA as $Y_t=\sum_{k=0}^{\infty} D_k \varepsilon_{t-k}$. 
% Then we find the following IR where $D_{ij,k}$ is the $ij$th element of $D_k$. Note that there is no time subscript for $D_k$:

\begin{align*}
    S I R F_{i j, h}=\frac{\partial Y_{i, t}}{\partial \varepsilon_{j, t-h}}=\frac{\partial Y_{i, t+h}}{\partial \varepsilon_{j, t}}=D_{i j, h}
\end{align*}

\paragraph{Forecast Error Var Decomp}

Suppose the structural shocks are mutually uncorrelated. Then 

\begin{align*}
    \operatorname{Var}\left(Y_{i, t+h}-E\left(Y_{i, t+h} \mid Y_t\right)\right)=\sum_{j=1}^n \sum_{k=0}^{h-1} D_{i j, k}^2 \sigma_{\varepsilon_j}^2
\end{align*}

and the fraction that is explained by the $j$th shock is

\begin{align*}
    F E V D_{i j, h}=\frac{\sum_{k=0}^{h-1} D_{i j, k}^2 \sigma_{\varepsilon_j}^2}{\sum_{j=1}^n \sum_{k=0}^{h-1} D_{i j, k}^2 \sigma_{\varepsilon_j}^2}
\end{align*}

\subsection{Invertibility}

If $n_\varepsilon > n_Y$ we cannot recover $H$ for structural shocks. Also: Can I determine $\varepsilon_t$ from current and lagged $Y$.

\subsection{Identification of $H$}

We can estimate $\Sigma_\eta$ from data and $\Sigma_\eta=H\Sigma_\varepsilon H^\prime$. We have $n(n+1)/2$ elements in $\Sigma_\eta$ and in $\Sigma_\varepsilon$, and $n^2$ in $H$. Thus, we have $n^2$ too many unknowns.

\subsubsection{Restrictions}

1. Uncorrelated structural shocks: make $\Sigma_\varepsilon$ diagonal. Still $n(n+1)/2$ too many.

2.1 Scale normalization: drop $\varepsilon_t$ unit: $\eta_t = H\varepsilon_t$.

2.2 Scale normalization: Set $\operatorname{Var}(\varepsilon_t)=I$ or $H_{ii}=1$ for $i=1, \ldots, n$.
Still $n(n-1)/2$ restrictions short.

3. other restrictions, e.g.: timing restriction. Set upper triangle of $H$ to zero. I.e. $\varepsilon_2$ does not affect $Y_1$, and $\varepsilon_3$ not $Y_1, Y_2$ etc.

\subsection{Local Projections}

Use companion form of the VAR:

\begin{align*}
    Z_t&=\Phi Z_{t-1}+e_t \quad \text{(AR)} \\
    Z_t&=e_t+\Phi e_{t-1}+\Phi^2 e_{t-2}+\ldots \: \text{(MA)} \\
    Z_{t+k}&=\Phi^k Z_t+v_{t+k} \quad \text{(Forecast)}
\end{align*}

Let $J = [I_n \: 0 \: \ldots \: 0]$, then $Y_t = J Z_t$ and $e_t = J^\prime \eta_t$. Thus

\begin{align*}
    Y_t&=\eta_t+J \Phi J^{\prime} \eta_{t-1}+J \Phi^2 J^{\prime} \eta_{t-2}+\ldots \\
    C_k&=J \Phi^k J^{\prime} \\
    Y_{t+k}&=C_k Y_t+\sum_{i=1}^{p-1} W_i Y_{t-i}+u_{t+k}
\end{align*}

$u_{t+k}$ has mean zero conditional on $(Y_t,\ldots,Y_{t-p+1})$. Thus this is a regression and $C_k$ are the coefficients on $Y_t$ from $Y_{t+k}$ onto $(Y_t,\ldots,Y_{t-p+1})$.

While the LP estimator is inefficient relative to estimators of IRS from a VAR, the LP estimators
are simple to construct, and potentially more robust to miss-specification than the VAR-plugin
estimators.

\subsection{Examples of Identification Schemes}

\subsubsection{IV instrument}

Split $H$: $H_1$ is first column, and $H^*$ the rest.

\begin{align*}
    Y_t=&H_1 \varepsilon_{1, t}+\sum_{k=1} C_k H_1 \varepsilon_{t-k}+H^* \varepsilon_t^* \\
    &+\sum_{k=1} C_k H^* \varepsilon_{t-k}^* \\
    \frac{\partial Y_{t+k}}{\partial \varepsilon_{1, t}} =&C_k H_1
\end{align*}

Use instrument $z_t$ that is only correlated with $\varepsilon_{1,t}$:

\begin{align*}
    \eta_t&=H \varepsilon_t=H_1 \varepsilon_{1, t}+H^* \varepsilon_t^* \\
    \mathbb{E}\left(\eta_t z_t\right)&=\Sigma_{\eta z}=H \mathbb{E}\left(\varepsilon_t z_t\right)=H_1 \mathbb{E}\left(\varepsilon_{1, t} z_t\right)
\end{align*}

This recovers $H_1$ up to a scale factor. Set $H_{1,1} = 1$ and done.

\begin{align}
    H_{j, 1}=\frac{\mathbb{E}\left(\eta_{j, 1} z_t\right)}{\mathbb{E}\left(\eta_{1,1} z_t\right)}
\end{align}

\subsubsection{iid shocks}

Let $\varepsilon_{j,t}$ be iid over $t$ and independent over $j$. If the distribution is not Gaussian, identification is possible.
Look for $H^{-1}$ that generates iid variables in $\varepsilon_t$.
\hrule 
\section{Notes}

\begin{align*}
    \left[
    \begin{array}{l}
        x \\
        y
    \end{array}
    \right]^{\prime}
    \left[
    \begin{array}{ll}
        a & b \\
        c & d
    \end{array}
    \right]^{-1}
    \left[
    \begin{array}{l}
        x \\
        y
    \end{array}
    \right]=\frac{d x^2-(b+c) x y+a y^2}{a d-b c}
\end{align*}

\subsection{Couples' data}

\begin{align*}
    &\hat{\beta}_{\text{2SLS}}-\beta  \\
    &=\left(\frac{1}{n} \sum_{i=1}^n \sum_{j=1}^2 x_{i j} z_{i j}^\prime\right)^{-1}\left(\frac{1}{n} \sum_{i=1}^n \sum_{j=1}^2 x_{i j} \varepsilon_{i j}\right) \\
    &\sqrt{n}\left(\hat{\beta}_{\text{2SLS}}-\beta\right)  \\
    &=\underbrace{\left(\frac{1}{n} \sum_{i=1}^n \sum_{i=1}^2 x_{i j} z_{i j}^\prime\right)^{-1}}_A \underbrace{\left(\frac{1}{\sqrt{n}} \sum_{i=1 j=1}^n \sum_{i j}^2 x_{i j}\right)}_B \\
    \text{LLN: }& A \stackrel{p}{\longrightarrow} E\left[\sum_{i=1}^2 x_{i j} z_{i j}^\prime\right]=C \\
    &B \stackrel{d}{\longrightarrow} N\left(0, E\left[\sum_{j=1}^2 x_{i j} \varepsilon_{i j}\left(\sum_{j=1}^2 x_{i j} \varepsilon_{i j}\right)^{\prime}\right]\right)=D \\
    \text{CLT: }& \sqrt{n}\left(\hat{\beta}_{\text{2SLS}}-\beta\right)\stackrel{d}{\longrightarrow} N\left(0, C^{-1}DC^{-1}\right)
\end{align*}

\subsection{Normal distribution}

\begin{align*}
    \mathbb{E}\left[X^2\right] &= \mu^2+\sigma^2 \\
    \mathbb{E}\left[X^3\right] &= \mu^3+3 \mu \sigma^2 \\
    \mathbb{E}\left[X^4\right] &= \mu^1+6 \mu^2 \sigma^2+3 \sigma^1 \\
    \mathbb{E}\left[X^2|\mu=0\right] &=  \sigma^2 \\
    \mathbb{E}\left[X^3|\mu=0\right] &=  0 \\
    \mathbb{E}\left[X^4|\mu=0\right] &= 3 \sigma^4
\end{align*}
 
\vspace{.8mm}
\hrule
% \end{multicols*}

% \newpage

% \begin{multicols*}{5}
% \setlength{\columnseprule}{0.4pt}
\newpage
\section{Discrete Choice Models}

\subsection{Linear Probability Model}

\begin{align*}
    V\left[\varepsilon_i \mid x_i\right] & =P\left(y_i=1 \mid x_i\right)-P\left(y_i=1 \mid x_i\right)^2
\end{align*}

\subsection{Nonlinear Approaches}

\begin{align*}
    P\left(y_i=1 \mid x_i\right) &= F\left(x_i^{\prime} \beta\right) \\
    \tag{Probit}
    F(\eta) &= \Phi(\eta) \\
    \tag{Logit}
    F(\eta) &= \frac{\exp(\eta)}{1+\exp(\eta)}
\end{align*}

\subsubsection{MLE}

\begin{align*}
    \mathcal{L} &= \prod_i F\left(x_i^{\prime} \beta\right)^{y_i}\left(1-F\left(x_i^{\prime} \beta\right)\right)^{1-y_i} \\
    \ln \mathcal{L} &= \sum y_i \ln F\left(x_i^{\prime} \beta\right)+\left(1-y_i\right) \ln \left(1-F\left(x_i^{\prime} \beta\right)\right)
\end{align*}

\subsection{Marginal Effects}

\begin{align}
    \tag{Probit}
    &\frac{\partial P\left(y_i=1 \mid x_i\right)}{\partial x_{i \ell}} = \phi\left(x_i^{\prime} \beta\right) \beta_{\ell} \\
    \tag{Logit}
    &\frac{\partial P\left(y_i=1 \mid x_i\right)}{\partial x_{i \ell}} = \frac{\exp \left(x_i^{\prime} \beta\right)}{\left(1+\exp \left(x_i^{\prime} \beta\right)\right)^2} \beta_{\ell}
\end{align}
\hrule
\section{Non-Linear Least Squares}

\begin{align*}
    y_i&=f\left(x_i, \beta\right)+\varepsilon_i \quad \text { with } \quad E\left[\varepsilon_i \mid x_i\right]=0 \\
    S_n(b)&=\sum_{i=1}^n\left(y_i-f\left(x_i, b\right)\right)^2
\end{align*}

\subsection{Asymptotic Distribution}

\begin{align*}
    \sqrt{n} & (\widehat{\beta}-\beta) \stackrel{d}{\longrightarrow} N\left(0, A^{-1} B A^{-1}\right) \\
    A &= E\left[\left(\frac{\partial f\left(x_i, \beta\right)}{\partial \beta}\right)\left(\frac{\partial f\left(x_i, \beta\right)}{\partial \beta}\right)^{\prime}\right] \\
    B &= E\left[\varepsilon_i^2\left(\frac{\partial f\left(x_i, \beta\right)}{\partial \beta}\right)\left(\frac{\partial f\left(x_i, \beta\right)}{\partial \beta}\right)^{\prime}\right] \\
    &= E\left[E\left[\varepsilon_i^2 \mid x_i\right]\left(\frac{\partial f\left(x_i, \beta\right)}{\partial \beta}\right)\left(\frac{\partial f\left(x_i, \beta\right)}{\partial \beta}\right)^{\prime}\right]
\end{align*}

\hrule
\section{Quantile Regression}

\begin{align*}
    y_i&=x_i^{\prime} \beta+\varepsilon_i \quad \text { with } \quad P\left(\varepsilon_i \leq 0 \mid x_i\right)=\alpha \\
    \rho_\alpha(\eta) &=
    \left\{
        \begin{array}{cc}
            -(1-\alpha) \eta & \text { if } \eta<0 \\
            \alpha \eta & \text { if } \eta \geq 0
        \end{array}
    \right. \\
    \widehat{\beta} &= \arg \min _b \frac{1}{n} \sum_{i=1}^n \rho_\alpha\left(y_i-x_i^{\prime} b\right) \\
    &=\arg \min _b \sum_{i=1}^n \rho_\alpha\left(y_i-x_i^{\prime} b\right) \\
    \sqrt{n} & (\hat{\beta}-\beta) \stackrel{d}{\longrightarrow} N\left(0, \alpha(1-\alpha) \Gamma^{-1} V \Gamma^{-1}\right) \\
    V &= E\left[x_i x_i^{\prime}\right] \: ; \: \Gamma = E\left[f_{\varepsilon \mid x}(0) x_i x_i^{\prime}\right]
\end{align*}

\subsection{Quantile IV}

\begin{align*}
    y_i=x_i^{\prime} \beta+u_i, \quad P\left(u_i \leq 0 \mid z_i\right) &= \alpha \\
    E\left[(1-\alpha)  1\left\{y_i \leq x_i^{\prime} \beta\right\}-\alpha  1\left\{y_i \geq x_i^{\prime} \beta\right\} \mid z_i\right] &= 0 \\
    E\left[\left((1-\alpha)  1\left\{y_i \leq x_i^{\prime} \beta\right\}-\alpha  1\left\{y_i \geq x_i^{\prime} \beta\right\}\right) g\left(z_i\right)\right] &= 0
\end{align*}

This is GMM with discontinuous objective function. Note, that if we knew $\beta$, one could try to find $\gamma$ as the quantile regression estimator:

\begin{align*}
    y_i-x_i^{\prime} \beta &= z_i^{\prime} \gamma+u_i, \quad P\left(u_i \leq 0 \mid z_i\right)=\alpha \\
    \widehat{\beta} &= \arg \min _b \widehat{\gamma}(b)^{\prime} W \widehat{\gamma}(b)
\end{align*}
\hrule
\section{Extremum Estimators}

\begin{align*}
    \widehat{\theta} &= \arg \max _{\theta \in \Theta} Q_n(\theta)=\arg \max _{\theta \in \Theta} n^{-1} Q_n(\theta) \\
    Q_n(\theta) &= \sum_{i=1}^n q\left(z_i, \theta\right) \: ; \: Q(\theta)=E\left[q\left(z_i, \theta\right)\right] \\
    0 &= Q_n^{\prime}(\widehat{\theta})=\sum_{i=1}^n q^{\prime}\left(z_i, \widehat{\theta}\right) \quad \text{(FOC)}
\end{align*}

Using a Taylor approximation (where $\tilde{\theta}$ lies between $\theta_0$ and $\hat{\theta}$), one can show that:

\begin{align*}
    \sqrt{n} \left(\widehat{\theta}-\theta_0\right)= &-\left[\frac{1}{n} \sum_{i=1}^n q^{\prime \prime}\left(z_i, \widetilde{\theta}\right)\right]^{-1} \cdot \\
    &\frac{1}{\sqrt{n}} \sum_{i=1}^n q^{\prime}\left(z_i, \theta_0\right) \\
    \sqrt{n} \left(\widehat{\theta}-\theta_0\right) \stackrel{d}{\longrightarrow}& N\left(0, A^{-1} V \left[q^{\prime}\left(z_i, \theta_0\right)\right] A^{-1}\right) \\
    A =& E\left[q^{\prime \prime}\left(z_i, \theta_0\right)\right]
\end{align*}

\subsubsection{MLE}

\begin{align*}
    \sqrt{n}\left( \widehat{\theta}_{\mathrm{MLE}}-\theta_0 \right)  &\stackrel{d}{\longrightarrow}  
    N\left(0, A^{-1} V\left[q^{\prime}\left(z_i, \theta_0\right)\right] A^{-1}\right) \\
    A &= E\left[q^{\prime \prime}\left(z_i, \theta_0\right)\right] \\
    q\left(z_i, \theta\right) & =\log \left(f\left(z_i, \theta\right)\right) \\
    q^{\prime}\left(z_i, \theta\right) & =\frac{\partial \log \left(f\left(z_i, \theta\right)\right)}{\partial \theta} \\
    q^{\prime \prime}\left(z_i, \theta\right) & =\frac{\partial^2 \log \left(f\left(z_i, \theta\right)\right)}{\partial \theta \partial \theta^{\prime}}
\end{align*}

If correctly specified:

\begin{align*}
    -E\left[\frac{\partial^2 \log \left(f\left(z_{i,} \theta\right)\right)}{\partial \theta \partial \theta^{\prime}}\right] 
    &=V\left[\frac{\partial \log \left(f\left(z_i, \theta\right)\right)}{\partial \theta}\right] = \mathcal{I} \\
    \sqrt{n}\left(\widehat{\theta}_{\mathrm{MLE}}-\theta_0\right) &\stackrel{d}{\longrightarrow} N\left(0, \mathcal{I}^{-1}\right)
\end{align*}

\subsubsection{Clustering}

\begin{align*}
    & Q_n(\theta)=\sum_{i=1}^n \sum_{t=1}^{T_i} q\left(z_{i t}, \theta\right)\\
    & 0=Q_n^{\prime}(\widehat{\theta})=\sum_{i=1}^n \sum_{t=1}^{T_i} q^{\prime}\left(z_{i t}, \widehat{\theta}\right) \\
    & \sqrt{n}\left(\widehat{\theta}-\theta_0\right) \stackrel{d}{\longrightarrow} N\left(0, ABA \right) \\
    & A = E\left[\sum_{t=1}^{T_i} q^{\prime \prime}\left(z_{i t}, \theta_0\right)\right]^{-1} \\
    & B = V\left[\left(\sum_{t=1}^{T_i} q^{\prime}\left(z_{i t}, \theta_0\right)\right)\right]
\end{align*}

\hrule
\section{Generalized MoM (GMM)}

\begin{align*}
    0 &= E\left[f\left(z_i, \theta_0\right)\right] \\
    \widehat{\theta} &= \arg \min _\theta\left(\frac{1}{n} \sum_{i=1}^n f\left(z_i, \theta\right)\right)^{\prime} W_n\left(\frac{1}{n} \sum_{i=1}^n f\left(z_i, \theta\right)\right)
\end{align*}

\subsubsection{Asymptotics}

\begin{align*}
    \sqrt{n}&\left(\widehat{\theta}-\theta_0\right) \stackrel{d}{\longrightarrow} N(0, \Sigma) \\
    \Sigma &= A^{-1} B^\prime W_0 S W_0 B A^{-1} \\
    A &= E\left[\frac{\partial f\left(z_i, \theta_0\right)}{\partial \theta}\right]^{\prime} W_0 E\left[\frac{\partial f\left(z_i, \theta_0\right)}{\partial \theta}\right] \\
    B &= E\left[\frac{\partial f\left(z_i, \theta_0\right)}{\partial \theta}\right] \\
    S&=V\left[f\left(z_i, \theta_0\right)\right] \\
    \text{if } W_0 &= S^{-1} \quad\text{efficient GMM}
\end{align*}

\begin{align*}
    \sqrt{n}\left(\widehat{\theta}-\theta_0\right) &\stackrel{d}{\longrightarrow} N\left(0, \left( G^\prime S^{-1} G\right)^{-1}\right) \\
    G &= E\left[\frac{\partial f\left(z_i, \theta_0\right)}{\partial \theta}\right]
\end{align*}

\subsubsection{MoM (just identified)}

\begin{align*}
    \sqrt{n}\left(\widehat{\theta}-\theta_0\right) &\stackrel{d}{\longrightarrow} 
    N\left(0, A^{-1} S (A^\prime)^{-1} \right) \\
    A &= E\left[\frac{\partial f\left(z_i, \theta_0\right)}{\partial \theta}\right]
\end{align*}

\hrule
\section{Sequential Estimators}

\begin{align*}
    0 & =\sum_{i=1}^{\prime \prime} q\left(x_i, \widehat{\theta}_1\right) \: ; \:
    0 =\sum_{i=1}^n r\left(x_i, \widehat{\theta}_1, \widehat{\theta}_2\right) \\
    f&\left(x_i, \theta\right)=\left(
    \begin{array}{c}
        q\left(x_i, \theta_1\right) \\
        r\left(x_i, \theta_1, \theta_2\right)
    \end{array}\right) \quad \text{GMM} \\
    Q_1 &= E\left[\frac{\partial q\left(\theta_{10}, \theta_{20}\right)}{\partial \theta_1^{\prime}}\right] \\
    R_1 &= E\left[\frac{\partial r\left(\theta_{10}, \theta_{20}\right)}{\partial \theta_1^{\prime}}\right] \;:\;
    R_2 = E\left[\frac{\partial r\left(\theta_{10}, \theta_{20}\right)}{\partial \theta_2^{\prime}}\right] \\
    \sqrt{n} & \left(\left(
    \begin{array}{c}
        \widehat{\theta}_1 \\
        \widehat{\theta}_2
    \end{array}\right)-\left(
    \begin{array}{c}
        \theta_{1} \\
        \theta_{2}
    \end{array}\right)\right) \\
    &\stackrel{d}{\longrightarrow} N\left(0,\left(
    \begin{array}{cc}
        Q_1^{-1} V_{11} Q_1^{-1} & \text {mess} \\
        \text {mess} & \text {mess}
    \end{array}\right)\right) \\
    \sqrt{n} & \left(\widehat{\theta}_2-\theta_{2}\right) \stackrel{d}{\longrightarrow} N\left(0, R_2^{-1} V_{22} R_2^{-1}\right) \text{ if } R_1 =  0
\end{align*}
\hrule
\section{Treatment Effects and Selection Models}

\subsection{Treatment Heterogeneity}

If effect only varies with observable covariates, let $\varepsilon_1 = \varepsilon_0 = \varepsilon$. If the effect is even common, additionally use $X^{\prime}\beta_0 = \alpha + X^{\prime}\beta_1$.

\begin{align*}
    Y_0 &= X^{\prime} \beta_0+\varepsilon_0 \\
    Y_1 &= X^{\prime} \beta_1+\varepsilon_1 \\
    Y &= X^{\prime} \beta_0+D\left(X^{\prime}\left(\beta_1-\beta_0\right)+\varepsilon_1-\varepsilon_0\right)+\varepsilon_0 \\
    TE &= Y_1-Y_0 = X^{\prime}\left(\beta_1-\beta_0\right)+\varepsilon_1-\varepsilon_0
\end{align*}

Unobservable. Focus on average instead.

\subsection{Parameters of Interest}

\begin{align*}
    \tag{ATE}
    E\left[TE\right] &\text { or } E\left[TE \mid X\right] \\
    \tag{ATET}
    E\left[TE \mid D=1\right] &\text { or } E\left[TE \mid D=1, X\right]
\end{align*}

\subsubsection{Bounds}

Assume $Y_k$ for $k\in\{0,1\}$ is bounded, so $y^{\ell} \leq Y_k \leq y^u$. Then $y^{\ell} \leq E\left[Y_k \mid D=0\right] \leq y^u$. Then we can find $E\left[TE\right] = E\left[Y_1 - Y_0\right]$ by using

\begin{align*}
    & \operatorname{Pr}(D=k) E\left[Y_k \mid D=k\right]+(1-\operatorname{Pr}(D=k)) y^{\ell} \\
    \leq & E\left[Y_k\right] \\
    \leq & \operatorname{Pr}(D=k) E\left[Y_k \mid D=k\right]+(1-\operatorname{Pr}(D=k)) y^u
\end{align*}

\subsubsection{Matching}

Assume that conditional on $X$, $(Y_1, Y_0)$ is independent of $D$, and that there are actually observations to match across treatment groups $1>\operatorname{Pr}(D=1 \mid X)>0$.

\paragraph{ATE}

\begin{align*}
    E\left[Y_1-Y_0\right] =& E\left[E\left[Y_1-Y_0 \mid X\right]\right]  \text{ (ATE)} \\
    =& E\left[E\left[Y_1 \mid X, D=1\right]\right. \\
     &\left.-E\left[Y_0 \mid X, D=0\right]\right] 
\end{align*}

\paragraph{ATET}

\begin{itemize}
    \item construct average for each $X$, and $D$
    \item difference each average across $D$s
    \item Average the differences. Weight by appearance in $D=1$
\end{itemize}

\subsubsection{Propensity Score Matching}

If $\left(Y_1, Y_0\right)$ is independent of $D$ conditional on $X$, then $\left(Y_1, Y_0\right)$ is independent of $D$ conditional on $P(X)=\operatorname{Pr}(D=1 \mid X)$.
Thus, if it is valid to match on $X$, then one can alternatively match on $P(X)$.
Very difficult to justify from an economic perspective.

\subsubsection{Differences-in-Differences Estimator}

\begin{align*}
    \widehat{\beta}_1^{\text {diff-in-diff }} &= \left(\left(\bar{Y}^{\text {treat,after }}-\bar{Y}^{\text {treat,before }}\right) \right. \\
    & \left.-\left(\bar{Y}^{\text {control,after }}-\bar{Y}^{\text {control,before }}\right)\right)
\end{align*}

\subsection{Randomized Experiments with Imperfect Compliance}

Let $Z$ be $1$ if assigned to treatment, and $0$ if assigned to control. Also let $D_1$ be the treatment status if $Z=1$, and $D_0$ the treatment status if $Z=0$. Also, $D_1, D_0$ are binary. Must assume 

\begin{itemize}
    \item Independence: $\left(Y_0, Y_1, D_0, D_1\right)$ is independent of $Z$ (random assignment)
    \item First Stage: $0<P(Z=1)<1$ and $P\left(D_1=1\right) \neq P\left(D_0=1\right)$
    \item Monotonicity: $D_1 \geq D_0 \longrightarrow$ (no defiers)
\end{itemize}

Then we have for the compliers (Local average TE = LATE):

\begin{align*}
    \alpha_\text{LATE} &= E\left[Y_1-Y_0 \mid D_1>D_0\right] \\
    &= \frac{E[Y \mid Z=1]-E[Y \mid Z=0]}{E[D \mid Z=1]-E[D \mid Z=0]}=\frac{\operatorname{cov}(Y, Z)}{\operatorname{cov}(D, Z)}
\end{align*}

Effectively, $Z$ acts as an instrument for the treatment, and one can run 2SLS of $Y$ on a constant and $D$, using $Z$ as instrument (one may include other controls $X$).

\subsubsection{Parameter Heterogeneity}

Every individual has own paremeter.

\begin{align*}
    y_i &= x_i^{\prime} \beta_i+\varepsilon_i \quad E\left[x_i \varepsilon_i\right]=0 \\
    \widehat{\beta} & \stackrel{p}{\longrightarrow} E\left[\beta_i\right]
\end{align*}

Assume $\beta_{1 i}$ and $\delta_{1 i}$ are distributed independently of $\left(u_i, v_i, z_i\right)$. And $E\left[u_i \mid z_i\right]=0$, $E\left[v_i \mid z_i\right]=0$, and $E\left[\delta_{1 i}\right] \neq 0$:

\begin{align*}
    \widehat{\beta}_1^{2 S L S} &\stackrel{p}{\longrightarrow} \frac{\operatorname{cov}\left(y_i, z_i\right)}{\operatorname{cov}\left(x_i z_i\right)}=\frac{E\left[\delta_{1 i} \beta_{1 i}\right]}{E\left[\delta_{1 i}\right]}
\end{align*}

2SLS estimates the causal effect for individuals for whom $Z_i$ is most influential (those with large $\delta_{1 i}$ ).

\subsection{Regression Discontinuity}

\begin{align*}
    & P(D=1 \mid X=x)= \begin{cases}0 & \text { for } x<c \\
    1 & \text { for } x \geq c\end{cases} \\
    & E[Y \mid X=x]= \begin{cases}E\left[Y_0 \mid X=x\right] & \text { for } x<c \\
    E\left[Y_1 \mid X=x\right] & \text { for } x \geq c\end{cases} \\
    & \lim _{x \searrow c} E[Y \mid X=x]-\lim _{x \nearrow_c} E[Y \mid X=x] \\
    & =E\left[Y_1-Y_0 \mid X=c\right]
\end{align*}
\hrule
\section{Nonparametrics}

\subsection{Kernel Density Estimator}

\begin{align*}
    \widehat{f}(x) &= \frac{1}{n h_n} \sum_{i=1}^n K\left(\frac{x-x_i}{h_n}\right) \\
    E[\widehat{f}(x)] &= f(x)+\frac{1}{2} h^2 f^{\prime \prime}(x) \int v^2 K(v) d v+O\left(h^4\right) \\
    V[\widehat{f}(x)] &= \frac{1}{n h} f(x) \int K(v)^2 d v+O\left(n^{-1}\right)
\end{align*}

\subsubsection{Epanechnikov kernel}

\begin{align*}
    K_{\text {opt }}(t)=\frac{3}{4 \cdot 5^{1 / 2}}\left(1-\frac{1}{5} t^2\right) 1\left(t^2 \leq 5\right)
\end{align*}
\hrule
% \input{Sections/Week4/6_ML_1}
% \hrule
\section{Machine Learning}

\subsection{Trees}

Highly intuitive, easy to explain, highly flexible BUT hard to interpret, discrete step function (even for continuous data), and might need a lot of leaves.

Uses regression sample split algorithm:

\begin{align*}
    Y_i &= \mu_1 1\left\{X_{d i} \leq \gamma\right\}+\mu_2 1\left\{X_{d i}>\gamma\right\}+\varepsilon_i\\
    \mathbb{E}\left[\varepsilon_i \mid X_i\right] &= 0
\end{align*}

\begin{itemize}
    \item The parameters are $d, \gamma, \mu_1$, and $\mu_2$
    \item $d$ and $\gamma$ are estimated by grid search
    \item The estimates produce a sample split
    \item need $N_{\text{min}}$ for stopping criteria
\end{itemize}

\subsection{Bagging (Bootstrap Aggregating)}

You generate a large number B of bootstrap samples.
Estimate your regression model on each bootstrap sample.
The average of the bootstrap estimates is the bagging estimator.

\subsection{Random Forests}

Random forests are a modification of bagged regression trees. The modification is to reduce estimation variance. 

1. Draw a nonparametric bootstrap sample.

2. Grow a regression tree on the bootstrap sample using m variables chose at random from the p regressors

\begin{align*}
    \widehat{m}_{\mathrm{rf}}(x)=B^{-1} \sum_{b=1}^B \widehat{m}_b(x)
\end{align*}


\subsection{Elastic Net (Ridge / Lasso)}

\begin{align*}
    y_i=\sum_{j=1}^k x_{i j} \beta_j+\varepsilon_i \: \text{(many regressors)}
\end{align*}

For Lasso, set $\alpha = 0$. For Ridge, set $\alpha = 1$. Get the parameters via $m$-fold cross validation.

\begin{align*}
    \min_{b_j} &\sum_{i=1}^n\left(y_i-\sum_{j=1}^k x_{i j} \beta_j\right)^2 \\
    &+\lambda\left((1-\alpha) \sum_{j=1}^k\left|b_j\right|+\alpha \sum_{j=1}^k b_j^2\right)
\end{align*}

\subsection{Double Selection Lasso (IV)}

Use Lasso to estimate

\begin{align*}
    D_i=x_i^{\prime} \gamma+v_i
\end{align*}

Let $x_1$ be the selected variables.

Use Lasso to estimate

\begin{align*}
    Y_i=x_i^{\prime} \delta+v_i
\end{align*}

Let $x_2$ be the selected variables.

Let $\widetilde{x}=x_1 \cup x_2$ and regress (OLS)

\begin{align*}
    y_i=D_i \theta+\widetilde{x}_i^{\prime} \beta+\varepsilon_i
\end{align*}

to get the estimator of $\theta$.
\hrule
\section{Notes}

\begin{align*}
    \left[
    \begin{array}{l}
        x \\
        y
    \end{array}
    \right]^{\prime}
    \left[
    \begin{array}{ll}
        a & b \\
        c & d
    \end{array}
    \right]^{-1}
    \left[
    \begin{array}{l}
        x \\
        y
    \end{array}
    \right]=\frac{d x^2-(b+c) x y+a y^2}{a d-b c}
\end{align*}

\subsection{Couples' data}

\begin{align*}
    &\hat{\beta}_{\text{2SLS}}-\beta  \\
    &=\left(\frac{1}{n} \sum_{i=1}^n \sum_{j=1}^2 x_{i j} z_{i j}^\prime\right)^{-1}\left(\frac{1}{n} \sum_{i=1}^n \sum_{j=1}^2 x_{i j} \varepsilon_{i j}\right) \\
    &\sqrt{n}\left(\hat{\beta}_{\text{2SLS}}-\beta\right)  \\
    &=\underbrace{\left(\frac{1}{n} \sum_{i=1}^n \sum_{i=1}^2 x_{i j} z_{i j}^\prime\right)^{-1}}_A \underbrace{\left(\frac{1}{\sqrt{n}} \sum_{i=1 j=1}^n \sum_{i j}^2 x_{i j}\right)}_B \\
    \text{LLN: }& A \stackrel{p}{\longrightarrow} E\left[\sum_{i=1}^2 x_{i j} z_{i j}^\prime\right]=C \\
    &B \stackrel{d}{\longrightarrow} N\left(0, E\left[\sum_{j=1}^2 x_{i j} \varepsilon_{i j}\left(\sum_{j=1}^2 x_{i j} \varepsilon_{i j}\right)^{\prime}\right]\right)=D \\
    \text{CLT: }& \sqrt{n}\left(\hat{\beta}_{\text{2SLS}}-\beta\right)\stackrel{d}{\longrightarrow} N\left(0, C^{-1}DC^{-1}\right)
\end{align*}

\subsection{Normal distribution}

\begin{align*}
    \mathbb{E}\left[X^2\right] &= \mu^2+\sigma^2 \\
    \mathbb{E}\left[X^3\right] &= \mu^3+3 \mu \sigma^2 \\
    \mathbb{E}\left[X^4\right] &= \mu^1+6 \mu^2 \sigma^2+3 \sigma^1 \\
    \mathbb{E}\left[X^2|\mu=0\right] &=  \sigma^2 \\
    \mathbb{E}\left[X^3|\mu=0\right] &=  0 \\
    \mathbb{E}\left[X^4|\mu=0\right] &= 3 \sigma^4
\end{align*}
\end{multicols*}

\end{document}
