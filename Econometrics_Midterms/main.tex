\documentclass[12pt]{article}
\usepackage[utf8]{inputenc}

%allows you to neatly integrate an abstract once you start with the actual content
\usepackage{abstract}
\usepackage{microtype}
\usepackage{enumitem}

%%%%%%%%%%%%%%%%%%%%%%%%%%%%%%%%%%%%%%%%%%%%%%%%%%%%%%%%%%%%%%%%%%%%%%%%%%%%%%%%%%%%%%%%%% 
% SETTING THE GEOMETRY
%%%%%%%%%%%%%%%%%%%%%%%%%%%%%%%%%%%%%%%%%%%%%%%%%%%%%%%%%%%%%%%%%%%%%%%%%%%%%%%%%%%%%%%%%%

%borders and such
\usepackage
[
margin=1.5in
%a4paper,% other options: a3paper, a5paper, etc
%left=2.5cm,
%right=2.5cm, top=5cm,
% use vmargin=2cm to make vertical margins equal to 2cm.
% us  hmargin=3cm to make horizontal margins equal to 3cm.
% use margin=3cm to make all margins  equal to 3cm.
]
{geometry}
%makes for onehalfspacing in the whole document
\usepackage{setspace} 
\onehalfspacing
% \doublespacing
% allows to enter other margins for big figures or tables
\usepackage{changepage}
% disallow footnotes to extend over more than one page
\interfootnotelinepenalty=10000

%%%%%%%%%%%%%%%%%%%%%%%%%%%%%%%%%%%%%%%%%%%%%%%%%%%%%%%%%%%%%%%%%%%%%%%%%%%%%%%%%%%%%%%%%% 
% FONTS & COLORS
%%%%%%%%%%%%%%%%%%%%%%%%%%%%%%%%%%%%%%%%%%%%%%%%%%%%%%%%%%%%%%%%%%%%%%%%%%%%%%%%%%%%%%%%%% 
% Obviously colors. Usage: \color{red} Red text. This also allows you to make custom colors
\usepackage{xcolor}      
%if you want to start a multicolomns part inside your normally one columned article
\usepackage{multicol}
%titlesec allows you to change the way, titles are presented in latex
% \usepackage{titlesec}
%lets you change the style of your (sub)section headers
\usepackage{sectsty}
%I chose to make my headers cyan-colored
\sectionfont{\fontsize{16}{15}\selectfont}
\subsectionfont{\fontsize{14}{15}\selectfont}
\subsubsectionfont{\fontsize{12}{15}\selectfont}
% \subsectionfont{\color{cyan}}
% \subsubsectionfont{\color{cyan}}  
% \paragraphfont{\color{cyan}} 

% superscript 1st, 2nd, 3rd, 4th ... use \nth{8}
\usepackage[super]{nth}

%%%%%%%%%%%%%%%%%%%%%%%%%%%%%%%%%%%%%%%%%%%%%%%%%%%%%%%%%%%%%%%%%%%%%%%%%%%%%%%%%%%%%%%%%% 
% CITATION
%%%%%%%%%%%%%%%%%%%%%%%%%%%%%%%%%%%%%%%%%%%%%%%%%%%%%%%%%%%%%%%%%%%%%%%%%%%%%%%%%%%%%%%%%% 

\usepackage[english]{babel}
\usepackage{csquotes}

\usepackage[authordate,backend=biber,uniquename=false,uniquelist=false]{biblatex-chicago}
\bibliography{references}

\ExecuteBibliographyOptions{
    giveninits=true,
    % isbn=false,
    url=true,
    doi=true,
    % uniquename=init
    }
    
\DeclareBibliographyAlias{webpage}{online} 
\AtEveryBibitem{\clearlist{language}}
\AtEveryBibitem{%
      \ifentrytype{online}
        {}
        {\clearfield{urlyear}\clearfield{urlmonth}\clearfield{urlday}}
    }

%%%%%%%%%%%%%%%%%%%%%%%%%%%%%%%%%%%%%%%%%%%%%%%%%%%%%%%%%%%%%%%%%%%%%%%%%%%%%%%%%%%%%%%%%% 
% INSERTING PICTURES
%%%%%%%%%%%%%%%%%%%%%%%%%%%%%%%%%%%%%%%%%%%%%%%%%%%%%%%%%%%%%%%%%%%%%%%%%%%%%%%%%%%%%%%%%% 

%very useful if you want to add a jpfg or png file. Usage: \includegraphics{path/to/file}
\usepackage{grffile}
\usepackage{graphicx}
\usepackage{subcaption}
\usepackage[format=plain,
            % labelfont=it,
            font=footnotesize,
            % width=0.85\textwidth,
            % textfont=it,
            % singlelinecheck=on
            ]{caption}
% Figures can't live outside the section
\usepackage[section]{placeins}
% and then also \FloatBarrier at the end of a section to enforce harder.

% For figures turned 90 degrees
\usepackage{rotating}
% For uppercase subfigure panels
\renewcommand{\thesubfigure}{\Alph{subfigure}}


% Draw pictures, like game trees
\usepackage{tikz}
\usetikzlibrary{trees}
% Node styles
\tikzset{
% Two node styles for game trees: solid and hollow
solid node/.style={circle,draw,inner sep=1.5,fill=black},
hollow node/.style={circle,draw,inner sep=1.5,fill=white}
}
\usepackage{pgfplots}
\usepgfplotslibrary{fillbetween}


%%%%%%%%%%%%%%%%%%%%%%%%%%%%%%%%%%%%%%%%%%%%%%%%%%%%%%%%%%%%%%%%%%%%%%%%%%%%%%%%%%%%%%%%%% 
% INSERTING TABLES
%%%%%%%%%%%%%%%%%%%%%%%%%%%%%%%%%%%%%%%%%%%%%%%%%%%%%%%%%%%%%%%%%%%%%%%%%%%%%%%%%%%%%%%%%% 

\usepackage{array}
\usepackage{tabularx}
\usepackage{longtable}
\newcolumntype{x}[1]{>{\let\newline\\\arraybackslash\hspace{0pt}}p{#1}}
\newcolumntype{y}[1]{>{\centering\let\newline\\\arraybackslash\hspace{0pt}}p{#1}}
\newcolumntype{s}[1]{>{\centering\arraybackslash}p{#1}}
\renewcommand*{\arraystretch}{1.2}

\usepackage{multirow}

%%%%%%%%%%%%%%%%%%%%%%%%%%%%%%%%%%%%%%%%%%%%%%%%%%%%%%%%%%%%%%%%%%%%%%%%%%%%%%%%%%%%%%%%%% 
% LINKS AND WWW
%%%%%%%%%%%%%%%%%%%%%%%%%%%%%%%%%%%%%%%%%%%%%%%%%%%%%%%%%%%%%%%%%%%%%%%%%%%%%%%%%%%%%%%%%% 

% Provides clickable links in the PDF-document for \ref and the TOC
\usepackage[]{hyperref} %you can remove the hidelinks option to make the hyperrefs visible. looks very ugly to me, but might be useful to less experienced readers of pdfs
% Lets you typeset urls with linebreak. Usage: \url{http://...}  
\usepackage{url}

%%%%%%%%%%%%%%%%%%%%%%%%%%%%%%%%%%%%%%%%%%%%%%%%%%%%%%%%%%%%%%%%%%%%%%%%%%%%%%%%%%%%%%%%%% 
% CODE INTEGRATION
%%%%%%%%%%%%%%%%%%%%%%%%%%%%%%%%%%%%%%%%%%%%%%%%%%%%%%%%%%%%%%%%%%%%%%%%%%%%%%%%%%%%%%%%%% 

% Source Code Listings. Usage: \begin{lstlisting}...\end{lstlisting}
\usepackage{listings}
%this defines a lot of parameters for code. makes it look nice IMO
\lstset{ 
	breaklines=true,                 % sets automatic line breaking
	commentstyle=\color{mygreen},    % comment style
	escapeinside={\%*}{*)},          % if you want to add LaTeX within your code
	frame=single,	                   % adds a frame around the code
	keepspaces=true,                 % keeps spaces in text, useful for keeping indentation of code (possibly needs columns=flexible)
	keywordstyle=\color{blue},       % keyword style
	numbers=left,                    % where to put the line-numbers; possible values are (none, left, right)
	numbersep=5pt,                   % how far the line-numbers are from the code
	numberstyle=\tiny\color{mygray}, % the style that is used for the line-numbers
	rulecolor=\color{black},         % if not set, the frame-color may be changed on line-breaks within not-black text (e.g. comments (green here))
	showspaces=false,                % show spaces everywhere adding particular underscores; it overrides 'showstringspaces'
	showstringspaces=false,          % underline spaces within strings only
	showtabs=false,                  % show tabs within strings adding particular underscores
	stepnumber=2,                    % the step between two line-numbers. If it's 1, each line will be numbered
	stringstyle=\color{mymauve},     % string literal style
	tabsize=2,	                   % sets default tabsize to 2 spaces
	title=\lstname                   % show the filename of files included with \lstinputlisting; also try caption instead of title
}         

%%%%%%%%%%%%%%%%%%%%%%%%%%%%%%%%%%%%%%%%%%%%%%%%%%%%%%%%%%%%%%%%%%%%%%%%%%%%%%%%%%%%%%%%%% 
% MATH
%%%%%%%%%%%%%%%%%%%%%%%%%%%%%%%%%%%%%%%%%%%%%%%%%%%%%%%%%%%%%%%%%%%%%%%%%%%%%%%%%%%%%%%%%% 

%some very useful packages for writing down beautiful formulas
\usepackage{amsmath}
\usepackage{amssymb}
\usepackage{mathtools}
\usepackage{mathrsfs}  
\usepackage{bm}
\usepackage{empheq}
% Nice rules for tables. Usage \begin{tabular}\toprule ... \midrule ... \bottomrule
\usepackage{booktabs}          

%%%%%%%%%%%%%%%%%%%%%%%%%%%%%%%%%%%%%%%%%%%%%%%%%%%%%%%%%%%%%%%%%%%%%%%%%%%%%%%%%%%%%%%%%% 
% APPENDIX
%%%%%%%%%%%%%%%%%%%%%%%%%%%%%%%%%%%%%%%%%%%%%%%%%%%%%%%%%%%%%%%%%%%%%%%%%%%%%%%%%%%%%%%%%% 

\usepackage{appendix}
% Change numbering of figures, tabels etc. in Appendix
\usepackage{chngcntr} 

\usepackage{titletoc}


%%%%%%%%%%%%%%%%%%%%%%%%%%%%%%%%%%%%%%%%%%%%%%%%%%%%%%%%%%%%%%%%%%%%%%%%%%%%%%%%%%%%%%%%%% 
% ALL ABOUT YOU
%%%%%%%%%%%%%%%%%%%%%%%%%%%%%%%%%%%%%%%%%%%%%%%%%%%%%%%%%%%%%%%%%%%%%%%%%%%%%%%%%%%%%%%%%% 


\title{22BDP Macroeconomics week 4}
\author{Tom Rodriguez, Bénédicte Droz}
\date{06.08.2023}

%start your document with \begin{document} and type everything inbetween \begin{document} and \end{document}
\begin{document}

\begin{titlepage}
	
	\newcommand{\HRule}{\rule{\linewidth}{0.5mm}} % Defines a new command for the horizontal lines, change thickness here
	
	\center % Center everything on the page
	
	%----------------------------------------------------------------------------------------
	%	HEADING SECTIONS
	%----------------------------------------------------------------------------------------
	
% 	\includegraphics{Figures/UZH.eps} \\[1.5cm] 
    %	\textsc{\LARGE University of Zurich}\\[1.5cm] % Name of your university/college and the size of the distance to the next line. \LARGE is used to make really big fonts
	\textsc{\large Beginning Doctoral Program Gerzensee}\\[0.5cm] % Major heading such as course name
	{\large Lectures held by Mark Watson and Bo Honor\'e}\\[1cm] % you can the name(s) of your instructor(s)
	
	
	%----------------------------------------------------------------------------------------
	%	TITLE SECTION
	%----------------------------------------------------------------------------------------
	
	\HRule \\[1.0cm]
	{ \LARGE \bfseries Econometrics Midterm Solutions}\\[0.4cm] % Title of your document, and yes, I am aware that it is stupid to have to list it twice. I don't know how to work around this.
	\HRule \\[2cm]
	
	%----------------------------------------------------------------------------------------
	%	AUTHOR SECTION
	%----------------------------------------------------------------------------------------
	

	
	\begin{flushleft}
        \Large List of Contributors:
        
		{\large 
            \href{https://rodrigueztom.github.io}{Tom Rodriguez (University of Fribourg)}
        }
	\end{flushleft}
	
	
	%----------------------------------------------------------------------------------------
	%	DATE SECTION
	%----------------------------------------------------------------------------------------
	
	{\large \today} % Date, change the \today to a set date if you want to be precise
	
	% Fill the rest of the page with whitespace
	\vfill 
	
\end{titlepage}

%%%%%%%%%%%%%%%%%%%%%%%%%%%%%%%%%%%%%%%%%%%%%%%%%%%%%%%%%%%%%%%%%%%%%%%%%%%%%%%%%%%%%%%%%% 
% GEOMETRY according to university guidelines
%%%%%%%%%%%%%%%%%%%%%%%%%%%%%%%%%%%%%%%%%%%%%%%%%%%%%%%%%%%%%%%%%%%%%%%%%%%%%%%%%%%%%%%%%% 

%This line can be used to adjust the geometry of the document according to the university guidelines for borders and such
% \newgeometry{margin=1.5in}

%%%%%%%%%%%%%%%%%%%%%%%%%%%%%%%%%%%%%%%%%%%%%%%%%%%%%%%%%%%%%%%%%%%%%%%%%%%%%%%%%%%%%%%%%% 
% ABSTRACT
%%%%%%%%%%%%%%%%%%%%%%%%%%%%%%%%%%%%%%%%%%%%%%%%%%%%%%%%%%%%%%%%%%%%%%%%%%%%%%%%%%%%%%%%%% 

%\thispagestyle{empty} %suppresses the footer and header since I want to start show them after the TOC, optional
\pagenumbering{roman}	
% \tableofcontents

% The \clearpage command is a pagebreak to separate the abstract from the content
%\clearpage
\newpage
\pagenumbering{arabic}

%%%%%%%%%%%%%%%%%%%%%%%%%%%%%%%%%%%%%%%%%%%%%%%%%%%%%%%%%%%%%%%%%%%%%%%%%%%%%%%%%%%%%%%%%% 
 % ACTUAL CONTENT 
%%%%%%%%%%%%%%%%%%%%%%%%%%%%%%%%%%%%%%%%%%%%%%%%%%%%%%%%%%%%%%%%%%%%%%%%%%%%%%%%%%%%%%%%%%

\newpage
\section{Microeconomics Midterm 2014 / 15}

{
\subsection*{Schmidt}

\subsubsection*{Exercise 1}

\begin{enumerate}[label=(\alph*)]
{\item 
\begin{align*}
    x_{1}(\lambda p, \lambda w)=\lambda^{1+\alpha-\delta} \frac{p_{1}^{\alpha} w}{p_{1}^{\delta}+p_{2}^{\delta}+p_{3}^{\delta}}=\lambda^{1+\alpha-\delta} x_{1}(p, w)
\end{align*}

Must have $1+\alpha-\delta=0$ or $\alpha=\delta-1$

\begin{align*}
    x_{2}(\lambda p, \lambda w)=\lambda^{1+\alpha-\delta} \frac{p_{2}^{\alpha} w}{p_{1}^{\delta}+p_{2}^{\delta}+p_{3}^{\delta}}+\beta \frac{p_{1}}{p_{3}} \frac{\lambda}{\lambda}
\end{align*}

No restriction on $\beta$.

\begin{align*}
    x_{3}(\lambda p, \lambda w)=\lambda^{1+\alpha-\sigma} \frac{\gamma p_{3}^{\alpha} w}{p_{1}^{\delta}+p_{2}^{\delta}+p_{3}^{\delta}}=\lambda^{1+\alpha-\delta} x_{3}(p, w)
\end{align*}

No restriction on $\gamma$.

In summary, we only need $\alpha=\delta-1$.
}
{\item 

\begin{align*}
    p_{1} x_{1}(\cdot)+p_{2} x_{2}(\cdot)+p_{3} x_{3}(\cdot) &= w \quad \text{to satisfy Walras' Law} \\
    \Leftrightarrow \frac{w}{p_{1}^{\delta}+p_{2}^{\sigma}+p_{3}^{\sigma}}\left[p_{1}^{1+\alpha}+p_{2}^{1+\alpha}+\gamma p_{3}^{1+\alpha}\right]+\beta \frac{p_{1} p_{2}}{p_{3}} &= w
\end{align*}

Must have $\beta=0$ :

\begin{align*}
    p_{1}^{\delta}+p_{2}^{\delta}+p_{3}^{\delta}=p_{1}^{1+\alpha}+p_{2}^{1+\alpha}+\gamma p_{3}^{1+\alpha}
\end{align*}

Must have $\gamma=1$ \& $\alpha=\delta-1$.

In summary:

\begin{align*}
    \alpha=\delta-1 \quad \beta=0 \quad \gamma=1
\end{align*}
}
\end{enumerate}
}
{
\subsubsection*{Exercise 2}

\begin{enumerate}[label=(\alph*)]
{\item 
Invert $e(p,u)$ as in equilibrium: $e(p, u)=w$ and also $u=v(p, w)$

\begin{align*}
    v\left(p, w\right)=w \frac{p_{1}+p_{2}}{p_{1} p_{2}}=w\left[\frac{1}{p_{1}}+\frac{1}{p_{2}}\right]
\end{align*}
}
{\item 
Roy's Identity

\begin{align*}
    x_1\left(p_1 w\right)=-\frac{\frac{\partial v(\cdot)}{\partial p_1}}{\frac{\partial v(\cdot)}{\partial w}}=\frac{w \frac{1}{p_1^2}}{\frac{p_1+p_2}{p_1 p_2}}=\frac{w}{p_1+p_2}\frac{p_2}{p_1} \\
    x_2\left(p_1 w\right)=\frac{w}{p_1+p_2} \frac{p_1}{p_2} \text { by symmetry }
\end{align*}
}
{\item 
\begin{align*}
    & \frac{x_1\left(p, w\right)}{x_2\left(p, w\right)}=\left(\frac{p_1}{p_2}\right)^{-2} \\
    & \eta_{12}=-(-2)\left(\frac{p_1}{p_2}\right)^{-3} \frac{\frac{p_1}{p_2}}{\left(\frac{p_1}{p_2}\right)^{-2}}=2
\end{align*}
}
\end{enumerate}
}
{
\subsubsection*{Exercise 2}

\begin{enumerate}[label=(\alph*)]
{\item 
Invert $e(p,u)$ as in equilibrium: $e(p, u)=w$ and also $u=v(p, w)$

\begin{align*}
    v\left(p, w\right)=w \frac{p_{1}+p_{2}}{p_{1} p_{2}}=w\left[\frac{1}{p_{1}}+\frac{1}{p_{2}}\right]
\end{align*}
}
{\item 
Roy's Identity

\begin{align*}
    x_1\left(p_1 w\right)=-\frac{\frac{\partial v(\cdot)}{\partial p_1}}{\frac{\partial v(\cdot)}{\partial w}}=\frac{w \frac{1}{p_1^2}}{\frac{p_1+p_2}{p_1 p_2}}=\frac{w}{p_1+p_2}\frac{p_2}{p_1} \\
    x_2\left(p_1 w\right)=\frac{w}{p_1+p_2} \frac{p_1}{p_2} \text { by symmetry }
\end{align*}
}
{\item 
CES utility:

\begin{align*}
    u\left(x_1, x_2\right)=\left[\frac{1}{2} x_1^\rho+\frac{1}{2} x_2^\rho\right]^{\frac{1}{\rho}} \text { where } \rho=1-\frac{1}{n_{12}}=1 / 2
\end{align*}
}
\end{enumerate}
}
{
\subsubsection*{Exercise 3}

The difference between consumer theory and production theory is mainly the fact that firms do not have budget constraints.
This problem introduces a budget constraint. Therefore, we are going to treat the problem like a consumer problem.
In that sense, the revenue is comparable to the utility function, and the cash constraint is like the wealth of a consumer.
Consequently, we are solving the following revenue maximization problem (which is the analogue to a utility maximization problem):

\begin{align*}
    \max_{z_1,z_2} pf(z_1,z_2) \\
    \operatorname{s.t.} \; w_1z_1 + w_2z_2 \leq C
\end{align*}

We will assume an interior solution (the budget constraint is binding).
Then, the revenue function $R(p, w_1, w_2, C)$ that the exercise gives us is just the equivalent to the indirect utility.

\begin{enumerate}[label=(\alph*)]
{\item 
As $R(p, w_1, w_2, C)$ works like the indirect utility, we apply Roy's identity to find the factor demand, which is the analogue to the Walrasian demand:

\begin{align*}
    z_1&=-\frac{\frac{\partial R}{\partial w_1}}{\frac{\partial R}{\partial C}} \\
    &= -\frac{p \cdot(-\alpha) \frac{1}{w_1}}{p \cdot \frac{1}{C}} \\
    &= \alpha \frac{C}{w_1}
\end{align*}
}
{\item 
We treat $R(p,w,C)$ as the indirect utility depending on income and invert it to find the cost function $C(p,w,R)$, which is the analogue to the expenditure function in consumer theory:

\begin{align*}
    R&=p\left[\gamma+\ln C(p,w,R)-\alpha \ln w_1-(1-\alpha) \ln w_2\right] \\
    \frac{R}{p}-\gamma &= \ln \left(\frac{C(p,w,R)}{w_1^\alpha w_2^{1-\alpha}} \right) \\
    \exp\left(\frac{R}{p}-\gamma\right)&=\frac{C(p,w,R)}{w_1^\alpha w_2^{1-\alpha}} \\
    C(p,w,R) &= w_1^\alpha w_2^{1-\alpha}\exp\left(\frac{R}{p}-\gamma\right)
\end{align*}
}
{\item 
Since the cost function from (b) happens to be the analogue to the expenditure function, we can apply Shephard's Lemma in order to find the factor demand for a given $R$ at minimum cost, as this is the analogue to the Hicksian demand in consumer theory.
In that spirit, let us call this function $h_1(p,w,R)$.

\begin{align*}
    h_1(p,w,R)&=\frac{\partial C\left(w,R\right)}{\partial w_1} \\
    &= \alpha \exp \left[\frac{R}{p}-\gamma\right] \cdot\left(\frac{w_2}{w_1}\right)^{1-\alpha}
\end{align*}
}
{\item 
In consumer theory, the Hicksian demand and the Walrasian demand meet at optimum. We can also show that here:

\begin{align*}
    h_1(w,R)&=z_1^* \\
    \alpha \exp \left[\frac{R}{p}-\gamma\right] \cdot\left(\frac{w_2}{w_1}\right)^{1-\alpha}&=\alpha \frac{C}{w_1}\\
    \exp \left[\frac{R}{p}-\gamma\right] w_1^\alpha w_2^{1-\alpha}&=C \\
    \frac{R}{p}-\gamma &= \ln \left(\frac{C}{w_1^\alpha w_2^{1-\alpha}} \right) \\
    R&=p\left[\gamma+\ln C-\alpha \ln w_1-(1-\alpha) \ln w_2\right]
\end{align*}

The last line is exactly the formula for the revenue that is observed by our econometrician friend in the optimum. Therefore, we have shown that the two demands are equal whenever the firm is acting optimally, i.e. maximizing its revenue or minimizing its cost. Put differently, the revenue maximization problem is the dual problem to the cost minimization problem and vice versa.
}
\end{enumerate}
}

\newpage
{
\subsection*{Gottardi}

\subsubsection*{Exercise 1}

\begin{enumerate}[label=(\alph*)]
{\item 
True. We need three things for FWT:

\begin{itemize}
    \item LNS, which is satisfied by monotonicity
    \item Complete markets, satisfied by two prices for two commodities
    \item free disposal (given)
\end{itemize}
}
{\item 
False. Convexity is violated by B. Consider the following illustration:

\begin{figure}[!ht]
    \centering
    \includegraphics[width=0.75\linewidth]{images/2014_15_1.png}    
\end{figure}

Because $B$ has non-convex preferences, $x$ is not a CE. Actually no CE exists.
}
{\item 
False by same argument as in (b).
}
\end{enumerate}
}
{
\subsubsection*{Exercise 2}

\begin{enumerate}[label=(\alph*)]
{\item 
PE allocations are along $x_{1}^{A}=x_{2}^{A}$. If we are at any other point, just give some to $B$ because $A$ only cares about lower amount.

\begin{figure}[!ht]
    \centering
    \includegraphics[width=0.75\linewidth]{images/2014_15_2.png}    
\end{figure}
}
{\item 
Let $p=\frac{p_{1}}{p_{2}}$

\underline{Consumer A:}

\begin{align*}
    x_{1}^{A}=x_{2}^{A} \quad\text{BC: } p x_{1}^{A}+x_{2}^{A}=6 p+2
\end{align*}

\underline{Consumer B:}

\begin{align*}
    x_1^B=\left\{\begin{array}{lll}
        \infty & \text { if } & p<1 / 3 \\
        \mathbb{R}^{+} & \text {if } & p=1 / 3 \\
        0 & \text { if } & \rho>1 / 3
    \end{array}\right. 
    \quad ; \quad
    x_2^B=\left\{\begin{array}{lll}
        \infty & \text { if } & p>1 / 3 \\
        \mathbb{R}^{+} & \text {if } & p=1 / 3 \\
        0 & \text { if } & p<1 / 3
    \end{array}\right. \\
    \text{BC: } p x_1^B+x_2^B=2 p+6
\end{align*}

\underline{Market Clearing:}

\begin{align*}
    & x_1^A+x_1^B = w_1^A+w_1^B=8 \\
    & x_2^A+x_2^B = w_2^A+w_2^B=8
\end{align*}

use $x_1^A=x_2^A \longrightarrow x_1^B=x_2^B$. Therefore $p=\frac{1}{3}$ so no excess demand for either good.

By $\operatorname{BC}^A$:

\begin{align*}
    & x_1^A=x_2^A=3 \\
    & x_1^B=x_2^B=5
\end{align*}

\underline{Competitive Equilibrium:}

\begin{align*}
    \left(x_1^A, x_2^A\right) & =(3,3) \\
    \left(x_1^B, x_2^B\right) & =(5,5) \\
    p & =\frac{1}{3}
\end{align*}

This is PE since $x_1^A = x_2^A$.
}
{\item 
Yes. The reason is that $p=\frac{1}{3}$ is the only possible equilibrium price. Otherwise markets cannot clear \& we have excess demand for one of the commodities.
}
\end{enumerate}
}
{
\subsubsection*{Exercise 3}

$w^1=(8,4) \text {; } \quad w^2=(2,6)$

\begin{enumerate}[label=(\alph*)]
{\item 
Note that we have (1) identical beliefs
and (2) no aggregate risk as $w_{1}=w_{2}=10$.

Therefore, full risk sharing is possible and $x_{1}^{h}=x_{2}^{h} \quad \forall h$ is PE. I.e. the $45^{\circ}$-line:

\begin{figure}[!ht]
    \centering
    \includegraphics[width=.75\textwidth]{images/2014_15_3.png}
\end{figure}
}
{\item 
\underline{Consumer h:}

\begin{align*}
    \max_{x_{1}^{h}, x_{2}^{h}} \quad & \pi u^{h}\left(x_{1}^{h}\right)+(1-\pi) u^{h}\left(x_{2}^{h}\right) \\
    \text { s.t. } \quad & q_{1} \theta_{1}^{h}+q_{2} \theta_{2}^{h}=0 \\
    & x_{1}^{n}=w_{1}^{h}+\theta_{1}^{h} \\
    & x_{2}^{u}=w_{1}^{h}+\theta_{2}^{h}
\end{align*}

Plug in the $\theta$s:

\begin{align*}
    \max _{x_{1}^{h}, x_{2}^{h}} \quad & \pi u^{h}\left(x_{1}^{h}\right)+(1-\pi) u^{h}\left(x_{2}^{h}\right) \\
    \text { s.t. } & q_{1}\left(x_{1}^{h}-w_{1}^{h}\right)+q_{2}\left(x_{2}^{h}-w_{2}^{h}\right)=0
\end{align*}

FOCs:

\begin{align*}
    \pi \frac{\partial u^{h}\left(x_{1}^{h}\right)}{\partial x_{1}^{h}}-\lambda q_{1}&=0 \\
    (1-\pi) \frac{\partial u^{h}\left(x_{2}^{h}\right)}{\partial x_{2}^{h}}-\lambda q_{2} &= 0 \\
    \Longrightarrow \quad \frac{q_{1}}{q_{2}} &= \frac{\pi}{1-\pi} \frac{\frac{\partial u^{h}\left(x_{1}^{h}\right)}{\partial x_{1}^{h}}}{\frac{\partial u^{h}\left(x_{2}^{h}\right)}{\partial x_{2}^{h}}}
\end{align*}

Perfect risk sharing implies: $x_{1}^{h} = x_{2}^{h}$, and therefore we have 

\begin{align*}
    \frac{q_1}{q_2}=\frac{\pi}{1-\pi}
\end{align*}

Plug this into the BC:

\begin{align*}
    \frac{q_{1}}{q_{2}}\left(x_{1}^{h}-w_{1}^{h}\right)+x_{1}^{h}-w_{2}^{h} &= 0 \\
    x_{1}^{h}\left(\frac{q_{1}}{q_{2}}+1\right) &= w_{2}^{h}+w_{1}^{h} \frac{q_{1}}{q_{2}} \\
    x_{1}^{h}=x_{2}^{h}=\left(\frac{q_{1}}{q_{2}}+1\right)^{-1}\left(w_{2}^{h}+w_{1}^{h} \frac{q_{1}}{q_{2}}\right) 
    &= (1-\pi)\left(w_{2}^{h}+w_{1}^{h} \frac{\pi}{1-\pi}\right) 
    =\pi w_{1}^{h}+(1-\pi) w_{2}^{h} \\
    x_{1}^1=x_{2}^1 &= \pi 8+(1-\pi) 4=4(1+\pi) \\
    x_{1}^{2}=x_{2}^{2} &= \pi 2+(1-\pi) 6=6-4 \pi
\end{align*}

\underline{Competitive Equilibrium: }

\begin{align*}
    \left(x_1^1, x_2^1\right) & =(4+4 \pi, 4+4 \pi) \\
    \left(x_1^2, x_2^2\right) & =(6-4 \pi, 6-4 \pi) \\
    \frac{q_1}{q_2} & =\frac{\pi}{1-\pi}
\end{align*}
}
\end{enumerate}
}\newpage
\section{Econometrics Final 2015 / 16}

{
\subsection*{Watson}

{
\subsubsection*{Exercise 1}

\begin{enumerate}[label=(\alph*)]
{\item 
No. Since $|\theta|>1$, this process is not invertible:

$$
\begin{aligned}
X_{t} & =(1-2 L) \varepsilon_{t} \\
X_{t} \frac{1}{1-2 L} & =\varepsilon_{t}
\end{aligned}
$$

We cannot turn $\frac{1}{1-2 L}$ into an infinite sum. Or, one con show that:

$$
\begin{aligned}
\varepsilon_{t} =& X_{t}+\theta \varepsilon_{t-1} \\
=& X_{t}+\theta\left(X_{t-1}+\theta \varepsilon_{t-2}\right) \\
&\vdots \\
=& \sum_{j=0}^{t} \theta^{j} X_{t-j}+\theta^{t} \varepsilon_{0}
\end{aligned}
$$

If $|\theta|>1, \varepsilon_{0}$ gets more important as $t$ grows. Thus, $\varepsilon_{t}$ cannot be recovered from past values of $X_{t}$.
}
{\item 
\begin{enumerate}[label=(\roman*)]
    \item 
    By observational equivalence, one could estimate:
    
    $$
    X_{t}=\tilde{\varepsilon}_{t}-\tilde{\theta} \tilde{\varepsilon}_{t-1} \text { where } \tilde{\theta}=1 / 2 ; \operatorname{Var}\left(\tilde{\varepsilon}_{t}\right)=1
    $$
    
    Then we also find
    
    $$
    X_{t} \frac{1}{1-\tilde{\theta}L}=\tilde{\varepsilon}_{t} \Rightarrow \tilde{\varepsilon}_{T}=\sum_{j=0}^{T-1} \tilde{\theta}^{j} X_{T-j}
    $$
    
    $$
    \mathbb{E}_{T}\left(X_{T+1}\right)=-\tilde{\theta} \sum_{j=0}^{T-1} \tilde{\theta}^{j} X_{T-j}
    $$
    \item 
    The first line directly follows from the result above. The jump to the second line is legal because $X_{1:T}$ are known, therefore they have no variance.
    
    $$
    \begin{aligned}
    \operatorname{Var}\left(X_{T+1}\right) & =\operatorname{Var}\left(\tilde{\varepsilon}_{T+1}-\sum_{j=0}^{T-1} \tilde{\theta}^{j} X_{T-j}\right) \\
    & =\operatorname{Var}\left(\varepsilon_{T+1}\right)=1
    \end{aligned}
    $$
\end{enumerate}
}
\end{enumerate}
}
{
\subsubsection*{Exercise 2}

\color{blue} Did not cover this in lectures \color{black}
}
{
\subsubsection*{Exercise 3}
Correlated errors. Use the Kalman Filter derived in exercise session 2:

$$
\left[\begin{array}{l}
    w_{t} \\ v_{t}
\end{array}\right] 
\stackrel{\text { iid }}{\sim} 
N\left(\left[\begin{array}{l}
    0 \\ 0
\end{array}\right],
\left[\begin{array}{ll}
    R & G \\ 
    G^{\prime} & Q
\end{array}\right]\right) \quad 
\begin{aligned} 
    & y_{t}=A^{\prime} X_{t}+H^{\prime} \xi_{t}+w_{t} \\ 
    & \xi_{t}=F \xi_{t-1}+v_{t}
\end{aligned}
$$


Here: $R=1 ; Q=2 ; G=1 ; A^{\prime} X_{t}=0 ; F=1 ; H=1$

$\xi_{t-1 | t-1}=3 ; P_{t-1 | t-1}=0.5 ; y_{t}=2$

\begin{center}
\begin{tabular}{|l|ll|l|l|}
\hline \# & Variable & Formula & Value \\
\hline 1 & $\xi_{t | t-1}$ & $F \xi_{t-1 | t-1}$ & 3 \\
2 & $Y_{t | t-1}$ & $A^{\prime} X_{t}+H^{\prime} \xi_{t | t-1}$ & 3 \\
3 & $P_{t | t-1}$ & $F P_{t-1 | t-1} F^{\prime}+Q$ & 2.5 \\
4 & $h_{t}$ & $H^{\prime} P_{t | t-1} H+R + \color{blue} H^\prime G^\prime + GH \color{black}$ & 5.5 \\
5 & $K_{t}$ & $\left( P_{t | t-1} H + G \right) \cdot h_{t}^{-1}$ & 0.64 \\
6 & $\eta_{t}$ & $Y_{t}-Y_{t | t-1}$ & -1 \\
7 & $\xi_{t | t}$ & $\xi_{t | t-1}+K_{t} \eta_{t}$ & 2.36 \\
8 & $P_{t | t}$ & $P_{t | t-1}-K_{t} \left( P_{t | t-1} H + G \right)$ & 1.08 \\
\hline
\end{tabular}
\end{center}

Therefore:

$$
\begin{aligned}
& \xi_{t | t}=2.36 \\
& P_{t | t}=1.08
\end{aligned}
$$
}
{
\subsubsection*{Exercise 4}

\begin{enumerate}[label=(\alph*)]
{\item 
OLS: 

$$
\sqrt{T}(\hat{\beta}-\beta)=\left(\frac{1}{T} \sum_{t=1}^{T} x_{t}^{2}\right)^{-1}\left(\frac{1}{\sqrt{T}} \sum_{t=1}^{T} x_{t} u_{t}\right)
$$

$$
\mathbb{E}\left(x_{t} u_{t}\right)=\mathbb{E}\left(\left(\varepsilon_{t+1}+\varepsilon_{t+2}\right) u_{t}\right)=0
$$

Unfortunately, $\left(x_{t} u_{t}\right\}$ is not a martingale difference sequence:

$$
\begin{aligned}
\mathbb{E}\left(x_{t}u_t \mid \Omega_{t-1}\right) & =\mathbb{E}\left(\left(\varepsilon_{t+1}+\varepsilon_{t+2}\right)\left(\phi u_{t-1}+\varepsilon_{t}\right) \mid \Omega_{t-1}\right) \\
& =\mathbb{E}\left(\phi u_{t-1} \varepsilon_{t+1}+\phi u_{t-1} \varepsilon_{t+2}+\varepsilon_{t} \varepsilon_{t+1}+\varepsilon_{t} \varepsilon_{t+2} \mid \Omega_{t-1}\right)
\end{aligned}
$$

What is $\Omega_{t-1}$ in this case? $\left\{u_{j}\right\}_{j=0}^{t=1}$ and $\left\{x_{j}\right\}_{j=0}^{t-1}$.

Then, we can find:

$$
\left.\begin{gathered}
x_{t-1}=\varepsilon_{t}+\varepsilon_{t+1} \\
x_{t-2}=\varepsilon_{t-1}+\varepsilon_{t} \\
\vdots \\
x_{0}=\varepsilon_{1}+\varepsilon_{2}
\end{gathered} \right\rvert\, \Rightarrow \text { recover } \varepsilon_{t+1} \pm \varepsilon_{1}
$$

Using $\left\{u_{j}\right\}_{j=0}^{t-1}: \quad \varepsilon_{1}=u_{1}-\phi u_{0}$

Since we are able to recover $\varepsilon_{t+1} \pm \varepsilon_{1}$, and $\varepsilon_{1}$, we can also recover all $\left\{\varepsilon_{j}\right\}_{j=1}^{t+1}$. Therefore, $\varepsilon_{t+1} \in \Omega_{t-1}$ and $\varepsilon_t \in \Omega_{t-1}$. Thus, we conclude that

$$
\mathbb{E}\left(x_{t} u_{t} \mid \Omega_{t-1}\right) \neq 0
$$

Thus, we have that:

$$
\begin{aligned}
& \frac{1}{\sqrt{T}} \sum_{t=1}^{T} x_{t} u_{t} \xrightarrow{d} N\left(0, \sum_{j=-\infty}^{\infty} \lambda_{j}\right)
\end{aligned}
$$

Now, we must find all $\lambda_j$:

$$
\begin{aligned}
& \lambda_{0}= \mathbb{E}\left(x^{2} u_{t}^{2}\right)=\mathbb{E}\left(x_{t}^{2}\right) \mathbb{E}\left(u_{t}^{2}\right)=2 \sigma^{2} \cdot \sigma^{2}\left(1-\phi^{2}\right)^{-1} \\
& \lambda_{1}=\mathbb{E}\left(x_{t} x_{t-1}\right) \mathbb{E}\left(u_{t} u_{t-1}\right)=\sigma^{2} \cdot \phi \sigma^{2}\left(1-\phi^{2}\right)^{-1} \\
& \lambda_{j}= 0 \quad \forall j>2 \text { since } \mathbb{E}\left(x_{t} x_{t-j}\right)=0 \quad \forall j>2 \end{aligned}
$$

We can then find the sum we were looking for:

$$
\begin{aligned}
& \sum_{j=-\infty}^{\infty} \lambda_{j}=2 \sigma^{2} \cdot \sigma^{2}\left(1-\phi^{2}\right)^{-1}(1+\phi)=\frac{2 \sigma^{4}}{1-\phi}
\end{aligned}
$$

Also: 

$$
\left(\frac{1}{T} \sum x_{t}^{2}\right)^{-1} \xrightarrow{p} \mathbb{E}\left(x_{t}^{2}\right)^{-1}=\left(2 \sigma^{2}\right)^{-1}
$$

Conclusion: 

$$
\sqrt{T}(\hat{\beta}-\beta) \xrightarrow{d} N\left(0, \frac{1}{2(1-\phi)}\right)
$$
}
{\item 
We know that

$$
\hat{\beta} \stackrel{a}{\sim} N\left(\beta, \frac{1}{T} \frac{1}{2(1-\phi)}\right)
$$

Thus, we construct the following $C I$ :

$$
\begin{aligned}
C I_{95} & =\hat{\beta} \pm 1.96 \sqrt{\frac{1}{T}\left(\sum_{j=-1}^{1} \hat{\lambda}_{j}\right)\left(\frac{1}{T} \sum x_{t}^{2}\right)^{-2}} \\
& =0.08 \pm 1.96 \sqrt{\frac{1}{400}(2.43+6.15+2.43)(1.95)^{-2}} \\
& =[-0.087 ; 0.247]
\end{aligned}
$$

Since $0 \in C I_{95}$, we cannot reject the null hypothesis.
}
\end{enumerate}
}
{
\subsubsection*{Exercise 5}

\begin{align*}
& \left|\begin{array}{l}
Y_{t}=Y_{t-1}+X_{t}=\sum_{u=1}^{t} X_{u} \\
X_{k}=X_{k-1}+\varepsilon_{k}=\sum_{j=1}^{u} \varepsilon_{j}
\end{array}\right| \Rightarrow Y_{t}=\sum_{u=1}^{t} \sum_{j=1}^{u} \varepsilon_{j}
\end{align*}

\begin{align*}
& \sum_{t=1}^{T} Y_{t}=\sum_{t=1}^{T} \sum_{u=1}^{t} \sum_{j=1}^{u} \varepsilon_{j} \\
& T^{-5 / 2} \sum_{t=1}^{T} Y_{t}=\frac{1}{T} \sum_{t=1}^{T} \boxed{\frac{1}{T} \sum_{u=1}^{t} \frac{1}{\sqrt{T}} \sum_{j=1}^{u} \varepsilon_{j}}
\end{align*}

The blue block is analogue to what we saw in class.
Let $\xi(t / T)=T^{-1 / 2} \sum_{i=1}^{t} \varepsilon_{i}(t / T)$, and $\xi_{T}(s)$ the function that linearly interpolates between these points, then $\xi_{T}(s) \xrightarrow{d} W$ as $\sigma^{2}=1$ in this case. $W$ is a Wiener process. As we are summing over $n$ observations, we have

$$
\boxed{\frac{1}{T} \sum_{u=1}^{t} \frac{1}{\sqrt{T}} \sum_{j=1}^{u} \varepsilon_{j}} \xrightarrow{d} \int_{0}^{u} W(s) d s
$$

And now, sum this over all $t$, then we get

$$
T^{-5 / 2} \sum_{t=1}^{T} Y_{t}=\frac{1}{T} \sum_{t=1}^{T} \frac{1}{T} \sum_{u=1}^{t} \frac{1}{\sqrt{T}} \sum_{j=1}^{u} \varepsilon_{j} \xrightarrow{d} \int_{0}^{1} \int_{0}^{u} W(s) d s d u
$$
}
}
{
\subsection*{Honor\'e}

{
\subsubsection*{Exercise 1}

\begin{enumerate}[label=(\alph*)]
{\item 
Solve the maximization problem:

$$
\begin{aligned}
    \max _{m>0} \mathbb{E}\left(-\frac{Y}{m}\right)-\ln (m) \\
    \text{FOC: }\mathbb{E}(Y) \frac{1}{m^{2}}-\frac{1}{m}=0 \\
    \Rightarrow m=\mathbb{E}(Y)=\mu
\end{aligned}
$$
}
{\item 
Solve the maximization problem:

$$
\begin{aligned}
    \max _{b} &\sum_{i=1}^{n} \frac{y_{i}}{f\left(x_{i}, b\right)}-\ln \left(f\left(x_{i}, b\right)\right) \\
    \text{FOC: } &\sum_{i=1}^{n}-y_{i} f\left(x_{i}, b\right)^{-2} f^{\prime}\left(x_{i}, b\right)-f\left(x_{i}, b\right)^{-1} f^{\prime}\left(x_{i} b\right)=0 \\
    &\sum_{i=1}^{n}\left(y_{i} f\left(x_{i}, b\right)^{-1}+1\right) f\left(x_{i}, b\right)^{-1} f^{\prime}\left(x_{i, b} b\right)=0
\end{aligned}
$$

\color{red} No clue how to continue \color{black}
}
\end{enumerate}
}
{
\subsubsection*{Exercise 2}

\color{red} No clue. I don't think we looked at ordered logit in lectures. \color{black}
}
{
\subsubsection*{Exercise 3}

$$
\begin{array}{lll}
P(D=1)=\alpha & \mathbb{E}\left(Y_{1} \mid D=1\right)=10 & \mathbb{E}\left(Y_{0} \mid D=0\right)=5 \\
& 0 \leq Y_{0} \leq Y_{1} & 0 \leq Y_{1} \leq 15
\end{array}
$$

$$
\begin{aligned}
& \left.\begin{array}{l}
\alpha 10+(1-\alpha) 0 \leq \mathbb{E}\left(Y_{1}\right) \\
\mathbb{E}\left(Y_{1}\right) \leq \alpha 10+(1-\alpha) 15
\end{array} \right\rvert\, \Rightarrow 10 \alpha \leq \mathbb{E}\left(Y_{1}\right) \leq 15-5{\alpha} \\
& \left.\begin{array}{l}
(1-\alpha) 5+\alpha 0 \leq \mathbb{E}\left(Y_{0}\right) \\
\mathbb{E}\left(Y_{0}\right) \leq(1-\alpha) 5+\alpha 15
\end{array} \right\rvert\, \Rightarrow 5-5 \alpha \leq \mathbb{E}\left(Y_{0}\right) \leq 5+10 \alpha
\end{aligned}
$$

By these conditions, we conclude that:

$$
\begin{aligned}
& \mathbb{E}\left(Y_{1}-Y_{0}\right) \in[15 \alpha-5,10-15 \alpha]
\end{aligned}
$$

Size of interval:

$$
15 \alpha-5-10+15 \alpha=30 \alpha-15
$$

Smallest if size is zero: $\alpha=1 / 2$
}
{
\subsubsection*{Exercise 4}

\begin{enumerate}[label=(\alph*)]
{\item 
Instead of regressing the mean, one wants to find a quantile. Let's say one is interested in the median ($50 \%$ quantile) effect that smoking has on the risk of lung cancer. Then, one would run a quantile regression, and would obtain the constant (risk of cancer for non-smokers), and $\hat{\beta}$ (the median increase in risk for smokers).
}
{\item 
Say, one has data that does not fit the linear model very well. Instead of going non-linear one could run multiple linear regressions on subsets of the data.
}
\end{enumerate}
}
{
\subsubsection*{Exercise 5}
Difference the model:

$$
\Delta \varepsilon_{i 2}=\Delta y_{i 2}-\Delta x_{1i2}^{\prime} \beta_{1}-\Delta x_{2i2}^{\prime} \beta_{2} \quad i=1, \dots, n
$$

Use the information on $\mathbb{E}(\varepsilon \mid x)$.

$$
\begin{array}{ll}
\mathbb{E}\left(\Delta \varepsilon_{i 2} \mid x_{1is}\right)=0 & \text { for } s=1,2 \\
\mathbb{E}\left(\Delta \varepsilon_{i 2} \mid x_{2is}\right)=0 & \text { for } s=1
\end{array}
$$

Thus, we found 3 moment conditions. The model is over-identified if $\operatorname{dim}\left(x_{1 i t}\right)+\operatorname{dim}\left(x_{2 i t}\right)<3$. I.e. only if $x_{1it}$ \& $x_{2it}$ are scalars.
}
}
\newpage
\section{Econometrics Final 2016 / 17}

{
\subsection*{Watson}

{
\subsubsection*{Exercise 1}

Kalman Filter equations:

$$
\begin{aligned}
& F=0.9 ; \quad H=R=Q=1 \\
& \operatorname{Var}\left(x_{t}\right)=\frac{1}{1-0.81} \cong 5.263
\end{aligned}
$$

\begin{center}
\begin{tabular}{|l|ll|l|l|}
\hline \# & Variable & & Formula & Value \\
\hline 1 & $x_{t|t-1}$ & $\left(\mu_{1}\right)$ & $F \cdot \mathbb{E}\left(x_{t-1} \mid y_{1: t-1}\right)$ & 0 \\
2 & $y_{t|t-1}$ & $\left(\mu_{2}\right)$ & $H \cdot \mu_{1}$ & 0 \\
3 & $P_{t|t-1}$ & $\left(\Sigma_{11}\right)$ & $F^{2} \cdot V\left(x_{t}\right)+Q$ & 5.263 \\
4 & $h_{t}$ & $\left(\Sigma_{22}\right)$ & $H^{2} P_{t | t-1}+R$ & 6.263 \\
5 & $K_{t}$ & $\left(\Sigma_{12} \Sigma_{22}^{-1}\right)$ & $P_{t| t-1} \cdot H \cdot h_{t}^{-1}$ & 0.840 \\
6 & $\eta_{t}$ & $\left(z_{2}-\mu_{2}\right)$ & $y_{t}-y_{t | t-1}$ & 1 \\
7 & $x_{t|t}$ & $\left(\mathbb{E}\left(z_{1} | z_{2}\right)\right)$ & $x_{t | t-1}+K_{t} \eta_{t}$ & 0.840 \\
8 & $P_{t|t}$ & $\left(V\left(z_{1} | z_{2}\right)\right)$ & $P_{t | t-1}-K_{t} H P_{t | t-1}$ & 0.841 \\
\hline
\end{tabular}
\end{center}

\begin{enumerate}[label=(\alph*)]
{\item 
Since everything is normally distributed, we find:

$$
\begin{aligned}
f\left(y_{t} \mid x_{t-1}=0, y_{t-1}=2\right) & =\frac{1}{\sqrt{2 \pi h_{t}}} \exp \left[-\frac{1}{2} \frac{\eta_{t}^{2}}{h_{t}}\right] \\
& \cong \frac{1}{\sqrt{2 \pi 6.263}} \exp \left[-\frac{1}{2} \frac{y_{t}^{2}}{6.263}\right]
\end{aligned}
$$

Remember, we don't know $y_t$ in (a).
}
{\item 
Also a normal distribution. Thus:

$$
\begin{aligned}
f\left(x_{t} \mid y_{t}=1, x_{t-1}=0, y_{t-1}=2\right) & =\frac{1}{\sqrt{2 \pi P_{t|t}}} \exp \left[-\frac{1}{2} \frac{\left(x_{t}-x_{t|t}\right)^{2}}{P_{t|t}}\right] \\
& \cong \frac{1}{\sqrt{2 \pi 0.841}} \exp \left[-\frac{1}{2} \frac{\left(x_{t}-0.84\right)^{2}}{0.841}\right]
\end{aligned}
$$
}
\end{enumerate}
}
{
\subsubsection*{Exercise 2}

\color{red} Could not solve this, here's what I did: \color{black}

$$
y_{t}\left|x_{t-1}=x_{t}\right| x_{t-1}+v_{t}
$$

$$
\begin{aligned}
& P\left(x_{t}=1 \mid x_{t-1}=0\right)=0.2  \\
& P\left(x_{t}=0 \mid x_{t-1}=0\right)=0.8 \\
\Rightarrow &f\left(x_{t} \mid x_{t-1}=0\right)=0.2^{x_{t}} \cdot 0.8^{1-x_{t}}
\end{aligned}
$$

$$
\begin{aligned}
f\left(y_{t} \mid y_{1: t-1}\right)= & f\left(y_{t} \mid x_{t}=1\right) P\left(x_{t}=1 \mid y_{1: t-1}\right) \\
& +f\left(y_{t} \mid x_{t}=0\right) P\left(x_{t}=0 \mid y_{1: t-1}\right) \\
= & f\left(1+v_{t}\right) \cdot 0.2+f\left(v_{t}\right) \cdot 0.8 \\
= & \frac{0.2}{\sqrt{2 \pi}} \exp \left[-\frac{1}{2}\left(y_{t}-1\right)^{2}\right]+\frac{0.8}{\sqrt{2 \pi}} \exp \left[-\frac{1}{2} y_{t}^{2}\right]
\end{aligned}
$$
}
{
\subsubsection*{Exercise 3}

\begin{enumerate}[label=(\alph*)]
{\item 

$$
\begin{aligned}
    \operatorname{Var}\left(x_{t}\right)=1+4=5 &\text{ from } \varepsilon_{t} \stackrel{\text{iid}}{\sim} N(0,1) \\
    \operatorname{Var} \left(x_{t}\right) =\left(1+\theta^{2}\right) \sigma_{\eta}^{2} &\text{ from } M A(1)
\end{aligned}
$$

Combine the two:

\begin{equation*}
5=\left(1+\theta^{2}\right) \sigma_{\eta}^{2} \tag{1}
\end{equation*}

Also get auto-covariance:

$$
\begin{aligned}
\operatorname{Cov}\left(x_{t}, x_{t+1}\right) & =\operatorname{Cov}\left(\varepsilon_{t}+2 \varepsilon_{t-1}, \varepsilon_{t+1}+2 \varepsilon_{t}\right) \\
& =2 \operatorname{Cov}\left(\varepsilon_{t}, \varepsilon_{t}\right)=2 \\
\operatorname{Cov}\left(x_{t}, x_{t+1}\right) & =\operatorname{Cov}\left(\eta_{t}+\theta \eta_{t-1}, \eta_{t+1}+\theta \eta_{t}\right) \\
& =\theta \operatorname{Cov}\left(\eta_{t}, \eta_{t}\right)=\theta \sigma_{\eta}^{2}
\end{aligned}
$$

Combine the two:

\begin{equation*}
2=\theta \sigma_{\eta}^{2} \tag{2}
\end{equation*}

Plug (2) into (1):

$$
\begin{aligned}
5 &= \left(1+\theta^{2}\right) \frac{2}{\theta} \\
\Leftrightarrow \quad 0 &= 2 \theta^{2}-5 \theta+2 \\
\Leftrightarrow \quad 0 &= \theta^{2}-2.5 \theta+1 \\
\Leftrightarrow \theta_{1 / 2} &= \frac{2.5 \pm \sqrt{2.25}}{2}=\frac{2.5 \pm 1.5}{2}=\{1 / 2 ; 2\}
\end{aligned}
$$

$$
\begin{aligned}
    \theta_{1}= \frac{1}{2} & \quad \text{By invertibility} \\
    \sigma_{\eta}^{2}=4 & \quad \text{By (2)}
\end{aligned}
$$
}
{\item 
$$
\begin{aligned}
& \eta_{t}+\theta \eta_{t-1}=(1+\theta L) \eta_{t}=\varepsilon_{t}+2 \varepsilon_{t-1}=(1+2 L) \varepsilon_{t} \\
& \begin{aligned}
\eta_{t} & =\frac{1+2 L}{1+\theta L} \varepsilon_{t}=(1+2 L)\left(1-\theta L+\theta^{2} L^{2}-\theta^{3} L^{3}+\ldots\right) \varepsilon_{t} \\
& =\left(1+2 L-\theta L-2 \theta L^{2}+\theta^{2} L^{2}+2 \theta^{2} L^{3}-\theta^{3} L^{3}-2 \theta^{3} L^{4}+\ldots\right) \varepsilon_{t} \\
& =\left(1+(2-\theta) L+\left(-2 \theta+\theta^{2}\right) L^{2}+\left(2 \theta^{2}-\theta^{3}\right) L^{3}+\ldots\right) \varepsilon_{t} \\
& =\left(1+(2-\theta) L+(2-\theta)(-\theta) L^{2}+(2-\theta)(-\theta)^{2} L^{3}+\ldots\right) \varepsilon_{t} \\
\eta_{t} & =\varepsilon_{t}+(2-\theta) \sum_{i=0}^{t}(-\theta)^{i} \varepsilon_{t-1-i}
\end{aligned}
\end{aligned}
$$

}
\end{enumerate}
}
{
\subsubsection*{Exercise 4}

\color{red} Did not cover this in lectures. \color{black}
}
{
\subsubsection*{Exercise 5}

\begin{enumerate}[label=(\alph*)]
{\item \color{white} asdf \color{black} % necessary for format of enumerate
\begin{enumerate}[label=(\roman*)]
{\item 
$$
\begin{aligned}
\sqrt{T}(\hat{\alpha}-\alpha) &= \left(\frac{1}{T} \sum x_{t-1}^{2}\right)^{-1}\left(\frac{1}{\sqrt{T}} \sum x_{t-1} \varepsilon_{t}\right) \\
\left(\frac{1}{T} \sum x_{t-1}^{2}\right)^{-1} &\xrightarrow{p} \mathbb{E}\left(x_{t-1}^{2}\right)^{-1}=\frac{1-\phi^{2}}{2} \\
\left(\frac{1}{\sqrt{T}} \sum x_{t-1} \varepsilon_{t}\right) &\xrightarrow{d} N\left(0, \mathbb{E}\left(x_{t-1}^{2} \varepsilon_{t}^{2}\right)\right)
\end{aligned}
$$

Use the fact that $x_{t} \perp \varepsilon_{t} \forall t$

$$
\begin{aligned}
\left(\frac{1}{\sqrt{T}} \sum x_{t-1} \varepsilon_{t}\right) \xrightarrow{d} & N\left(0, \mathbb{E}\left(x_{t-1}^{2}\right) \mathbb{E}\left(\varepsilon_{t}^{2}\right)\right)
\end{aligned}
$$

Combine the two: (Slutsky)

$$
\begin{aligned}
& \sqrt{T}(\hat{\alpha}-\alpha) \xrightarrow{d} N\left(0, \frac{1-\phi^{2}}{2}\right)
\end{aligned}
$$
}
{\item 
$$
\begin{aligned}
\sqrt{T}(\hat{\phi}-\phi) &= \left(\frac{1}{T} \sum x_{t-1}^{2}\right)^{-1}\left(\frac{1}{\sqrt{T}} \sum x_{t-1} v_{t}\right) \\
\left(\frac{1}{T} \sum x_{t-1}^{2}\right)^{-1}  &\xrightarrow{p} \mathbb{E}\left(x_{t-1}^{2}\right)^{-1}=\frac{1-\phi^{2}}{2} \\
\left(\frac{1}{\sqrt{T}} \sum x_{t-1} v_{t}\right) &\xrightarrow{d} N\left(0, \mathbb{E}\left(x_{t-1}^{2} v_{t}^{2}\right)\right)
\end{aligned}
$$

Use the fact that $x_{t} \perp v_{t} \forall t$

$$
\begin{aligned}
\left(\frac{1}{\sqrt{T}} \sum x_{t-1} v_{t}\right) \xrightarrow{d} & N\left(0, \mathbb{E}\left(x_{t-1}^{2}\right) \mathbb{E}\left(v_{t}^{2}\right)\right)
\end{aligned}
$$

Combine the two: (Slutsky)

$$
\sqrt{T}(\hat{\phi}-\phi) \xrightarrow{d} N\left(0, \frac{1-\phi^{2}}{2}\right)
$$
}
{\item 
Write in matrix notation:

$$
\left[\begin{array}{l}
y_{t} \\
x_{t}
\end{array}\right]=\underbrace{\left[\begin{array}{ll}
x_{t-1} & 0 \\
0 & x_{t-1}
\end{array}\right]}_{X}\left[\begin{array}{l}
\alpha \\
\phi
\end{array}\right]+\underbrace{\left[\begin{array}{l}
\varepsilon_{t} \\
v_{t}
\end{array}\right]}_{\eta_{t}}
$$

Apply GMM:

$$
\sqrt{T}\left(\begin{array}{l}
\hat{\alpha}-\alpha \\
\hat{\phi}-\phi
\end{array}\right) \xrightarrow{\alpha} N(0, \Omega)
$$

And $\Omega=\mathbb{E}\left(X X^{\prime}\right)^{-1} \mathbb{E}\left(\left(\eta^{\prime} X\right)^{\prime}(\eta X)^{\prime}\right) \mathbb{E}\left(X X^{\prime}\right)^{-1}$

$$
\begin{aligned}
\mathbb{E}(X X^\prime)^{-1} & =\left[\begin{array}{cc}
\mathbb{E}\left(x_{t-1}^{2}\right)^{-1} & 0 \\
0 & \mathbb{E}\left(x_{t-1}^{2}\right)^{-1}
\end{array}\right]=\frac{1-\phi^{2}}{2}\left[\begin{array}{ll}
1 & 0 \\
0 & 1
\end{array}\right] \\
\eta^{\prime} X & =\left[\begin{array}{ll}
\varepsilon_{t} \\
v_{t}
\end{array}\right]^{\prime}\left[\begin{array}{cc}
x_{t-1} & 0 \\
0 & x_{t-1}
\end{array}\right]=\left[\begin{array}{ll}
\varepsilon_{t} x_{t-1} & v_{t} x_{t-1}
\end{array}\right] \\
\mathbb{E}\left(\left(\eta^{\prime} X\right)^{\prime}(\eta X)^{\prime}\right) & =\left[\begin{array}{ll}
\mathbb{E}\left(\varepsilon_{t}^{2} x_{t-1}^{2}\right) & \mathbb{E}\left(\varepsilon_{t} v_{t} x_{t-1}^{2}\right) \\
\mathbb{E}\left(\varepsilon_{t} v_{t} x_{t-1}^{2}\right) & \mathbb{E}\left(v_{t}^{2} x_{t-1}^{2}\right)
\end{array}\right] \\
& =\left[\begin{array}{ll}
\mathbb{E}\left(\varepsilon_{t}^{2}\right) \mathbb{E}\left(x_{t-1}^{2}\right) & \mathbb{E}\left(\varepsilon_{t}\right) \mathbb{E}\left(v_{t}\right) \mathbb{E}\left(x_{t-1}^{2}\right) \\
\mathbb{E}\left(\varepsilon_{t}\right) \mathbb{E}\left(v_{t}\right) \mathbb{E}\left(x_{t-1}^{2}\right) & \mathbb{E}\left(v_{t}^{2}\right) \mathbb{E}\left(x_{t-1}^{2}\right)
\end{array}\right] \\
& =\left[\begin{array}{ll}
1 & 0 \\
0 & 2
\end{array}\right] \cdot \frac{2}{1-\phi^{2}} \\
\Omega & =\frac{1-\phi^{2}}{2}\left[\begin{array}{ll}
1 & 1 \\
1 & 2
\end{array}\right]
\end{aligned}
$$
}
\end{enumerate}
}
{\item 
$$
C I=\left[\hat{\hat{\alpha}} \pm 1.96 \frac{1}{\sqrt{T}} \sqrt{\frac{1-\hat{\phi}^{2}}{2}}\right]=[1.203 ; 1.396]
$$
}
\end{enumerate}
}
}

\newpage
{
\subsection*{Honor\'e}

{
\subsubsection*{Exercise 1}

\begin{enumerate}[label=(\alph*)]
{\item 
The issue is that the asymptotics for OLS only hold for $n \rightarrow \infty$, with a fixed number of parameters. But as $n \rightarrow \infty$, the number of the $\alpha_{i}$ also goes to infinity.

Then, we cannot say anything about the distributions of $(\beta, \gamma, \delta)$.
}
{\item 
We should use first differences ( $\alpha_{i}$ drop out):

$$
\Delta y_{i t}=\Delta x_{i t}^{\prime} \beta+\Delta x_{i t-1}^{\prime} \gamma+\Delta x_{i t-2}^{\prime} \delta+\Delta \varepsilon_{i t}
$$

We need to start at $T=4$, otherwise the explanatory variables are not well defined. Also note, that by assumption:

$$
\mathbb{E}\left(\Delta \varepsilon_{i t} \mid x_{i t}, x_{i t-1}, x_{i t-2}, \ldots\right)=0
$$

Thus, the errors are uncorrelated, and OLS should recover the coefficients.

\color{red}
Not super sure if GMM would be better with moment conditions:

$$
\mathbb{E}\left(\Delta \varepsilon_{i t} x_{i t}\right)=\mathbb{E}\left(\left(\Delta y_{i t}-\Delta x_{i t}^{\prime} \beta-\Delta x_{i t-1}^{\prime} \gamma-\Delta x_{i t-2}^{\prime} \delta\right) x_{i t}\right)=0
$$
\color{black}
}
\end{enumerate}
}
{
\subsubsection*{Exercise 2}

\begin{enumerate}[label=(\alph*)]
{\item 
$$
\begin{aligned}
& \sqrt{n}(\hat{\beta}-\beta) \xrightarrow{d} N\left(0, A^{-1} B A^{-1}\right) \\
& A=\mathbb{E}\left(x_{i}^{2} \exp \left(x_{i} \beta\right)^{2}\right)=\mathbb{E}\left(x_{i}^{2} \exp \left(2 x_{i}\beta\right)\right) \\
& B=\mathbb{E}\left(\varepsilon_{i}^{2} x_{i}^{2} \exp \left(2 x_{i} \beta\right)\right)=\mathbb{E}\left(\mathbb{E}\left(\varepsilon_{i}^{2} \mid x_{i}\right) x_{i}^{2} \exp \left(2 \beta x_{i}\right)\right) \\
& \quad=\mathbb{E}\left(x_{i}^{2} \exp \left(3 \beta x_{i}\right)\right) \\
& \sqrt{n}(\hat{\beta}-\beta) \xrightarrow{d} N\left(0, \frac{\mathbb{E}\left(x_{i}^{2} \exp \left(3 \beta x_{i}\right)\right)}{\left[\mathbb{E}\left(x_{i}^{2} \exp \left(2 x_{i}\beta\right)\right)\right]^{2}}\right)
\end{aligned}
$$
}
{\item 
$$
\begin{aligned}
& \sqrt{n}(\hat{\beta}-\beta) \xrightarrow{d} N\left(0, \Gamma^{-1} S \Gamma^{-1}\right) \\
& \Gamma=\mathbb{E}\left(-x_{i} \exp \left(x_{i} \beta\right)\right) \\
& S=V\left(f\left(x_{i}, \beta\right)\right)=V\left(y_{i}-\exp \left(x_{i} \beta\right)\right)=V\left(\varepsilon_{i}\right)=\mathbb{E}\left(\exp \left(x_{i} \beta\right)\right) \\
& \sqrt{n}(\hat{\beta}-\beta) \xrightarrow{d} N\left(0, \frac{\mathbb{E}\left(\exp \left(x_{i} \beta\right)\right)}{\left[\mathbb{E}\left(x_{i} \exp \left(x_{i} \beta\right)\right)\right]^{2}}\right)
\end{aligned}
$$
}
{\item 
$$
\begin{aligned}
& \sqrt{n}(\hat{\beta}-\beta) \xrightarrow{d} N\left(0,\left(G^{\prime} S^{-1} G\right)^{-1}\right) \\
& G=\mathbb{E}\left[\begin{array}{l}
-x_{i} \exp \left(x_{i} \beta\right) \\
-x_{i}^{2} \exp \left(x_{i} \beta+2\right)
\end{array}\right] \\
& S^{-1}=V\left[\begin{array}{l}
\varepsilon_{i} \\
\left(y_{i}-\exp \left(x_{i} \beta\right) \underbrace{\exp (2)}_{\text {\color{red}WTF?\color{black}}} x_{i}\right)
\end{array}\right]
\end{aligned}
$$
}
\end{enumerate}
}
{
\subsubsection*{Exercise 3}

Approximately: (Nonparametrics, slide 11)

$$
\begin{aligned}
\operatorname{Bias}(\hat{f}(x)) & \cong \frac{1}{2} h^{2} f^{\prime \prime}(x) \int v^{2} K(v) d v \\
& =\frac{1}{2} h^{2} f^{\prime \prime}(x) \int v^{2} \frac{1}{2} d v \\
& =\frac{1}{2} h^{2} \frac{1}{4} \exp \left(-\frac{x}{2}\right)\left(\frac{x^{2}}{2}-1\right) \frac{1}{2}\left[\frac{1}{3} v^{3}\right]_{-1}^{1} \\
& =\frac{1}{2} h^{2} \frac{1}{4} \exp \left(-\frac{x}{2}\right)\left(\frac{x^{2}}{2}-1\right) \frac{1}{3} \\
\operatorname{Bias}(\hat{f}(1)) & \cong \frac{1}{n^{1 / 4}} \exp \left(-\frac{1}{2}\right)\left(-\frac{1}{16}\right)
\end{aligned}
$$

$$
\begin{aligned}
V(\hat{f}(x)) & \cong \frac{1}{n h} f(x) \int K(v)^{2} d v \\
& =\frac{1}{3 n^{3 / 4}} \frac{1}{2} \exp \left(-\frac{x}{2}\right) \int_{-1}^{1} \frac{1}{4} d v \\
& =\frac{1}{3 n^{3 / 4}} \frac{1}{2} \exp \left(-\frac{x}{2}\right) \frac{1}{2} \\
V(\hat{f}(1)) & =\frac{1}{12} \frac{1}{n^{3 / 4}} \exp \left(-\frac{1}{2}\right) \\
\operatorname{MSE}(\hat{f}(1)) & =\left[\frac{1}{n^{1 / 4}} \exp \left(-\frac{1}{2}\right)\left(-\frac{1}{16}\right)\right]^{2}+\frac{1}{12} \frac{1}{n^{3 / 4}} \exp \left(-\frac{1}{2}\right) \\
& =n^{-1 / 2} \exp (-1) 2^{-8}+n^{-3 / 4} \frac{1}{12} \exp \left( -\frac{1}{2} \right) \\
& =\operatorname{const}_1 n^{-1 / 2}+\operatorname{const}_2 n^{-3 / 4}
\end{aligned}
$$
}
{
\subsubsection*{Exercise 4}

\begin{enumerate}[label=(\alph*)]
{\item 
It is given by $\phi\left(x_{0}^{\prime} \beta\right) \beta_{l}$, where $\phi(\cdot)$ is the pdf of a standard normal, and $\beta_{l}$ the coefficient on the explanatory variable.
}
{\item 
We can estimate the marginal effect by:

$$
g(\hat{\beta})=\phi\left(x_{0}^{\prime} \hat{\beta}\right) \hat{\beta}_{l}
$$

Also recall from the lecture that

$$
\sqrt{n}(\hat{\beta}-\beta) \xrightarrow{d} N(0, \Sigma)
$$

Now, we can apply the delta-method as $g(\cdot)$ is a non-linear function of $\beta$.

$$
\sqrt{n}(g(\hat{\beta})-g(\beta)) \xrightarrow{d} N\left(0,\left(\frac{\partial g(\beta)}{\partial \beta}\right)^{\prime} \Sigma \frac{\partial{g}(\beta)}{\partial \beta}\right)
$$
}
{\item 
\color{red} Very long answer... \color{black}
}
\end{enumerate}
}
}
\newpage
\section{Macroeconomics Final 2017 / 18}

{
\subsection*{Exercise 1}

\begin{enumerate}[label=(\arabic*)]
{
\item 
$u_{t}$ is what we call a cost-push-shock, and it is exogenous. Say cost increases due to an unpredictable shock (e.g. Covid).
All else equal, $p_{t}$ increases \& thus inflation increases.

Usually: $u_{t}=\left(\hat{y}_{t}^{e}-\hat{y}_{t}^{n}\right) \kappa$

Divine coincidence is broken due to $u_{t}$ influencing $\pi_{t}$ independently of output.
}
{
\item 
In class we found that

$$
r_{t}^{e}=\rho+\sigma \mathbb{E}_{t}\left(y_{t+1}^{e}-y_{t}^{e}\right)=\rho+\mathbb{E}_{t}\left(y_{t+1}^{e}-y_{t}^{e}\right)
$$

Therefore, we conclude that $\varepsilon_{t}$ is the expected change in the efficient output. 
Since $\mathbb{E} \left( \varepsilon_{t} \right) = 0$, we have that $y_t^e$ is also white noise. 
Additionally, $\varepsilon_t$ captures anything that influences the efficient output.
}
{
\item 
Conjecture: 

$$
\begin{aligned}
\pi_{t}=\psi_{\pi} u_{t}+\delta_{\pi} \varepsilon_{t} \\
x_{t}=\psi_{x} u_{t}+\delta_{x} \varepsilon_{t}
\end{aligned}
$$

Plug conjectures in:

$$
\begin{aligned}
\psi_{\pi} u_{t}+\delta_{\pi} \varepsilon_{t} & =\beta \cdot 0+\kappa\left(\psi_{x} u_{t}+\delta_{x} \varepsilon_{t}\right)+u_{t} \\
& =\left(\kappa \psi_{x}+1\right) u_{t}+\kappa \delta_{x} \varepsilon_{t} \\
\psi_{x} u_{t}+\delta_{x} \varepsilon_{t} & =0-\left(\phi_{\pi}\left(\psi_{\pi} u_{t}+\delta_{\pi} \varepsilon_{t}\right)-0-\varepsilon_{t}\right) \\
& =-\phi_{\pi} \psi_{\pi} u_{t}+\left(-\phi_{\pi} \delta_{\pi}+1\right) \varepsilon_{t}
\end{aligned}
$$

Therefore:

$$
\begin{aligned}
\psi_{\pi} &= \kappa \psi_{x}+1 \\
\psi_{x} &= -\phi_{\pi} \psi_{\pi} \\
\Longrightarrow 
\left(\psi_{\pi}, \psi_{x}\right) &= \left(\frac{1}{1+\kappa \phi_{\pi}}, \frac{-\phi_{\pi}}{1+\kappa \phi_{\pi}}\right)
\end{aligned}
$$

$$
\begin{aligned}
\delta_{\pi} &= \kappa \delta_{x} \\
\delta_{x} &= -\phi_{\pi} \delta_{\pi}+1 \\
\Longrightarrow 
\left(\delta_{\pi}, \delta_{x}\right) &= \left(\frac{\kappa}{1+\kappa \phi_{\pi}}, \frac{1}{1+\kappa \phi_{\pi}}\right)
\end{aligned}
$$

Together, we obtain that:

$$
\begin{aligned}
& \pi_{t}=\frac{1}{1+\kappa \phi_{\pi}} u_{t}+\frac{\kappa}{1+\kappa \phi_{\pi}} \varepsilon_{t} \\
& x_{t}=\frac{-\phi_{\pi}}{1+\kappa \phi_{\pi}} u_{t}+\frac{1}{1+\kappa \phi_{\pi}} \varepsilon_{t}
\end{aligned}
$$
}
{
\item 
\begin{align*}
    \min _{\phi_{\pi}}& \operatorname{Var}\left(x_{t}\right)+\vartheta \operatorname{Var}\left(\pi_{t}\right) \\
    \min _{\phi_{\pi}}& \left[\left(\frac{-\phi_{\pi}}{1+\kappa \phi_{\pi}}\right)^{2}+\left(\frac{1}{1+\kappa \phi_{\pi}}\right)^{2} \vartheta\right] \sigma_{u}^{2} +\left[\left(\frac{-1}{1+\kappa \phi_{\pi}}\right)^{2}+\left(\frac{-\kappa}{1+\kappa \phi_{\pi}}\right)^{2} v\right] \sigma_{\varepsilon}^{2} \\
    \min _{\phi_{\pi}}& \frac{1}{\left(1+\kappa \phi_{\pi}\right)^{2}}\left[\left(\phi_{\pi}^{2}+\vartheta\right) \sigma_u^{2} +\left(1+\kappa^{2} \vartheta\right) \sigma_{\varepsilon}^{2}\right]
\end{align*}

FOC:

\begin{align*}
    & 2 \phi_\pi \sigma_u^2\left(1+\kappa \phi_\pi\right)^2 -2\left(1+\kappa \phi_\pi\right) \kappa \left[\left(\phi_\pi^2+\vartheta\right) \sigma_u^2+\left(1+\kappa^2 \vartheta\right) \sigma_{\varepsilon}^2\right]=0 \\
    & \Longleftrightarrow \phi_\pi+\kappa \phi_\pi^2=\kappa \phi_\pi^2+\vartheta \kappa+\left(1+\kappa^2 \vartheta\right)\frac{\sigma_{\varepsilon}^2}{\sigma_u^2} \kappa \\
    & \Longleftrightarrow \phi_\pi= \vartheta\kappa+\kappa\left(1+\kappa^2 \vartheta\right) \frac{\sigma_\varepsilon^2}{\sigma_u^2}
\end{align*}

As cost-push-shocks become arbitrarily small ( $\sigma^{2}_u \rightarrow 0$ ) $\phi_\pi$ explodes since reacting to inflation is more and more important. This means inflation stabilization would be sufficient to stabilize output, i.e. the divine coincidence would work again.
}
{
\item 
Flexible prices: $\mu_{t}=\mu \quad \forall t$ and $x_{t}=0 \quad \forall t$

$$
\begin{aligned}
\longrightarrow & \pi_{t}=\beta \mathbb{E}_{t}\left(\pi_{t+1}\right)+u_{t} ; \mathbb{E}_{t}\left(\pi_{t+1}\right)=\phi_{\pi} \pi_{t}-\varepsilon_{t} \\
\longrightarrow & \pi_{t}=\beta \phi_{\pi} \pi_{t}-\beta \varepsilon_{t}+u_{t} \\
& \pi_{t}=\frac{1}{1-\beta \phi_{\pi}} v_{t}-\frac{\beta}{1-\beta \phi_{\pi}} \varepsilon_{t}
\end{aligned}
$$

\color{red} Nope. Forget monetary policy, solve model. Plug in Taylor-rule etc. \color{black}
}
\end{enumerate}
}
\newpage
\section{Microeconomics Midterm 2018 / 19}

{
\subsection*{Schmidt}

{
\subsubsection*{Exercise 1}

\begin{enumerate}[label=(\alph*)]
{\item 
Violation of WARP:

$$
\left|\begin{array}{l}
p y^{\prime} \leqslant w \\
p^{\prime} y \leqslant w^{\prime}
\end{array}\right|
$$

Plug in the prices \& incomes (by Walras Law):

$$
\begin{aligned}
& \left|\begin{array}{rl}
30(12+x) & \leqslant 600 \\
540 & \leqslant 360+24 x
\end{array}\right| \\
& \Leftrightarrow\left|\begin{array}{c}
x \leq 8 \\
7.5 \leqslant x
\end{array}\right|
\end{aligned}
$$

Thus, WARP is violated if $x \in[7.5,8]$.
}
{\item 
For this, bundle from year 2 must be affordable in year 1:

$$
x \leq 8
$$

But we exclude all $x$ for which WARP is violated and find: $x \in[0,7.5)$
}
{\item 
The quantity has increased, so the income must have decreased to find $\frac{\partial y_{2}}{\partial w}<0$ :

$$
\begin{aligned}
600 & <360+24 x \\
\Leftrightarrow \quad 10 & <x
\end{aligned}
$$
}
\end{enumerate}
}
{
\subsubsection*{Exercise 2}

\begin{enumerate}[label=(\alph*)]
{\item 
Use hint because if $g(h(\cdot))$ represents preferences, then any strictly monotone transformation does it as well.
Thus $h(\cdot)$ represents the preferences \& is homogeneous. 
Call $h(\cdot)$ now $u(\cdot)$: 

EMP:

\begin{align*}
    \min _{x} p x \\
    \text{ s.t. } u(x)=1
\end{align*}

FOC:

\begin{align*}
    p_{l}-\lambda \frac{\partial u(x)}{\partial x_{l}}=0 \quad \forall l
\end{align*}

by Euler

$$
e(p, u=1)=\sum_{l} p_{l} x_{l}=\lambda \sum_{l} x_{l} \frac{\partial u(x)}{\partial x_{l}}=\lambda
$$

now let $u(x)=u$:

\begin{align*}
    \min _{x} p x \\
    \text{ s.t. } u(x)=u
\end{align*}

FOC:

$$
\begin{aligned}
    p_{l}-\lambda \frac{\partial u(x)}{\partial x_{l}}=0 \quad \forall l \\
    e(p, u)=\sum_{l} p_{l} x_{l} 
    =\lambda \sum_{l} x_{l} \frac{\partial u(x)}{\partial x_{l}} 
    =\lambda u=u \cdot e(p)
\end{aligned}
$$
}
{\item 
UMP:

$$
\begin{gathered}
    \max _{x} u(x) \\
    \text { st. } p x=1
\end{gathered}
$$

FOC:

$$
\begin{gathered}
\frac{\partial u(x)}{\partial x_{l}}-\lambda p_{l}=0 \quad \forall l \\
\Leftrightarrow \frac{\partial u(x)}{\partial x_{l}}=\lambda p_{l} \quad \mid \cdot x_{l} \\
\frac{\partial u(x)}{\partial x_{l}} x_{l}=\lambda p_l x_{l}
\end{gathered}
$$

Sum over $x_{l}$ and use Euler:

\begin{align*}
    \sum_{l} \frac{\partial u(x)}{\partial x_{l}} x_{l}=\lambda \sum_{l} p_l x_{l}  \tag{I} \\
    u\left(x^{*}\right)=v(p)=\lambda p x=\lambda
\end{align*}


Let $p x=1$ : Same $F O C$ and up to (I) nothing changes:

\begin{align*}
    & \sum_{l} \frac{\partial u(x)}{\partial x_{l}} x_{l}=\lambda \sum_l p_l x_{l} \\
    & u\left(x^{*}\right)=v(p, w)=\lambda p x=\lambda w=v(p) w \tag{II}
\end{align*}
}
{\item 
Follows from applying Roy's identity to (II):

$$
x_{l}(p, w)=-\frac{\frac{\partial v(p, w)}{\partial p_{l}}}{\frac{\left.\partial v p_{1} w\right)}{\partial w}}=-\frac{\frac{\partial v(p)}{\partial p_{l}} w}{v(p)}=x_{l}(p) w
$$
}
\end{enumerate}
}
{
\subsubsection*{Exercise 3}

\begin{enumerate}[label=(\alph*)]
{\item 
Leontief implies: $x_{1}^{*}=x_{2}^{*}$ and $u=x_{1}^{*}=x_{2}^{*}$ Thus, they must be able to afford the old bundle as Leontief does not allow for substitutions. They will also choose to consume it.

Before moving: $p_{1}=p_{2}=1$

$$
\rightarrow \quad x_{1}^{*}=x_{2}^{*}=\frac{w}{2}=500=u_{0}
$$

After moving: Set $u_{1}=u_{0}$. Thus

$$
x_{1}^{*}=x_{2}^{*}=500
$$

to afford this:

$$
e(p, u)=500(1+4)=2500
$$

As initial wage is 1000, we have $R=1500$.

This is the negative of CV.
}
{\item 
This is Cobb Douglas utility. Thus

$$
x_{1}^{*}=\frac{w}{2 p_{1}} \quad ; \quad x_{2}^{*}=\frac{w}{2 p_{2}}
$$

Before moving:

$$
\begin{aligned}
& x_{1}^{*}=x_{2}^{*}=500 \\
& v(p, w)=500
\end{aligned}
$$

After moving:

$$
v(p, w)=(w+R)\left(\frac{1}{2 p_{1}}\right)^{1 / 2}\left(\frac{1}{2 p_{2}}\right)^{1 / 2}=500
$$

$$
\Leftrightarrow \quad w+R=2000 \Leftrightarrow R=1000
$$

As CD utility allows for substitution, they choose to buy less of $x_{1}$ as it has become much more expensive.
}
\end{enumerate}
}
{
\subsubsection*{Exercise 4}

\begin{enumerate}[label=(\alph*)]
{\item 
This agent exhibits decreasing absolute risk aversion.

$$
r^{A}=-\frac{u^{\prime \prime}(x)}{u^{\prime}(x)}=-\frac{-\rho\left(1_{-\rho}\right)_{x}-\rho-1}{(1-\rho)_{x}^{-\rho}}=\rho x^{-1}
$$
}
{\item 
Agent maximizes expected utility:

$$
\max _{a} \int u(W-a W+a W \pi) d F(\pi)
$$

Assume interior solution:

$$
\text { FCC: } \quad \int u^{\prime}(w-a w+a W \pi)(\pi w-w) d F(\pi)=0
$$

Plug in functional form:

$$
\begin{aligned}
& \int(1-\rho)(1-a+a \pi)^{-\rho}(\pi-1) w^{1-\rho} d F(\pi)=0 \\
\Leftrightarrow & \underbrace{(1-\rho) w^{1-\rho}}_{\neq 0} \int(1-a+a \pi)^{-\rho}(\pi-1) d F(\pi)=0 \\
\Rightarrow \quad & \quad \int(1-a+a \pi)^{-\rho}(\pi-1) d F(\pi)=0
\end{aligned}
$$

This expression implicitly defines $a^{*}$ and is independent of $W$.
}
\end{enumerate}
}
}

{
\subsection*{Gottardi}

{
\subsubsection*{Exercise 1}

\begin{enumerate}[label=(\roman*)]
{\item 
\underline{Consumer A:}

$$
\begin{aligned}
& \max _{x_{1}^{A}, x_{2}^{A}} \ln \left(x_{1}^{A}\right)+2 \ln \left(x_{2}^{A}\right) \\
& \text { s.t. } p x_{1}^{A}+x_{2}^{A}=16 p
\end{aligned}
$$

FOC:

$$
\begin{aligned}
    & \left[x_{1}^{A}\right]: \frac{1}{x_{1}^{A}}-\lambda p=0 \\
    & {\left[x_{2}^{A}\right]: 2 \frac{1}{x_{2}^{A}}-\lambda=0} \\
    & \longrightarrow x_{2}^{A}=2 x_{1}^{A} p\longrightarrow x_{1}^{A}=\frac{16}{3}
\end{aligned}
$$

\underline{Consumer B:}

\begin{align*}
    \max _{x_{1}^{B} x_{2}^{B}} \ln \left(x_{1}^{B}\right)+\ln \left(x_{2}^{B}\right) \\
    \text{s.t. } p x_{1}^{B}+x_{2}^{B}=12
\end{align*}

FOC: 

\begin{align*}
    & \left[x_{1}^{B}\right]: \frac{1}{x_{1}^{B}}-\lambda p=0 \\
    & \left[x_{2}^{B}\right]: \frac{1}{x_{2}^{B}}-\lambda=0 \\
    & \longrightarrow x_{2}^{B}=x_{1}^{B} p \longrightarrow x_{2}^{B}=6
\end{align*}

\underline{Markets:}

$$
x_{1}^{A}+x_{1}^{B}=16 \quad ; \quad x_{2}^{A}+x_{2}^{B}=12
$$

$$
\Leftrightarrow x_{1}^{B}=16\frac{2}{3} \quad \Leftrightarrow \quad x_{2}^{A}=6
$$

Combine with either $F O C$ to find:

$$
\begin{aligned}
x_{2}^{B} &= p x_{1}^{B} \\
6 &= p 16\frac{2}{3} \\
p &= 9 / 16
\end{aligned}
$$

\underline{Competitive Equilibrium:}

$$
\begin{aligned}
\left(x_{1}^{A}, x_{2}^{A}\right) & =(16 \cdot \frac{1}{3},6) \\
\left(x_{1}^{B}, x_{2}^{B}\right) & =(16 \cdot \frac{2}{3},6) \\
p & =9 / 16
\end{aligned}
$$
}
{\item 
$$
M R S^{A}=\frac{x_{2}^{A}}{2 x_{1}^{A}} \stackrel{!}{=} M R S^{B}=\frac{x_{2}^{B}}{x_{1}^{B}}
$$

Use market clearing: $\quad x_{1}^{B}=16-x_{1}^{A}$

\begin{align*}
& \longrightarrow \quad \frac{x_{2}^{A}}{2 x_{1}^{A}}=\frac{12-x_{2}^{A}}{16-x_{1}^{A}} \\
& \Leftrightarrow \quad \frac{16-x_{1}^{A}}{2 x_{1}^{A}}=\frac{12}{x_{2}^{A}}-1 \\
& \Leftrightarrow \quad \frac{16-x_{1}^{A}+2 x_{1}^{A}}{2 x_{1}^{A}}=\frac{12}{x_{2}^{A}} \\
& \Leftrightarrow \quad x_{2}^{A}=\frac{24 x_{1}^{A}}{16+x_{1}^{A}} \tag{I}
\end{align*}

\begin{figure}[!htp]
    \centering
    \includegraphics[width=.75\textwidth]{images/2018_19_1.png}
\end{figure}
}
{\item 
Plugging into $(I)$ :

$$
\begin{aligned}
& 8=\frac{24 \cdot 8}{16+8} \\
& 8=8
\end{aligned}
$$

Yes, it is PE.

Find transfers:

$$
\begin{aligned}
& T^{A}=\left[\begin{array}{l}
x_{1}^{A} \\
x_{2}^{A}
\end{array}\right]-\left[\begin{array}{l}
w_{1}^{A} \\
w_{2}^{A}
\end{array}\right]=\left[\begin{array}{l}
8 \\
8
\end{array}\right]-\left[\begin{array}{l}
16 \\
0
\end{array}\right]=\left[\begin{array}{c}
-8 \\
8
\end{array}\right] \\
& T^{B}=\left[\begin{array}{l}
x_{1}^{B} \\
x_{2}^{B}
\end{array}\right]-\left[\begin{array}{l}
w_{1}^{B} \\
w_{2}^{B}
\end{array}\right]=\left[\begin{array}{l}
8 \\
4
\end{array}\right]-\left[\begin{array}{c}
0 \\
12
\end{array}\right]=\left[\begin{array}{c}
8 \\
-8
\end{array}\right]
\end{aligned}
$$

Prices given by MRS:

$$
p=M R S^{A}=M R S^{B}=1 / 2
$$
}
\end{enumerate}
}
{
\subsubsection*{Exercise 2}
\begin{enumerate}[label=(\roman*)]
{\item 
at $t=0$ :

$$
q_{1} \theta_{1}+q_{2} \theta_{2}=0
$$

at $t=1$ and $s=1$ :

$$
x_{1}=2 \theta_{1}+\theta_{2}+4
$$

at $t=1$ and $s=2$ : 

$$
\quad x_{2}=\theta_{1}+2 \theta_{2}+8
$$
}
{\item 
consumer solves:

$$
\begin{aligned}
& \max _{\theta_{1}, \theta_{2}} \frac{1}{2}\left[\ln \left(2 \theta_{1}+\theta_{2}+4\right)+\ln \left(\theta_{1}+2 \theta_{2}+8\right)\right] \\
& \text { s.t. } q_{1} \theta_{1}+q_{2} \theta_{2}=0
\end{aligned}
$$

FOCs:

$$
\begin{aligned}
& \left[\theta_{1}\right]:\left(2 \theta_{1}+\theta_{2}+4\right)^{-1}+\frac{1}{2}\left(\theta_{1}+2 \theta_{2}+8\right)^{-1}-\lambda q_{1}=0 \\
& \left[\theta_{2}\right]: \frac{1}{2}\left(2 \theta_{1}+\theta_{2}+4\right)^{-1}+\left(\theta_{1}+2 \theta_{2}+8\right)^{-1}-\lambda q_{2}=0
\end{aligned}
$$

Suppose $q_{1}=q_{2}=1$ :

\begin{align*}
\frac{1}{2}\left(2 \theta_{1}+\theta_{2}+4\right)^{-1} & =\frac{1}{2}\left(\theta_{1}+2 \theta_{2}+8\right)^{-1} \\
2 \theta_{1}+\theta_{2}+4 & =\theta_{1}+2 \theta_{2}+8 \\
\theta_{1} & =\theta_{2}+4 \tag{II}
\end{align*}

Market clearing: $\theta_{1}=-\theta_{2}=0$ as there is only one consumer. This violates (II). Thus $q_{1}=q_{2}=1$ is not possible!

This result is the consequence of risk-aversion. The consumer is poorer in state 1, so she wants to buy insurance against it via asset 1. Unfortunately, she cannot because there is nobody else in the economy to trade with. To offset this excess demand for asset $1$ we must have $q_{1}>q_{2}$ which makes it less attractive.
}
{\item 
There is no risk aversion and therefore the assets are not interesting as an insurance as they have the same expected return. As a consequence the prices reflect the state probabilities. As $\pi_{1}=\pi_{2}=1 / 2$, will find $q_{1}=q_{2}$ and $q_{1}=q_{2}=1$ is a CE.
}
\end{enumerate}
}
}\newpage
\section{Microeconomics Midterm 2019 / 20}

{
\subsection*{Schmidt}

{
\subsubsection*{Exercise 1}

\begin{enumerate}[label=(\roman*)]
{\item 
First, find incomes:

$$
w^{0}=42 \quad w^{1}=36 \quad w^{2}=50
$$

Look for violations of WARP:

\begin{table}[!htp]
    \centering
    \begin{tabular}{|c|c|l|l|l|l|}
    \hline
    $t$ & $t^{\prime}$ & $p^{t} x^{t^{1}}$ & $\sum$ & $w^{t}$ & revealed preferences \\
    \hline \multirow{2}{*}{0} & 1 & $p^{0} x^{1}=48$ & $>$ & 42 & - \\
    \cline { 2 - 6 } & 2 & $p^{0} x^{2}=40$ & $<$ & 42 & $x^{0}>x^{2}$ \\
    \hline \multirow{2}{*}{1} & 0 & $p^{1} x^{0}=33$ & $<$ & 36 & $x^{1}>x^{0}$ \\
    \cline { 2 - 6 } & 2 & $p^{1} x^{2}=39$ & $>$ & 36 & - \\
    \hline \multirow{2}{*}{2} & 0 & $p^{2} x^{0}=52$ & $>$ & 50 & - \\
    \cline { 2 - 6 } & 1 & $p^{2} x^{1}=48$ & $<$ & 50 & $x^{2}>x^{1}$ \\
    \hline
    \end{tabular}
\end{table}

From the table we see that we never have $p^{t} x^{t^{\prime}} \leq w^{t}$ and $p^{t^{\prime}} x^{t} \leq w^{t^{\prime}}$. Therefore, WARP is satisfied.
}
{\item 
From the last row we have $x^{0}>x^{2}$ and $x^{2}>x^{1}$

Transitivity implies $x^{0}>x^{\prime}$ but we found the opposite: $x^{\prime}>x^{0}$. Therefore, transitivity is violated.
}
\end{enumerate}
}
{
\subsubsection*{Exercise 2}

\begin{enumerate}[label=(\alph*)]
{\item 
Consumer 1: at optimum $e_{1}(\cdot)=w_{1}$ \& $u_{1}=v_{1}(\cdot)$

$$
\begin{aligned}
w_{1} & =v_{1}\left(p, w_{1}\right) \sqrt{p_{1} p_{2}} \\
\Leftrightarrow v_{1}\left(p, w_1\right) & =\frac{w_{1}}{\sqrt{p_{1} p_{2}}}
\end{aligned}
$$

Use Roy's identity:

$$
\begin{aligned}
x_{1}^1(p, w) & =-\frac{\frac{\partial v_{1}\left(p_{1} w\right)}{\partial p_{1}}}{\frac{\partial v_{1}\left(p_{1} w\right)}{\partial w_{1}}}=-\frac{-\frac{1}{2} \frac{w_{1}}{\sqrt{p_{2}} p_{1}^{-3 / 2}}}{\frac{1}{\sqrt{p_{1} p_{2}}}} \\
& =\frac{w_{1}}{2 p_{1}}
\end{aligned}
$$

By symmetry: $x_{2}^1\left(p_{1} w\right)=\frac{w_{1}}{2 p_{2}}$

Consumer 2: Transform utility function.

$$
u_{2}\left(x_{1}, x_{2}\right)=x_{1}^{\frac{3}{3+a}} x_{2}^{\frac{a}{3+a}}
$$

This is standard Cobb-Dauglas:

$$
x_{1}^{2}(p, w)=\frac{3}{3+a} \frac{w_{2}}{p_{1}} ; x_{2}^1 (p, w)=\frac{a}{3+a} \frac{w_{2}}{p_{2}}
$$
}
{\item 
Good 1: $\quad x_{1}^{1}+x_{1}^{2}=\frac{1}{p_{1}}\left[\frac{1}{2} w_{1}+\frac{3}{3+a} w_{2}\right]$

$$
\longrightarrow \frac{1}{2}=\frac{3}{3+a} \Longleftrightarrow a=3
$$

Good 2: $x_{2}^{1}+x_{2}^{2}=\frac{1}{p_{2}}\left[\frac{1}{2} w_{1}+\frac{a}{3+a} w_{2}\right]$

$$
\longrightarrow \frac{1}{2}=\frac{a}{3+a} \Longleftrightarrow a=3
$$

Thus $a=3$ solves the problem for both goods.
}
\end{enumerate}
}
{
\subsubsection*{Exercise 3}

\begin{enumerate}[label=(\alph*)]
{\item 
Firm solves:

\begin{align*}
    \min _{x} c(w, y) = \min_x w x \\
    \text { s.t. } f(x)=y
\end{align*}

FOC:

\begin{align*}
    w_{l}=\lambda \frac{\partial f(x)}{\partial x_{l}} \quad \forall l
\end{align*}

Use Euler:

\begin{align*}
    c(w, y)=\lambda \sum_{l} \frac{\partial f(x)}{\partial x_{l}} x_{l}=\lambda f(x)=\lambda y
\end{align*}

If $y=1: c(w, 1)=\lambda$

If $y \neq 1: c(w, y)=\lambda y=c(w, 1) y=c(w) y$
}
{\item 
We have:

$$
\begin{aligned}
& c(w, y)=w x \\
& \frac{\partial(w, y)}{\partial w_{l}}=\frac{\partial w x}{\partial w_{l}}=x_{l}
\end{aligned}
$$

And from (a):

$$
\frac{\partial c(w, y)}{\partial w_{l}}=\frac{\partial c(w, 1)}{\partial w_{l}} y
$$

Together:

$$
x_{l}=\frac{\partial c\left(w, 1\right)}{\partial w_{l}} y
$$
}
{\item 
Profits are:

$$
\pi=p f(x)-w x=p f(x)-\sum_{l} w_{l} x_{l}
$$

Plug in the $w_l$ from exercise

$$
\begin{aligned}
\pi & =p f(x)-\sum_{l} p \frac{\partial f(x)}{\partial x_{l}} x_{l} \\
& =p\left[f(x)-\sum_{l} \frac{\partial f(x)}{\partial x_{l}} x_{l}\right]=p[f(x)-f(x)]
\end{aligned}
$$

The last equality follows from CRS \& Euler's formula. Clearly, $\pi=0$.
}
\end{enumerate}
}
{
\subsubsection*{Exercise 4}

\begin{enumerate}[label=(\roman*)]
{\item 
DM maximize expected utility:

$$
\begin{aligned}
\max _{\alpha, \beta} E U(\cdot) & =\max _{\alpha, s} \int u(w-\alpha-\beta+\alpha z+\beta) d F(z) \\
& =\max _{\alpha} \int u(w-\alpha+\alpha z) d F(z)
\end{aligned}
$$

Get first order derivative:

$$
\frac{\partial E U}{\partial \alpha}=\int u^{\prime}(w-\alpha+\alpha z)(z-1) d F(z)
$$

Suppose $\alpha=0$ :

$$
\begin{aligned}
& \int u^{\prime}(w)(z-1) d F(z)
= u^{\prime}(w)\left[\int z d F(z)-1\right]>0
\end{aligned}
$$

As the expected marginal utility is positive at $\alpha=0$, the DM will invert some $\alpha>0$.
}
{\item 
As we saw in (i), $\alpha=0$ is not optimal (for both agents). They increase $\alpha$, which lowers the marginal expected utility, until $\frac{\partial E U}{\partial \alpha}=0$.
Because $v(\cdot)$ is a concave transformation of $u(\cdot)$, we know that $v^{\prime}(\cdot)$ decreases faster then $u^{\prime}(\cdot)$.
Therefore, $\int v^{\prime}(\cdot)(z-1) d F(z)=0$ is reached at a lower value of $\alpha$ than for $\int u^{\prime}(\cdot)(z-1) d F(z)$. Thus:

$$
\alpha_{v}^{*}<\alpha_{u}^{*}
$$
}
\end{enumerate}
}
}

{
\subsection*{Gottardi}

{
\subsubsection*{Exercise 1}

\begin{enumerate}[label=(\roman*)]
{\item 
\underline{Consumer A:}

\begin{align*}
\max _{x_1^A, x_2} x_1^A x_2^A \text { s.t. } p x_1^A+x_2^A=p 8
\end{align*}

FOCs

\begin{align*}
    {\left[x_1^A\right]:} & x_2^A-\lambda p=0 \\
    {\left[x_2^A\right]:} & x_1^A-\lambda=0 \\
    \rightarrow& x_2^A=p x_1^A\tag{I}
\end{align*}

Combine (I) with BC:

\begin{align*}
    x_1^A=4 \quad ; \quad x_2^A=4 p
\end{align*}

\underline{Consumer B:}

\begin{align*}
    \max _{x_{1}^{B} x_{3}^{B}} x_{1}^{B}+2 x_{2}^{B} \text{ s.t. }p x_{1}^{B}+x_{2}^{B}=6
\end{align*}

By linearity:

$$
\begin{aligned}
& x_{1}^{B}=\left\{\begin{array}{lll}
\infty & \text { if } & p<1 / 2 \\
\mathbb{R}^{+} & \text {if } & p=1 / 2 \\
0 & \text { if } & p>1 / 2
\end{array}\right. \\
& x_{2}^{B}=\left\{\begin{array}{lll}
\infty & \text { if } & p>1 / 2 \\
\mathbb{R}^{+} & \text {if } & p=1 / 2 \\
0 & \text { if } & p<1 / 2
\end{array}\right.
\end{aligned}
$$

\underline{Markets:}

$p=1 / 2$ otherwise we would have excess demand for one of the goods.

$$
\begin{aligned}
    & \longrightarrow x_{2}^{A}=2 \\
    & \longrightarrow x_{1}^{B}=8-4=4 \\
    & \longrightarrow x_{2}^{B}=6-2=4 \\
\end{aligned}
$$

\underline{Competitive Equilibrium:}

$$
\begin{aligned}
    & \left(x_{1}^{A}, x_{2}^{A}\right)=(4,2) \\
    & \left(x_{1}^{B}, x_{2}^{B}\right)=(4,4) \\
    & p=\frac{1}{2}
\end{aligned}
$$
}
{\item 
$M RS^{A}=x_{2}^{A} / x_{1}^{A} \stackrel{!}{=} M R S^{B}=1 / 2 \longrightarrow x_{2}^{A}=\frac{1}{2} x_{1}^{A}$ in blue:

\begin{figure}[!htp]
    \centering
    \includegraphics[width=.5\textwidth]{images/2019_20_1.png}
\end{figure}
}
\end{enumerate}
}
{
\subsubsection*{Exercise 2}

\begin{enumerate}
    \item LNS of preferences
    \item complete markets
    \item free disposal
\end{enumerate}

Suppose (1) is violated. Then we could construct the following situation (A violates LNS):

\begin{figure}[!htp]
    \centering
    \includegraphics[width=.75\textwidth]{images/2019_20_2.png}
\end{figure}

Although at $x$ both agents are optimizing given the prices, we could make $B$ better off without hurting $A$ if we moved to the bottom left. Thus the CE at $x$ is not PE.
}
{
\subsubsection*{Exercise 3}

\begin{align*}
    \left(w_{1}, w_{2}\right)=(9,16)
\end{align*}

\begin{enumerate}[label=(\alph*)]
{\item 
$$
\begin{array}{ll}
t=0: & q_{1} \theta_{1}+q_{2} \theta_{2}=0 \\
t=1 \text { and } s=1: & x_{1}=w_{1}+\theta_{1}+3 \theta_{2}=9+\theta_{1}+3 \theta_{2} \\
t=1 \text { and } s=2: & x_{2}=w_{2}+3 \theta_{1}+\theta_{2}=16+3 \theta_{1}+\theta_{2}
\end{array}
$$
}
{\item 
Solve the maximization problem. I already substitute $x_{1}$ and $x_{2}$ from the BC s into the EU-function :

$$
\begin{gathered}
\max _{\theta_{1} \theta_{2}} \frac{1}{2}\left[\sqrt{9+\theta_{1}+3 \theta_{2}}+\sqrt{16+3 \theta_{1}+\theta_{2}}\right] \\
\text { st. } \quad q_{1} \theta_{1}+q_{2} \theta_{2}=0
\end{gathered}
$$

FOC:

$$
\begin{aligned}
& \frac{1}{2}\left[\frac{1 / 2}{\sqrt{9+\theta_{1}+3 \theta_{2}}}+\frac{1 / 2 \cdot 3}{\sqrt{16+3 \theta_{1}+\theta_{2}}}\right]-\lambda q_{1}=0 \\
& \frac{1}{2}\left[\frac{1 / 2 \cdot 3}{\sqrt{9+\theta_{1}+3 \theta_{2}}}+\frac{1 / 2}{\sqrt{16+3 \theta_{1}+\theta_{2}}}\right]-\lambda q_{2}=0
\end{aligned}
$$

Since there is only one consumer. must have no trade equilibrium: $\theta_{1}=\theta_{2}=0$. Plug into FOCs:

$$
\begin{aligned}
    \frac{1}{2}\left[\frac{1 / 2}{3}+\frac{1 / 2 \cdot 3}{4}\right]-\lambda q_{1}=0 &\Longleftrightarrow \lambda q_{1}=\frac{1}{4}\left[\frac{1}{3}+\frac{3}{4}\right]=\frac{13}{4 \cdot 12} \\
    \frac{1}{2}\left[\frac{1 / 2 \cdot 3}{3}+\frac{1 / 2}{4}\right]-\lambda q_{2}=0 &\Longleftrightarrow \lambda q_{2}=\frac{1}{4}\left[1+\frac{1}{4}\right]=\frac{5}{4 \cdot 4} \\
    \longrightarrow & \frac{q_{1}}{q_{2}}=\frac{13}{12} \cdot \frac{4}{5}=\frac{13}{15}
\end{aligned}
$$
}
{\item 
$\mathbb{E}\left(r_{1}\right)=\frac{1}{2}(1+3)=2=\mathbb{E}\left(r_{2}\right)=\frac{1}{2}(3+1)$

Thus:

$$
\frac{q_{1}}{q_{2}}<1 \Longleftrightarrow \frac{1}{q_{2}}<\frac{1}{q_{1}} \Leftrightarrow \frac{\mathbb{E}\left(r_{1}\right)}{q_{1}}>\frac{\mathbb{E}\left(r_{2}\right)}{q_{2}}
$$

The expected rate of return for asset 1 is larger than for asset 2.

Since the consumer is richer in state 2 and risk-averse, she would like to buy asset 2 as insurance. Because she is alone in the economy, this demand for asset 2 increases $q_{2}$ relative to $q_{1}$. This in turn leads to $\frac{1}{q_{1}}>\frac{1}{q_{2}}$ and $\frac{\mathbb{E}\left(r_{1}\right)}{q_{1}}>\frac{\mathbb{E}\left(r_{2}\right)}{q_{2}}$.
}
\end{enumerate}
}
}\newpage
\section{Econometrics Final 2020 / 21}

{
\subsection*{Watson}

{
\subsubsection*{Exercise 1}

\begin{enumerate}[label=(\alph*)]
{\item 
$$
x_{t}=y_{t}+y_{t-2}
$$

(1) 

$$
\mathbb{E}\left(x_{t}\right)=\mathbb{E}\left(y_{t}\right)+\mathbb{E}\left(y_{t-2}\right)=\mu_{y}+\mu_{y}=2 \mu_{y} \quad \forall t
$$

(2)

$$
\begin{aligned}
\operatorname{Cov}\left(x_{t}, x_{t+k}\right)= & \operatorname{Cov}\left(y_{t}+y_{t-2}, y_{t+k}+y_{t+k-2}\right) \\
= & \operatorname{Cov}\left(y_{t}+y_{t-2}, y_{t+k}\right) +\operatorname{Cov}\left(y_{t}+y_{t-2}, y_{t+k-2}\right) \\
= & \operatorname{Cov}\left(y_{t}, y_{t+k}\right)+\operatorname{Cov}\left(y_{t-2}, y_{t+k}\right) +\operatorname{Cov}\left(y_{t}, y_{t+k-2}\right)+\operatorname{Cov}\left(y_{t-2}, y_{t+k-2}\right) \\
= & \lambda_{k}+\lambda_{k+2}+\lambda_{k-2}+\lambda_{k} \\
= & 2 \lambda_{k}+\lambda_{k+2}+\lambda_{k-2} \quad \forall t
\end{aligned}
$$

Where $\lambda_{k}=\operatorname{Cov}\left(y_{t}, y_{t+k}\right)$ does not depend on $t$ by stationarity of $y_{t}$. This concludes the proof.
}
{\item 
I do not believe that $x_{t}$ is strictly stationary, since it is made up of stationary series: $x_{t}=y_{t}+y_{t-2}$.
}
\end{enumerate}
}
{
\subsubsection*{Exercise 2}

\begin{enumerate}[label=(\alph*)]
{\item 
$$
\begin{aligned}
\mathbb{E}\left(\left(\hat{x}_{t}-x_{t}\right)^{2}\right) & =\mathbb{E}\left[\left(\frac{1}{2}\left(2 x_{t}+e_{1 t}+e_{2 t}\right)-x_{t}\right)^{2}\right] \\
& =\mathbb{E}\left[\left(\frac{1}{2}\left(e_{1t}+e_{2 t}\right)\right)^{2}\right] \\
& =\frac{1}{4}\left(\mathbb{E}\left(e_{1 t}^{2}\right)+\mathbb{E}\left(e_{2 t}^{2}\right)+2 \mathbb{E}\left(e_{1 t} e_{2 t}\right)\right) \\
& =\frac{1}{4}(1+4+0)=\frac{5}{4}
\end{aligned}
$$
}
{\item 
$$
\begin{aligned}
& \mathbb{E}\left(\left(\lambda_{1} y_{1 t}+\lambda_{2} y_{2 t}-x_{t}\right)^{2}\right) \\
= & \mathbb{E}\left(\left(\lambda_{1}\left(x_{t}+e_{1 t}\right)+\lambda_{2}\left(x_{t}+e_{2 t}\right)-x_{t}\right)^{2}\right) \\
= & \mathbb{E}\left(\left(\left(\lambda_{1}+\lambda_{2}-1\right) x_{t}+\lambda_{1} e_{1 t}+\lambda_{2} e_{2 t}\right)^{2}\right)
\end{aligned}
$$

Let $\lambda_{1}+\lambda_{2}=1$, then:

$$
\begin{aligned}
\text { MSE } & =\mathbb{E}\left(\left(\lambda_{1} e_{1}+\lambda_{2} e_{2 t}\right)^{2}\right) \\
& =\lambda_{1}^{2}+4\left(1-\lambda_{1}\right)^{2}
\end{aligned}
$$

FOC :

$$
\begin{aligned}
& 2 \lambda_{1}+8\left(1-\lambda_{1}\right)(-1)=0 \\
& \lambda_{1}=4 / 5 \longrightarrow \lambda_{2}=1 / 5
\end{aligned}
$$
}
\end{enumerate}
}
{
\subsubsection*{Exercise 3}

\begin{enumerate}[label=(\alph*)]
{\item 
$$
\begin{aligned}
\hat{\beta} & =\left(\frac{1}{T} \sum_{t=1}^{T} x_{t}^{2}\right)^{-1}\left(\frac{1}{T} \sum_{t=1}^{T} x_{t}\left(\beta x_{t}+e_{t}\right)\right) \\
& =\underbrace{\left(\frac{1}{T} \sum_{t=1}^{T} x_{t}^{2}\right)^{-1}}_{\xrightarrow{p} \mathbb{E}\left(x_{t}^{2}\right)^{-1}} \underbrace{\left(\frac{1}{T} \sum_{t=1}^{T} x_{t} e_{t}\right)}_{\xrightarrow{p} \mathbb{E}\left(x_{t} e_{t}\right)}+\beta
\end{aligned}
$$

$$
\begin{aligned}
\mathbb{E}\left(x_{t}^{2}\right)^{-1} & =8 / 3 \\
\mathbb{E}\left(x_{t} e_{t}\right) & =\mathbb{E}\left(0.5 x_{t-1}+\varepsilon_{t}+\eta_{t}\right)\left(0.8 e_{t-1}+\eta_{t}\right) \\
& =0.4 \mathbb{E}\left(x_{t-1} e_{t-1}\right)+\mathbb{E}\left(\eta_{t}^{2}\right) \neq 0
\end{aligned}
$$

Since $x_{t}$ and $e_t$ are correlated, $\hat{\beta}$ is inconsistent!
}
{\item 
$$
\begin{aligned}
\tilde{\beta} & =\left(\frac{1}{T} \sum_{t=1}^{T} z_{t} x_{t}\right)^{-1}\left(\frac{1}{T} \sum_{t=1}^{T} z_{t} y_{t}\right) \\
&=\left(\frac{1}{T} \sum_{t=1}^{T} z_{t} x_{t}\right)^{-1}\left(\frac{1}{T} \sum_{t=1}^{T} z_{t}\left(\beta x_{t}+e_{t}\right)\right) \\
&=\beta+\left(\frac{1}{T} \sum_{t=1}^{T} z_{t} x_{t}\right)^{-1}\left(\frac{1}{T} \sum_{t=1}^{T} z_{t} e_{t}\right) \\
\sqrt{T}(\tilde{\beta}-\beta) &= \left(\frac{1}{T} \sum_{t=1}^{T} z_{t} x_{t}\right)^{-1}\left(\frac{1}{\sqrt{T}} \sum_{t=1}^{T} z_{t} e_{t}\right)
\end{aligned}
$$

$$
\begin{aligned}
\left(\frac{1}{T} \sum_{t=1}^{T} z_{t} x_{t}\right)^{-1} &\xrightarrow{p} \mathbb{E}\left(z_{t} x_{t}\right)^{-1} =\mathbb{E}\left(\left(\varepsilon_{t}+v_{t}\right)\left(0.5 x_{t-1}+\varepsilon_{t}+\eta_{t}\right)\right)^{-1} =\mathbb{E}\left(\varepsilon_{t}^{2}\right)^{-1}=1 \\
\left(\frac{1}{\sqrt{T}} \sum_{t=1}^{T} z_{t} e_{t}\right) &\xrightarrow{d} N\left(0, \mathbb{E}\left(z_{t}^{2}\right) \mathbb{E}\left(e_{t}^{2}\right)\right) =N\left(0,2 \frac{1}{1-0.64}\right)
\end{aligned}
$$

In combination, this tells us (using Slutsky):

$$
\sqrt{T}(\tilde{\beta}-\beta) \xrightarrow{d} N\left(0, \frac{2}{(1-0.64)}\right)=N(0,5.6)
$$
}
\end{enumerate}
}
{
\subsubsection*{Exercise 4}

\begin{enumerate}[label=(\alph*)]
{\item 
$$
\begin{aligned}
& P\left(\left.\bar{y}^{2}>\frac{\ln (T)}{T} \right\rvert\, y_{t}=\mu+\varepsilon_{t}\right)=P\left(\left.\left(\frac{1}{T} \sum_{t=1}^{T} y_{t}\right)^{2}>\frac{\ln (T)}{T} \right\rvert\, y_{t}=\mu+\varepsilon_{t}\right) \\
&=P\left(\left(\frac{1}{T} \sum_{t=1}^{T} \mu+\varepsilon_{t}\right)^{2}>\frac{\ln (T)}{T}\right)=P\left(\left(\mu+\frac{1}{T} \sum_{t=1}^{T} \varepsilon_{t}\right)^{2}>\frac{\ln (T)}{T}\right) \\
&=P\left(\left(\frac{\ln (T)}{T}\right)^{1 / 2}<\mu+\frac{1}{T} \sum_{t=1}^{T} \varepsilon_{t}\right)+P\left(\mu+\frac{1}{T} \sum_{t=1}^{T} \varepsilon_{t}<-\left(\frac{\ln (T)}{T}\right)^{1 / 2}\right) \\
&=P\left(\frac{1}{\sqrt{T}} \sum_{t=1}^{T} \varepsilon_{t}<\sqrt{T} \mu-\ln (T)^{1 / 2}\right)+P\left(\frac{1}{T} \sum_{t=1}^{T} \varepsilon_{t}<-\ln (T)^{1 / 2}-\sqrt{T} \mu\right)
\end{aligned}
$$

Since $\frac{1}{\sqrt{T}} \sum \varepsilon_{t} \Rightarrow N(0,1)$ :

$$
=\Phi\left(\sqrt{T} \mu-\ln (T)^{T / 2}\right)+\Phi\left(-\ln (T)^{1 / 2}-\sqrt{T} \mu\right) \xrightarrow{T \rightarrow \infty} \Phi(\infty)+\Phi(-\infty)=1
$$
}
{\item 
$$
\begin{aligned}
P\left(\left.\bar{y}^{2}<\frac{\ln (T)}{T} \right\rvert\, y_{t}=\varepsilon_{t}\right)&=P\left(\left.\left(\frac{1}{T} \sum_{t=1}^{T} y_{t}\right)^{2}<\frac{\ln (T)}{T} \right\rvert\, y_{t}=\varepsilon_{t}\right) \\
=P\left(\left(\frac{1}{T} \sum_{t=1}^{1} \varepsilon_{t}\right)^{2}<\frac{\ln (T)}{T}\right)&=P\left(\left(\frac{1}{\sqrt{T}} \sum_{t=1}^{T} \varepsilon_{t}\right)^{2}<\ln (T)\right) \\
&\cong P\left(\chi_{1}^{2}<\ln (T)\right) \xrightarrow{T \rightarrow \infty} 1
\end{aligned}
$$
}
\end{enumerate}
}
}

\newpage
{
\subsection*{Honor\'e}

{
\subsubsection*{Exercise 1}

\begin{enumerate}[label=(\arabic*)]
{\item 
 MLE:

$$
\begin{aligned}
L&=\prod_{i=1}^{n} P\left(y=y_{i} \mid x_{i}\right) \\
&=\prod_{i=1}^{n} P\left(y=1 \mid x_{i}\right)^{y_{i}} \left(1-P\left(y=1 \mid x_{i}\right)\right)^{1-y_{i}} \\
&=\prod_{i=1}^{n}\left[\frac{\exp \left(x_{i}^{\prime} \beta\right)}{1+\exp \left(x_{i}^{\prime} \beta\right)}\right]^{y_{i}}\left[\frac{1}{1+\exp \left(x_{i}^{\prime} \beta\right)}\right]^{1-y_{i}} \\
l&=\sum_{i=1}^{n} y_{i} \ln \left(\frac{\exp \left(x_{i}^{\prime} \beta\right)}{1+\exp \left(x_{i}^{\prime} \beta\right)}\right)+\left(1-y_{i}\right) \ln \left(\frac{1}{1+\exp \left(x_{i}^{\prime} \beta\right)}\right) \\
&=\sum_{i=1}^{n} y_{i} x_{i}^{\prime} \beta-\ln \left(1+\exp \left(x_{i}^{\prime} \beta\right)\right) \\
\frac{\partial l}{\partial b} &=\sum_{i=1}^{n} y_{i} x_{i}^{\prime}-\frac{\exp \left(x_{i}^{\prime} b\right)}{1+\exp \left(x_{i}^{\prime} b\right)} x_{i}^{\prime} \stackrel{!}{=} 0 \\
\frac{\partial^{2} l}{\partial b^{2}} &=-\sum_{i=1}^{n}\left[\frac{\exp \left(x_{i}^{\prime} b\right) x_{i}\left(1+\exp \left(x_{i}^{\prime} b\right)\right)-\exp \left(x_{i}^{\prime} b\right) x_{i} \exp \left(x_{i}^{\prime} b\right)}{\left(1+\exp \left(x_{i}^{\prime} b\right)\right)^{2}} x_{i}^{\prime}\right] \\
&=-\sum_{i=1}^{n} \frac{\exp \left(x_{i}^{\prime} b\right) x_{i}}{1+\exp \left(x_{i}^{\prime} b\right)}\left(1-\frac{\exp \left(x_{i}^{\prime} b\right)}{1+\exp \left(x_{i}^{\prime} b\right)}\right) x_{i}^{\prime} \\
&=-\sum_{i=1}^{n} \frac{\exp \left(x_{i}^{\prime} b\right)}{1+\exp \left(x_{i}^{\prime} b\right)} \frac{1}{1+\exp \left(x_{i}^{\prime} b\right)} x_{i} x_{i}^{\prime} \\
-\mathbb{E}\left(\frac{\partial^{2} e}{\partial b^{2}}\right)^{-1}&=n \cdot \mathbb{E}\left[\frac{\exp \left(x_{i}^{\prime} b\right)}{1+\exp \left(x_{i}^{\prime} b\right)} \frac{1}{1+\exp \left(x_{i}^{\prime} b\right)} x_{i} x_{i}^{\prime}\right]^{-1}
\end{aligned}
$$

Thus:

$$
\sqrt{\sqrt{n}(b-\beta) \xrightarrow{d} N\left(0, \mathbb{E}\left[\frac{\exp \left(x_{i}^{\prime} b\right)}{1+\exp \left(x_{\prime}^{i} b\right)} \frac{1}{1+\exp \left(x_{i}^{\prime} b\right)} x_{i} x_{i}^{\prime}\right]^{-1}\right)}
$$
}
{\item 
$$
\begin{aligned}
& \sqrt{n}(b-\beta) \xrightarrow{d} N\left(O, A^{-1} B A^{-1}\right) \\
\frac{\partial f\left(x_{i}, \beta\right)}{\partial \beta} &= \frac{\exp \left(x_{i}^{\prime} \beta\right)}{1+\exp \left(x_{i}^{\prime} \beta\right)} \frac{1}{1+\exp \left(x_{i}^{\prime} \beta\right)} x_{i} \\
A &= \mathbb{E}\left[\left(\frac{\partial f\left(x_{i}, \beta\right)}{\partial \beta}\right)\left(\frac{\partial f\left(x_{i}, \beta\right)}{\partial \beta}\right)^{\prime}\right]=\mathbb{E}\left[\frac{\exp \left(2 x_{i}^{\prime} \beta\right)}{\left(1+\exp \left(x_{i}^{\prime} \beta\right)\right)^{4}} x_{i} x_{i}^{\prime}\right] \\
B &= \mathbb{E}\left[\mathbb{E}\left(\varepsilon_{i}^{2} \mid x_{i}\right)\left(\frac{\partial f\left(x_{i}, \beta\right)}{\partial \beta}\right)\left(\frac{\partial f\left(x_{i}, \beta\right)}{\partial \beta}\right)^{\prime}\right] \\
& =\mathbb{E}\left[\frac{\exp \left(x_{i}^{\prime} \beta\right)}{\left(1+\exp \left(x_{i}^{\prime} \beta\right)\right)^{2}} \frac{\exp \left(2 x_{i}^{\prime} \beta\right)}{\left(1+\exp \left(x_{i}^{\prime} \beta\right)\right)^{4}} x_{i}^{\prime} x_{i}^{\prime}\right]
\end{aligned}
$$
}
{\item 
$\sqrt{n}(b-\beta) \xrightarrow{d} N(0, V)$

Let it be efficient $G M M$ : $V=\left(G^{\prime} S^{-1} G\right)^{-1}$

$$
\begin{aligned}
& G=\mathbb{E}\left(\frac{\partial f\left(x_{i}, \beta\right)}{\partial \beta}\right)=\mathbb{E}\left[\frac{\exp \left(x_{i}^{\prime} \beta\right)}{\left(1+\exp \left(x_{i}^{\prime} \beta\right)\right)^{2}} x_{i} x_{i}^{\prime}\right] \\
& S=\mathbb{E}\left[\mathbb{E}\left(\varepsilon_{i}^{2} \mid x_{i}\right) x_{i} x_{i}^{\prime}\right]=\mathbb{E}\left[\frac{\exp \left(x_{i}^{\prime} \beta\right)}{\left(1+\exp \left(x_{i}^{\prime} \beta\right)\right)^{2}} x_{i} x_{i}^{\prime}\right] \\
& \sqrt{n}(b-\beta) \xrightarrow{d} N\left(0, \mathbb{E}\left[\frac{\exp \left(x_{i}^{\prime} \beta\right)}{\left(1+\exp \left(x_{i}^{\prime} \beta\right)\right)^{2}} x_{i} x_{i}^{\prime}\right]^{-1}\right)
\end{aligned}
$$

(same as in (1))
}
{\item 
$\hat{\beta} \xrightarrow{p} \underset{b}{\operatorname{argmax}} \mathbb{E}\left(\ln \left(f\left(x_{i}^\prime b\right)\right)\right) \equiv \tilde{\beta} \neq \beta$

$$
\begin{aligned}
& \sqrt{n}(\hat{\beta}-\tilde{\beta}) \xrightarrow{d} N\left(0, \mathbb{E}\left(\frac{\partial^{2} \ln \left(f\left(x_{i}^\prime \beta\right)\right)}{\partial \beta \partial \beta^{\prime}}\right)^{-1} V\left(\tilde{\varepsilon}_{i} \mid x_{i}\right) \mathbb{E}\left(\frac{\partial^{2} \ln \left(f\left(x_{i}^\prime \beta\right)\right)}{\partial \beta \partial \beta^{\prime}}\right)^{-1}\right)
\end{aligned}
$$

It will be inconsistent but it will choose the "best" estimator in the class of $f(\cdot)$.
}
\end{enumerate}
}
{
\subsubsection*{Exercise 2}

\begin{enumerate}[label=(\arabic*)]
{\item 
This will lead to issues since we are actually regressing $y$ on its lagged values. This means we have an endogenous error term \& the estimator is not consistent.
}
{\item 
We should take first differences:

\begin{center}
\begin{tabular}{l|l}
$t$ & regression equation \\
\hline 5 & $\Delta y_{i 5}=\Delta x_{i 5}^{\prime} \beta_{1}+\Delta y_{i 4} \beta_{2}+\Delta \varepsilon_{i 5}$ \\
4 & $\Delta y_{i 4}=\Delta x_{i 4}^{\prime} \beta_{1}+\Delta y_{i 3} \beta_{2}+\Delta \varepsilon_{i 4}$ \\
3 & $\Delta y_{i 3}=\Delta x_{i 3}^{\prime} \beta_{1}+\Delta y_{i 2} \beta_{2}+\Delta \varepsilon_{i 3}$
\end{tabular}
\end{center}

We can then use the following instruments:

\begin{center}
\begin{tabular}{l|r r}
$t$ & instruments & \\\hline
5 & $y_{i 3}, y_{i 2}, y_{i 1}$ & $\left\{x_{i s}\right\}_{s=1}^5$ \\
4 & $y_{i 2}, y_{i 1}$ & $\left\{x_{i s}\right\}_{s=1}^5$ \\
3 & $y_{i 1}$ & $\left\{x_{i s}\right\}_{s=1}^5$
\end{tabular}
\end{center}

Note: cannot use forward looking instrument due to exogeneity constraint.

We must assume that the instruments are valid. The model is over-identified if there are more instruments than regressors.
}
\end{enumerate}
}
{
\subsubsection*{Exercise 3}

\begin{enumerate}[label=(\arabic*)]
{\item 
Conditional on age, the assignment is independent of the outcomes that a person would have.
}
{\item 
$$
\begin{aligned}
A T E & =\mathbb{E}\left(Y_{1}-Y_{0}\right)=\mathbb{E}\left(\mathbb{E}\left(Y_{1} \mid X_{1}, D=1\right)-\mathbb{E}\left(Y_{0} \mid X_{1}, D=0\right)\right) \\
& =\frac{3}{17}\left(\frac{100+80}{2}-80\right) \\
& +\frac{4}{17}\left(\frac{55+50}{2}-\frac{55+65}{2}\right) \\
& +\frac{3}{17}\left(40-\frac{50+30}{2}\right) \\
& +\frac{4}{17}\left(\frac{40+35}{2}-\frac{45+20}{2}\right) \\
& +\frac{3}{17}\left(\frac{20+25}{2}-25\right) \\
& \cong 0.735
\end{aligned}
$$
}
\end{enumerate}
}
}
\newpage
\section{Microeconomics Midterm 21 / 22}

{
\subsection*{Schmidt}

{
\subsubsection*{Exercise 1}

\begin{enumerate}[label=(\alph*)]
{\item 
Clearly, WA is violated as $15 \in[0.22 .5]$ by the result in (b).
}
{\item 
WA: if $x \neq x^{\prime}$ and $p^{\prime} x \leqslant w^{\prime} \Rightarrow p x^{\prime}>w$

Thus check bundles in other price-wealth situations:

$$
\begin{aligned}
\left|\begin{array}{l}
4 \cdot 30+8 \cdot y=120+8 y \leqslant w_{0}=4 \cdot 15+8 \cdot 30=300 \\
12 \cdot 15+6 \cdot 30=360 \leqslant w_{1}=12 \cdot 30+6 y=360+6 y \\
\end{array}\right| \\
\Leftrightarrow\left|\begin{array}{l}
y \leqslant 22.5 \\
0 \leqslant y
\end{array}\right|
\Longleftrightarrow y \in[0.22 .5]
\end{aligned}
$$

$W A$ is violated if $y \in[0,22.5]$.
}
\end{enumerate}
}
{
\subsubsection*{Exercise 2}

\begin{enumerate}[label=(\alph*)]
{\item 
Use Roy's identity:

$$
x_{l}(p, w)=-\frac{\frac{\partial v(p, w)}{\partial p_{l}}}{\frac{\partial v(p, w)}{\partial w}}
$$

Then let $f(\cdot)$ be a movotonic tranformation:

$$
\tilde{x}_{l}(p, w)=-\frac{\frac{\partial f(v(p, w))}{\partial p_{l}}}{\frac{\partial f(p, w))}{\partial w}}=-\frac{\frac{\partial f(p, w)}{\partial v(p, w)}}{\frac{\partial f(p, w)}{\partial v(p, w)}} \frac{\frac{\partial v(p, w)}{\partial p_{l}}}{\frac{\partial v(p, w)}{\partial w}}=x_{l}(p, w)
$$
}
{\item 
(1) find $w(v(p, w))$ :

$$
w=\left(\frac{p_{1}}{\alpha}\right)^{\alpha}\left(\frac{p_{2}}{1-\alpha}\right)^{1-\alpha} v(p, w)
$$

At optimum: $w=e(p, u)$ and $v(p, w)=u$.

(2) Apply Shephard's Lemma to e(p,u):

$$
\begin{aligned}
h_{1}(p, u)=\frac{\partial e\left(p_{1} u\right)}{\partial p_{1}} & =\alpha\left(\frac{1}{\alpha}\right)^{\alpha}\left(\frac{p_{2}}{p_{1}(1-\alpha)}\right)^{1-\alpha} u \\
& =\left(\frac{p_{2}}{p_{1}} \frac{\alpha}{1-\alpha}\right)^{1-\alpha} u
\end{aligned}
$$
}
{\item 
case 1:

$$
\begin{aligned}
& \alpha=\alpha\left(p_1 / p_{2}\right) \longrightarrow \alpha\left(\lambda p_{1} / \lambda p_{2}\right)=\alpha \\
& h_{1}\left(\lambda p_{1}, u\right)=\left(\frac{\lambda p_{2}}{\lambda p_{1}} \frac{\alpha}{1-\alpha}\right)^{1-\alpha} u \\
& =\left(\frac{p_{2}}{p_{1}} \frac{\alpha}{1-\alpha}\right)^{1-\alpha} u=h_{1} (p_{1}, u)
\end{aligned}
$$

case 2: $\alpha=\alpha\left(p_{1}\right)$

$$
\begin{aligned}
h_{1}(\lambda p_1, u) & =\left(\frac{\lambda p_{2}}{\lambda p_{1}} \frac{\alpha\left(\lambda {p_{1}}\right)}{1-\alpha \left(\lambda p_{1}\right)}\right)^{1-\alpha\left(\lambda_{p_{1}}\right)} u \\
& =\left(\frac{p_{2}}{p_{1}} \frac{\alpha\left(\lambda p_{1}\right)}{1-\alpha\left(\lambda {p_{1}}\right)}\right)^{1-\alpha\left(\lambda_{p_{1}}\right)} u \neq h_{1}\left(p_{1} , u\right)
\end{aligned}
$$
}
\end{enumerate}
}
{
\subsubsection*{Exercise 3}

The difference between consumer theory and production theory is mainly the fact that firms do not have budget constraints.
This problem introduces a budget constraint. Therefore, we are going to treat the problem like a consumer problem.
In that sense, the revenue is comparable to the utility function, and the cash constraint is like the wealth of a consumer.
Consequently, we are solving the following revenue maximization problem (which is the analogue to a utility maximization problem):

\begin{align*}
    \max_{z_1,z_2} pf(z_1,z_2) \\
    \operatorname{s.t.} \; w_1z_1 + w_2z_2 \leq C
\end{align*}

We will assume an interior solution (the budget constraint is binding).
Then, the revenue function $R(p, w_1, w_2, C)$ that the exercise gives us is just the equivalent to the indirect utility.

\begin{enumerate}[label=(\alph*)]
{\item 
As $R(p, w_1, w_2, C)$ works like the indirect utility, we apply Roy's identity to find the factor demand, which is the analogue to the Walrasian demand:

\begin{align*}
    z_1&=-\frac{\frac{\partial R}{\partial w_1}}{\frac{\partial R}{\partial C}} \\
    &= -\frac{p \cdot(-\alpha) \frac{1}{w_1}}{p \cdot \frac{1}{C}} \\
    &= \alpha \frac{C}{w_1}
\end{align*}
}
{\item 
We treat $R(p,w,C)$ as the indirect utility depending on income and invert it to find the cost function $C(p,w,R)$, which is the analogue to the expenditure function in consumer theory:

\begin{align*}
    R&=p\left[\gamma+\ln C(p,w,R)-\alpha \ln w_1-(1-\alpha) \ln w_2\right] \\
    \frac{R}{p}-\gamma &= \ln \left(\frac{C(p,w,R)}{w_1^\alpha w_2^{1-\alpha}} \right) \\
    \exp\left(\frac{R}{p}-\gamma\right)&=\frac{C(p,w,R)}{w_1^\alpha w_2^{1-\alpha}} \\
    C(p,w,R) &= w_1^\alpha w_2^{1-\alpha}\exp\left(\frac{R}{p}-\gamma\right)
\end{align*}
}
{\item 
Since the cost function from (b) happens to be the analogue to the expenditure function, we can apply Shephard's Lemma in order to find the factor demand for a given $R$ at minimum cost, as this is the analogue to the Hicksian demand in consumer theory.
In that spirit, let us call this function $h_1(p,w,R)$.

\begin{align*}
    h_1(p,w,R)&=\frac{\partial C\left(w,R\right)}{\partial w_1} \\
    &= \alpha \exp \left[\frac{R}{p}-\gamma\right] \cdot\left(\frac{w_2}{w_1}\right)^{1-\alpha}
\end{align*}
}
{\item 
In consumer theory, the Hicksian demand and the Walrasian demand meet at optimum. We can also show that here:

\begin{align*}
    h_1(w,R)&=z_1^* \\
    \alpha \exp \left[\frac{R}{p}-\gamma\right] \cdot\left(\frac{w_2}{w_1}\right)^{1-\alpha}&=\alpha \frac{C}{w_1}\\
    \exp \left[\frac{R}{p}-\gamma\right] w_1^\alpha w_2^{1-\alpha}&=C \\
    \frac{R}{p}-\gamma &= \ln \left(\frac{C}{w_1^\alpha w_2^{1-\alpha}} \right) \\
    R&=p\left[\gamma+\ln C-\alpha \ln w_1-(1-\alpha) \ln w_2\right]
\end{align*}

The last line is exactly the formula for the revenue that is observed by our econometrician friend in the optimum. Therefore, we have shown that the two demands are equal whenever the firm is acting optimally, i.e. maximizing its revenue or minimizing its cost. Put differently, the revenue maximization problem is the dual problem to the cost minimization problem and vice versa.
}
\end{enumerate}
}
{
\subsubsection*{Exercise 4}

\begin{enumerate}[label=(\alph*)]
{\item 
\underline{IF:}

$u(x)=\beta x^{1-\rho}+\gamma$

$$
r^{R}=-x \frac{u^{\prime \prime}(x)}{u^{\prime}(x)}=-x \frac{\beta(1-  \rho)(-\rho) x^{-\rho-1}}{\beta(1-\rho) x^{-\rho}}=\rho
$$

\underline{ONLY IF:}

$r^{R}=-x \frac{u^{\prime \prime}(x)}{u^{\prime}(x)}=-x \frac{\partial \ln \left(u^{\prime}(x)\right)}{\partial x}=\rho$

$$
\begin{aligned}
& \Longleftrightarrow \quad \frac{\partial \ln \left(u^{\prime}(x)\right)}{\partial x}=-\rho \frac{1}{x} \\
& \Longleftrightarrow \int_{\underline{x}}^{x} \frac{\partial \ln \left(u^{\prime}(t)\right)}{\partial t} d t=-\rho \int_{\underline{x}}^{x} \frac{1}{t} d t \\
& \Longleftrightarrow \quad \ln \left(u^{\prime}(x)\right)-\ln \left(u^{\prime}(\underline{x})\right)=-\rho (\ln (x)-\ln (\underline{x})) \\
& \Leftrightarrow \quad u^{\prime}(x)=x^{-p} \frac{u^{\prime}(\underline{x})}{\underline{x}^{-\rho}}=x^{-\rho} \alpha \\
& \int_{\underline{x}}^{x} u^{\prime}(y) d y=\alpha \int_{\underline{x}}^{x} y^{-\rho} d y \\
& \Leftrightarrow \quad u(x)-u(\underline{x})=\frac{\alpha}{1-\rho}\left(x^{1-p}-\underline{x}^{1-p}\right) \\
& \Leftrightarrow \quad u(x)=\beta x^{1-p} + \gamma
\end{aligned}
$$

Risk aversion:

$$
\begin{aligned}
& u^{\prime \prime}(x)<0 \\
\Leftrightarrow &\beta(1-\rho)(-\rho) x^{-\rho-1}<0 \\
\Rightarrow &\beta(1-\rho) {\rho}>0
\end{aligned}
$$

This only holds when $(\beta>0$ and $\rho<1)$ or $(\beta<0$ and $\rho>1$ ).
}
\end{enumerate}
}
}

\newpage
{
\subsection*{Gottardi}

{
\subsubsection*{Exercise 1}

\begin{enumerate}[label=(\roman*)]
{\item 
Pareto Efficient:

\begin{align*}
    M R S^{A}=3 \frac{x_{2}^{A}}{x_{1}^{A}}&=M R S^{B}=\frac{1}{2} \\
    \Longleftrightarrow \quad x_{2}^{A}&=\frac{1}{6} x_{1}^{A}
\end{align*}

\begin{figure}[!htp]
    \centering
    \includegraphics[width=.75\textwidth]{images/2021_22_1.png}
\end{figure}
}
{
\item 
\underline{consumer A:}

$$
\begin{aligned}
& \max _{x_{1}^{A}, x_{2}^{A}} 3 \ln \left(x_{1}^{A}\right)+\ln \left(x_{2}^{A}\right) \\
& \text { s.t. } x_{1}^{A}+p x_{2}^{A}=8+2 p 
\end{aligned}
$$

FOCs:

$$
\begin{aligned}
& \left[x_{1}^{A}\right]: \frac{3}{x_{1}^{A}}-\lambda=0 \\
& \left[x_{2}^{A}\right]: \frac{1}{x_{2}^{A}}-\lambda p=0 \\
& \rightarrow x_{1}^{A}=3 p x_{2}^{A}
\end{aligned}
$$

\underline{consumer B:} linear utility leads to:

$$
\begin{aligned}
& x_{1}^{B}=\left\{\begin{array}{lll}
\infty & \text { if } & p \geq 2 \\
\mathbb{R}^{+} & \text { if } & p=2 \\
0 & \text { if } & p<2
\end{array}\right. \\
& x_{2}^{B}=\left\{\begin{array}{lll}
\infty & \text { if } p \leq 2 \\
\mathbb{R}^{+} & \text { if } p=2 \\
0 & \text { if } p>2
\end{array}\right.
\end{aligned}
$$

\underline{market:} 

$$
x_{1}^{A}+x_{1}^{B}=10=x_{2}^{A}+x_{2}^{B}
$$

In order for markets to clear with no excess demand, we must have $p=2$ because of consumer B's preferences.
Therefore

$$
x_{1}^{A}=6 x_{2}^{A}
$$

plug into $B C^{A}: \quad 8 x_{2}^{A}=8+4 \quad \Longleftrightarrow x_{2}^{A}=12 / 8=3 / 2$

$$
\rightarrow x_{1}^{A}=9 \rightarrow\left(x_{1}^{B}, x_{2}^{B}\right)=(1,17 / 2)
$$

\underline{Competitive Equilibrium:}

\begin{align*}
    \left(x_{1}^{A}, x_{2}^{A}\right) &= (9,3 / 2) \\
    \left(x_{1}^{B}, x_{2}^{B}\right) &= (1,17 / 2) \\
    \frac{p_2}{p_1} &= 2
\end{align*}

Since $x_{2}^{A}=1 / 6 x_{1}^{A}$, PE is achieved.
}
{
\item 

$$
\begin{aligned}
& u^{A}\left(w_{1}^{A}, w_{2}^{A}\right)=3 \ln (8)+\ln (2) \cong 6.931 \\
& u^{A}\left(x_{1}^{A}, x_{2}^{A}\right)=3 \ln (9)+\ln (3 / 2) \cong 6.997 \\
& u^{B}\left(w_{1}^{B}, w_{2}^{B}\right)=2+2 \cdot 8=18 \\
& u^{B}\left(x_{1}^{D}, x_{2}^{B}\right)=1+2 \cdot 17 / 2=18
\end{aligned}
$$

By FWT this is always true, when preferences do not violate LNS, there is free disposal and markets are complete.
}
\end{enumerate}
}
{
\subsubsection*{Exercise 2}

Autarky: A sells, B buys good 1.

Effect depends on price change \& preferences.

Assume $\frac{P_1}{P_2}$ goes up (the other way round the argument can be reversed).
This makes the seller better off as she gets more per unit sold and might even sell more. For $B$ it depends on her preferences. If she can substitute and switch to selling good 1, she profits. 
If she has to buy good 1 at a higher price, she loses. It is also possible that her utility does not change despite the price change. 

If prices remain the same, nothing changes.
}
{
\subsubsection*{Exercise 3}

$$
\left(w_{1}, w_{2}\right)=(1,4)
$$

\begin{enumerate}[label=(\roman*)]
{\item 
at $t=0: \quad q_{1} \theta_{1}+q_{2} \theta_{2}=0$

at $t=1, s=1: \quad x_{1}=w_{1}+\theta_{1} 4+\theta_{2}$

at $t=1, s=2: \quad x_{2}=w_{2}+\theta_{2}$
}
{
\item 

\underline{Consumer problem:}

$$
\begin{gathered}
\max _{\theta_{1}, \theta_{2}} 1 / 4\left(w_{1}+4 \theta_{1}+\theta_{2}\right)^{1 / 2}+3 / 4\left(w_{2}+\theta_{2}\right)^{1 / 2} \\
\text { s.t. } \quad q_{1} \theta_{1}+q_{2} \theta_{2}=0
\end{gathered}
$$

FOCs:

\begin{align*}
    &\left[\theta_{1} \right]: \frac{4}{8\left(w_{1}+4 \theta_{1}+\theta_{2}\right)^{1 / 2}}-\lambda q_{1}=0 \\
    &\left[\theta_{1} \right]: \frac{1}{8\left(w_{1}+4 \theta_{1}+\theta_{2}\right)^{1 / 2}}+\frac{3}{8\left(w_{2}+\theta_{2}\right)^{1 / 2}}-\lambda q_{2}=0
\end{align*}

\underline{market clearing:}

\begin{align*}
    \theta_{1}=-\theta_{2}=0 \tag{I}
\end{align*}

Plug (I) into FOCs:

$$
\begin{aligned}
& \frac{1}{2}=\lambda q_{1} ; \quad \frac{1}{8}+\frac{3}{16}=\lambda q_{2} \\
\longrightarrow & \frac{q_{1}}{q_{2}}=\frac{1}{2} \frac{16}{5}=\frac{8}{5}
\end{aligned}
$$
}
{
\item 

$\mathbb{E}\left(r_{1}\right)=\mathbb{E}\left(r_{2}\right)=1$

$$
\longrightarrow \frac{\mathbb{E}\left(r_{1}\right)}{q_{1}}>\frac{\mathbb{E}\left(r_{2}\right)}{q_{2}} \Longleftrightarrow 1>\frac{q_{1}}{q_{2}}=\frac{8}{5}
$$

We see that the inequality above is INCORRECT, we have run into a CONTRADICTION.

Usual intuition: $q_{1} / q_{2}>1$ because consumer wants to insure against poor state where she has less income. This leads to a lower expected rate of return for asset 1. Otherwise the consumer would buy asset 1 but she cannot because of market clearing.
}
\end{enumerate}
}
}\newpage
\input{Midterms_Years/2022_23}\newpage
\input{Midterms_Years/2023_24}\newpage

\end{document}
