\section{Macroeconomics Final 2018 / 19}

{
\subsection*{Exercise 1}

\begin{enumerate}[label=(\alph*)]
{\item 
False

We showed in class that in this case, the optimal interest rate is zero. Together with the result of the representative consumer behaviour:

$$
\pi=i_{t}-\rho=-\rho
$$

I.e. constant deflation at rate $\rho$.
}
{
\item 
False

We analyzed the following welfare loss:

$$
W=-\mathbb{E}_{0} \sum_{t=0}^{\infty} \beta^{t} \frac{U_{t}-U}{U_{c} C}
$$

And looked at a 2nd order Taylor approximation:

$$
W \approx \frac{1}{2} \mathbb{E}_{0} \sum_{t=0}^{\infty} \beta^{t}\left(-\Phi \hat{x}_{t}+\left(G+\frac{\varphi+\alpha}{1-\alpha}\right) \hat{x}_{t}^{2}+\frac{\varepsilon}{\lambda} \pi_{t}^{2}\right)+t . i . \rho \text {. }
$$

Note, that $\pi_{t}$ does enter the welfare loss such that $\pi_{t}=0$ would lead to minimal welfare losses.
}
{
\item 
True

We have seen that with a distortional labour income tax, the divine coincidence breaks down. This means, inflation targeting is insufficient to stabilize outcome.

More specifically in exercise 2, PS3, we found the following NKPC:

$$
\pi_{t}=\beta \mathbb{E}_{t}\left(\pi_{t+1}\right)+\lambda(1+\varphi) x_{t}+\lambda\left(\mu+\tau_{t}\right)
$$
}
{
\item 
True.

It is only attainable if the natural real wage does not change, i.e. is constant.
}
\end{enumerate}
}
{
\subsection*{Exercise 2}

\begin{enumerate}[label=(\alph*)]
{\item 
$$
\max _{N_{t}}\left(P_{t}-1\right) N_{t}
$$

This implies

$$
N_{t}=\left\{\begin{array}{lll}
\infty & \text { if } & P_{t} > 1 \\
\mathbb{R}^{+} & \text { if } & P_{t}=1 \\
0 & \text { if } & P_{t}<1
\end{array}\right.
$$

Therefore, we may conclude that $P_{t}=1$, and the firm employs everyone that wants to work.
}
{
\item 
First, obtain labour supply by solving

$$
\begin{aligned}
& \max \mathbb{E}_{0}\left[\sum_{t=0}^{\infty} \beta^{t}\left(C_{t}-\frac{N_{t}^{1+\varphi}}{1+\varphi}\right)\right] \\
& \text { s.t. } 0=B_{t-1}+M_{t-1}+N_{t}-P_{t} C_{t}-Q_{t} B_{t}-M_{t}
\end{aligned}
$$

Classic result from FOCs:

$$
\begin{aligned}
& N_{t}^{\varphi}=\frac{W_{t}}{P_{t}}=1 \\
\Rightarrow & N_{t}=1
\end{aligned}
$$

Thus, labour supply is 1 , and this is the maximum that can be produced. Use CIA \& market clearing:

$$
G_{t}=Y_{t}=\frac{M_{t}}{Z_{t}}
$$

We conclude that:

$$
Y_{t}=\min \left\{\frac{M_{t}}{Z_t}, 1\right\}
$$
}
{
\item 
The positive shock impacts the CIA constraint. Increase in money demanded per unit of consumption. If supply does not increase (e.g. because $N_{t}=1$ is hit), need to reduce consumption, and lower output.
}
{
\item 
Social planner problem:

$$
\begin{aligned}
\max & \mathbb{E}_{0}\left[ \sum_{t=0}^{\infty} \beta^{t}\left(C_{t}-\frac{N_{t}^{1+\varphi}}{1+\varphi}\right)\right] \\
\text { s.t. } & C_{t}=Y_{t}=N_{t}
\end{aligned}
$$

Plug constraint into objective function:

$$
\max _{N_{t}} \mathbb{E}_{0}\left[\sum_{t=0}^{\infty} \beta^{t}\left(N_{t}-\frac{N_{t}^{1+\varphi}}{1+\varphi}\right)\right]
$$

FOC:

$$
1-N_{t}^{\varphi}=0 \Leftrightarrow N_{t}=1
$$

By production technology:

$$
Y_{t}^{e}=1
$$

Use $Y_{e}=1, P_{t}=1$ in CIA constraint:

$$
M_{t}=Z_{t}
$$

I.e. the money supply reacts to $Z_t$.
}
{
\item 
No. Both short- and long-run $Y_{t}=\min \left\{\frac{M_{t}}{Z_{t}}, 1\right\}$.

Here, prices and wages are fixed, and there is a persistent effect of money on output.

In NK model, both prices \& wages can be adjusted.
}
\end{enumerate}
}
