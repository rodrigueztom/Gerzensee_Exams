\section{Macroeconomics Final 2020 / 21}

{
\subsection*{Exercise 1}

\begin{enumerate}[label=(\alph*)]
{\item 
False

Expansionary shock lowers the interest rate, and increases the output. But as the labour supply is defined as

$$
w_{t}-p_{t}=\sigma c_{t}+n_{t} \text{ and } c_{t}=y_{t}
$$

Higher $y_t$ leads to higher real wages.
}
{
\item 
True

Inflation targeting is optimal when wages are flexible as this stabilized output at the same time by the "divine coincidence". Under sticky wages, this breaks down \& the central bank should also aim to stabilize output (\$ unemployment).
}
{
\item 
 False \color{red}?\color{black}

In the classical model, prices react to money growth more than 1:1. 
While prices react much less to money growth, as they are sticky, they still react, albeit to a lesser degree.
}
{
\item 
False

Because some firms are not able to adjust their prices, the inflation from last period is somewhat predictive for current inflation.
}
\end{enumerate}
}
{
\subsection*{Exercise 2}

Market clearing: 

\begin{align*}
    Y_{t} &= C_{t}+\gamma_{t} Y_{t} \\
    \Leftrightarrow Y_{t} &= \frac{1}{1-\gamma_{t}} C_{t}=F_{t} C_{t} \\
    y_{t} &= f_{t} + c_{t}
\end{align*}

HH optimality:

\begin{align*}
    N_{t} C_{t} &= \frac{W_{t}}{P_{t}} \\
    \Longrightarrow n_{t}+c_{t} &= w_{t}-p_{t} \\
    Q_{t} &= \beta \mathbb{E}_{t}\left\{\frac{C_{t}}{C_{t+1}} \frac{P_{t}}{P_{t+1}}\right\} \\
    \Rightarrow \dot{i}_{t} &= \beta+\mathbb{E}_{t}\left(c_{t+1}\right)-c_{t}+\mathbb{E}\left(\pi_{t+1}\right)
\end{align*}

Firms: 

\begin{align*}
    P_{t} &= \psi_{t} M_{t}=W_{t} M_{t} \\
    \Rightarrow p_{t} &= \mu_{t}+w_{t}
\end{align*}

\begin{enumerate}[label=(\alph*)]
{\item 
$\mu$ is constant:

$$
\begin{aligned}
\mu & =p_{t}-w_{t}=-\left(n_{t}+c_{t}\right) \\
& =-\left(y_{t}+y_{t}-f_{t}\right)=f_{t}-2 y_{t} \\
\longrightarrow y_{t}^{n} & =\frac{1}{2}\left(f_{t}-\mu\right)
\end{aligned}
$$

Since $y^{n}_t$ rises in $f_{t}$, we conclude that $\gamma_{t}$ increases output. Note, that the increase is not $1: 1$ as $g_t$ crowds out some of the consumption.
}
{
\item 
By market clearing:

$$
Y_{t}=G_{t}+C_{t}=F_{t} C_{t} \quad \text { (since } G_{t}=\gamma_{t} Y_{t} \text { ) }
$$

In logs:

\begin{equation*}
y_{t}=f_{t}+c_t \tag{1}
\end{equation*}

Need to find DIS. Stat with EE, plug in $c_t$ from (1):

$$
\begin{aligned}
& \underbrace{i_{t}}_{=\rho+\phi_{y}\hat{y}_t}=\rho+\mathbb{E}_{t}\left(c_{t+1}\right)-c_{t}+\underbrace{\mathbb{E}\left(\pi_{t+1}\right)}_{=0 \text{ by const. prices}} \\
& y_t-f_{t}=\mathbb{E}_{t}\left(y_{t+1}-f_{t+1}\right)-\phi_{y} \hat{y}_t
\end{aligned}
$$

Express in deviations from SS. By $\gamma_t \sim iid(\gamma)$, the expected deviation from SS is zero.

$$
\begin{aligned}
& \hat{y}_{t}=\mathbb{E}_{t}\left(\hat{y}_{t+1}\right)+\hat{f}_{t}-\phi_{y} \hat{y}_{t} \\
& \hat{y}_{t}=\frac{1}{1+\phi_{y}}\left(\mathbb{E}_{t}\left(\hat{y}_{t+1}\right)+\hat{f}_{t}\right)
\end{aligned}
$$

Now, iterate forward, using $\mathbb{E}_{t}\left(\hat{f}_{t+1}\right)=0$ :

$$
\begin{aligned}
    \mathbb{E}_{t} \left(\hat{y}_{t+1}\right) &= \frac{1}{1+\phi_{y}} \mathbb{E}_{t}\left(\hat{y}_{t+2}\right) \\
    &\vdots \\
    \mathbb{E}_{t}\left(\hat{y}_{t+1}\right) &= \left(\frac{1}{1+\phi_{y}}\right)^{j} \mathbb{E}_{t}\left(\hat{y}_{t+j+1}\right) \xrightarrow{j \rightarrow \infty} 0
\end{aligned}
$$

Thus, we are left with

$$
\hat{y}_{t}=\frac{1}{1+\phi_{y}} \hat{f}_{t}
$$

Channel: Increase in aggregate demand for goods which is met by increased production.
}
{
\item 
Generally:

$$
\frac{d Y_{t}}{d G_{t}}=\frac{d Y_{t} / d \gamma_{t}}{d G_{t} / d \gamma_{t}}=\frac{d Y_{t} / d \gamma_{t}}{d\left(\gamma_{t} Y_{t}\right) / d \gamma_{t}}=\frac{d Y_{t} / d \gamma_{t}}{Y_{t}+\gamma_{t} \left( d Y_{t} / d \gamma_{t}\right)}
$$

In (a):

$$
\begin{aligned}
& y_{t}^{n}=\frac{1}{2}\left(f_{t}-\mu\right) \Rightarrow y_{t}^{n}=\left[F_{t} M\right]^{1 / 2}=\left(1-\gamma_{t}\right)^{-1 / 2} M^{1 / 2} \\
& d Y_{t} / d \gamma_{t}=-\frac{1}{2}\left(1-\gamma_{t}\right)^{-3 / 2}(-1) M^{1 / 2}=\frac{Y_{t}^{n}}{2\left(1-\gamma_{t}\right)} \\
& \Rightarrow \frac{d Y_{t}}{d G_{t}}=\frac{Y_{t}^{n}}{2\left(1-\gamma_{t}\right) Y_{t}^{n}+Y_{t}^{n} \gamma_{t}}=\frac{1}{2-\gamma_{t}}
\end{aligned}
$$

In (b):

$$
d Y_{t} / d \gamma_{t}=Y \frac{d y_{t}}{d f_{t}} \frac{d f_{t}}{d \gamma_{t}}=Y \frac{1}{1+\phi_{y}} \frac{1}{1-\gamma}=\frac{1}{1+\phi_{y}(1-\gamma)}
$$
}
{
\item 
In (a) mandatory policy is neutral (no effect of $\phi_{y}$ on $Y_{t}^{n}$).

In (b), if monetary policy is more countercyclical ($\phi_{y}$ bigger), the spending multiplier is smaller. Reason: effect of government spending is partly offset by the rise in the interest rate.
}
\end{enumerate}
}